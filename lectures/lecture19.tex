\chapter{Oct.~29 --- Urysohn Metrization, Part 3}

\section{Proof of Urysohn's Metrization Theorem}

\begin{theorem}[Urysohn metrization theorem]
  If a topological space $X$ is normal and second countable, i.e. it
  has a countable basis, then
  $X$ is metrizable.
\end{theorem}

\begin{proof}
  The idea is to embed $X$ into the Hilbert cube
  $[0, 1]^\N$ with the product topology. Note that
  $[0, 1]^\N$ is metrizable, e.g. take the metric
  \[
    d(x, y) = \sum_{i = 1}^\infty \frac{1}{2^i} 
    d_i(x_i, y_i),
  \]
  where $d_i$ is a metric on $[0, 1]$. We saw last
  time that $d$ is a metric, and it also induces
  the product topology (check this as an exercise,
  we can throw away the tail since the series
  converges).

  Now we give the embedding. Since $X$ is second
  countable, let $\{B_n\}_{n = 1}^\infty$ be a
  basis for the topology on $X$. For each pair of
  indices $n < m$ with $\overline{B_n} \subseteq B_m$,
  note that $\overline{B_n} \cap B_m^c = \varnothing$.
  Now $\overline{B_n}$ and $B_m^c$ are also
  closed, so by Urysohn's lemma there exists a
  continuous
  function $g_{n, m} : X \to [0, 1]$ such that
  $g_{n, m}(\overline{B_n}) = \{1\}$ and
  $g_{n, m}(B_m^c) = \{0\}$. Now define
  $f : X \to f(X) \subseteq [0, 1]^\N$ (with the
  subspace topology) by
  \[
    f(x) = (g_{n, m}(x))_{n, m \in \N, \overline{B_n} \subseteq B_m}.
  \]
  Note that these pairs $n, m$ exist since
  there is some basis element $B_m$ which contains
  $x$. Also note that $f(X)$ is metrizable (simply
  obtain a metric from $[0, 1]^\N$ by restricting
  $d$).
  Then $\{x\}$ is closed (we
  assume $T_1$) and $B_m^c$ is closed, so by
  normality there exists an open set $U_n$ containing
  $x$ such that $\overline{U_n} \subseteq B_m^c$.
  Then there exists a basis element $B_n \subseteq U_n$
  containing $x$ with the same property.

  Now we show that $f : X \to f(X)$ is a
  topological embedding. We show that:
  \begin{enumerate}[(i)]
    \item $f$ is injective: Assume
      $x \ne y$, and we want to find a pair $n, m$
      such that $g_{n, m}(x) \ne g_{n, m}(y)$.
      Since we assume $X$ is $T_1$, normal implies
      Hausdorff and thus
      we can find disjoint open sets $U_x, U_y$
      with $x \in U_x$ and $y \in U_y$. From this
      we can find $V_x, V_y$ open such that
      \[
        x \in V_x \subseteq \overline{V_x} \subseteq U_x
        \quad \text{and} \quad
        y \in V_y \subseteq \overline{V_y} \subseteq U_y.
      \]
      We can get such sets by considering
      $\{x\}$ and $U_x^c$ and using normality. Now
      there exists a basis element $B_n$ such that
      $x \in B_n \subseteq V_x$. Then by considering
      $\{x\}$ and $B_n^c$,  using normality we can
      find a basis element $B_m$ such that
      \[
        x \in B_m \subseteq \overline{B_m} \subseteq B_n.
      \]
      Then $g_{n, m}(x) = 1$ (we have
      $x \in \overline{B_m}$) but $g_{n, m}(y) = 0$
      (we have $y \notin B_n$),
      so we get $f(x) \ne f(y)$.
    \item $f$ is continuous: The function
      $f : X \to [0, 1]^\N$ is continuous by the
      universal property of products since
      each $g_{n, m}$ is continuous. Now
      $f(X)$ has the subspace topology, so $f : X \to f(X)$
      is continuous.
    \item $f$ is open (on $f(X)$ with the subspace topology): Fix an open set $B \subseteq X$,
      and we would like to show that
      $f(B) \subseteq f(X) \subseteq [0, 1]^\N$.
      Fix $y = f(x) \in f(B)$.
      Since $f(X)$ is metrizable, it suffices
      to find $\epsilon > 0$ such that whenever
      $z \in f(X)$ and $d(z, y) < \epsilon)$, then
      we have $z \in f(B)$.

      Assume otherwise that this
      does not hold, i.e. for every $\epsilon > 0$
      there exists $z_\epsilon$ such that
      $d(z_\epsilon, y) < \epsilon$ and $z_\epsilon \notin f(B)$.
      Now set $\epsilon_n = 1 / n$, which gives a
      sequence $z_n \in f(X) \subseteq [0, 1]^\N$.
      Since $[0, 1]^\N$ is compact metric space,
      it is sequentially compact, so there exists a
      convergent subsequence $z_{n_k} \to z \in [0, 1]^\N$.
      Since $d(z_{n_k}, y) < 1 / n_k \le 1 / k$ for
      every $k$, so $d(z, y) = 0$, i.e. $z = y \in f(X)$.

      Now let $z_n = f(w_n)$ for some point
      $w_n \in X$, which is unique for each $z_n$
      since $f$ is injective. Note that by
      continuity and injectivity, we have
      $w_n \to x$. Since
      $z_n \notin f(B)$, we must have $w_n \notin B$.
      Then $w_n \in B^c$ and $x$ is a limit point
      of $B^c$, so we must have $x \in B^c$ since
      $B^c$ is closed. Contradiction.
  \end{enumerate}
  Then $f$ is a topological embedding, and so the
  metric on $f(X)$ gives a metric on $X$.
\end{proof}

\begin{remark}
  The idea in the previous proof is to generate
  bump functions similar to
  \[
    f(x) =
    \begin{cases}
      \exp(-1 / (1 - x^2)^2) & \text{if } |x| < 1, \\
      0 & \text{otherwise}
    \end{cases}
  \]
  in analysis. Then the peaks roughly indicate where
  the points in $X$ are.
\end{remark}

\begin{remark}
  Normality can sometimes be difficult to show, but
  any compact Hausdorff space is normal.
\end{remark}

\section{Omissions in Point-Set Topology}
Here are some topics we omitted but may be of
interest:
\begin{itemize}
  \item More on metrization:
    The Nagata-Smirnov and Bing metrization theorems.
  \item Dimension theory: The \emph{topological dimension}
    of a space, for instance we would expect that
    the dimension of $\R^n$ is $n$. This is distinct
    from the notion of \emph{Hausdorff dimension}.
  \item Topological manifolds: We can cover
    a space with an \emph{atlas} consisting
    of of \emph{charts}
    each homeomorphic to $\R^n$. The charts should
    be compatible on their intersection, i.e.
    $(f \circ g^{-1})|_{A \cap B} = \id$.
  \item The Baire category theorem.
\end{itemize}
