\chapter{Nov.~21 --- Fundamental Group of the Circle}

\section{More on Functors}
\begin{example}
  Let $X, Y$ be topological spaces and
  $f : X \to Y$ be a continuous map.
  The \emph{forgetful functor} from the category
  of topological spaces $\mathsf{Top}$
  to the category of sets $\mathsf{Set}$
  is the one which maps $X, Y$ as topological
  spaces to $X, Y$ as sets and $f$ as a continuous map
  to $f$ as a just a function. Essentially
  this functor forgets the topological structure.
\end{example}

\begin{remark}
  Recall that a map $f : X \to Y$ in
  $\mathsf{Top}$ induces a map
  $f_* : \pi_1(X) \to \pi_1(Y)$ in $\mathsf{Grp}$.
\end{remark}

\begin{prop}
  Let $X, Y$ be topological spaces and
  $f : X \to Y$ be a continuous map. Then
  the induced map $f_* : \pi_1(X) \to \pi_1(Y)$
  is a group homomorphism, i.e.
  \[
    f_*([\gamma] * [\delta])
    = f_*([\gamma]) * f_*([\delta])
  \]
\end{prop}

\begin{proof}
  Consider $\gamma, \delta$ as curves.
  Then applying $f$ on each part of
  $\gamma * \delta : [0, 1] \to X$ gives
  \[
    (f \circ \gamma) * (f \circ \delta)
    = f \circ (\gamma * \delta)
    : [0, 1] \to Y.
  \]
  To move to the fundamental group, we must
  take equivalence classes. This gives
  \[
    [f \circ \gamma] * [f \circ \delta]
    = [(f \circ \gamma) * (f \circ \delta)]
    = [f \circ (\gamma * \delta)],
  \]
  where the first equality follows from the
  definition of the group operation in $\pi_1$.
  Now if we define the induced
  map via $f_*([\gamma]) = [f \circ \gamma]$,
  the above equality reads
  $f_*([\gamma]) * f_*([\delta]) = f_*([\gamma] * [\delta])$.
\end{proof}

\begin{theorem}[Borsuk-Ulam]
  For any continuous function $f : S^2 \to \R^2$,
  there exist two antipodal points
  $x, -x \in S^2$ such that $f(x) = f(-x)$.
\end{theorem}

\begin{proof}
  Assume not. Then we can define a map
  $g : S^2 \to \R^2$ by
  \[
    g(x) = f(x) - f(-x)
  \]
  which is never zero. Since $g(x) \ne 0$, we can
  define $h : S^2 \to S^1$ by
  \[
    h(x) = \frac{g(x)}{\|g(x)\|}.
  \]
  The idea now is that by looking at the equator
  of $S^2$, we obtain a nontrivial composition
  $S^1 \to S^2 \to S^1$. Thus we can argue for
  a contradiction as in the
  no retract theorem since $\pi_1(S^2) = 0$.
\end{proof}

\section{Fundamental Group of the Circle}

\begin{remark}
  To move towards finding the fundamental group of
  $S^1$, one can consider \emph{covering spaces}.
  Consider $\R$ realized as a line winding
  around in a circle above $S^1$, and consider
  the projection onto $S^1$. Note that although the
  basepoint $x_0 \in S^1$ has infinitely many
  preimages, once we fix a preimage, then
  the projection is a local homeomorphism.
  One can locally follow movements in $S^1$
  by moving in $\R$.

  In particular, curves in $S^1$ lift uniquely
  to curves
  in $\R$. Furthermore, homotopies of loops in $S^1$
  also lift to homotopies fixing endpoints in $\R$.
  This allows
  us to distinguish between homotopy classes
  of loops in $S^1$ by looking at where the
  corresponding curves end in $\R$. One can also
  easily see in this way that $\pi_1(S^1)$ is
  abelian, since concatenating loops in this
  fashion in $\R$ is commutative.

  With some more technicalities, one can show
  that $\pi_1(S^1) = \Z$. Read more in Munkres
  or Hatcher.
\end{remark}

\begin{example}
  Recall that the $2$-torus
  $\mathbb{T}^2 = S^1 \times S^1$ was by
  constructed by gluing the opposite sides of a
  square. One can show in general that
  if $X \times Y$ is given the product topology, then
  \[
    \pi_1(X \times Y) \cong \pi_1(X) \times \pi_1(Y).
  \]
  This then tells us that
  the fundamental group of the torus is
  \[
    \pi_1(\mathbb{T}^2) = \pi_1(S^1 \times S^1)
    = \pi_1(S^1) \times \pi_1(S^1) = \Z \times \Z.
  \]
  The generators are precisely the loops going
  around the two circles in the torus. In particular,
  these loops are from different
  components in the fundamental group, so
  they cannot be in the same homotopy class.
  This calculation
  also shows that the fundamental group of the
  torus is abelian.
\end{example}

\begin{example}
  Note that a cylinder $C$ is homotopy equivalent
  to a circle, so $\pi_1(C) = \pi_1(S^1) = \Z$.
  One can also think about the cylinder as
  $C = S^1 \times I$. Since $\pi_1(I) = 0$, we have
  \[
    \pi_1(C) = \pi_1(S^1 \times I) = \pi_1(S^1) \times \pi_1(I) = \Z \times \{0\} \cong \Z.
  \]
\end{example}

\begin{example}
  Consider the M\"obius strip $M$. Note that
  although $M$ is not homeomorphic to either the
  circle or the cylinder, it is homotopy equivalent
  to the circle. Consider squashing the
  fundamental polygon (square) of the M\"obius strip
  to the circle in the center.
  Thus $\pi_1(M) = \pi_1(S^1) = \Z$.
\end{example}

\section{Projective Spaces}
\begin{definition}
  The \emph{projective space} $\mathbb{RP}^n$ is
  the space of lines through the origin in
  $\R^{n + 1}$, i.e.
  \[
    \mathbb{RP}^n = (\R^{n + 1} \setminus \{0\}) / {\sim},
  \]
  where the relation $\sim$ is defined
  by $x \sim cx$ for some nonzero $c \in \R$.
\end{definition}

\begin{remark}
  The fundamental group of
  $\mathbb{RP}^n$ is $\Z/2\Z$ for $n \ge 2$.
  This is because repeating a curve in this
  space is essentially
  the same as negating it by the above equivalence
  relation.
\end{remark}
