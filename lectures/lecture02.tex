\chapter{Aug.~22 --- Topology, Basis, Continuity}

\section{Topological Spaces}
\begin{definition}
  A \emph{topology} $\mathcal{T} \subseteq \mathcal{P}(X)$ is a
  collection of sets such that
  \begin{enumerate}[(i)]
    \item $\varnothing, X \in \mathcal{T}$,
    \item for any index set $I$, if $\{s_i\}_{i \in I} \subseteq \mathcal{T}$,
      then $\bigcup_{i \in I} s_i \in \mathcal{T}$ (closure under arbitrary union),
    \item and if $\{s_i\}_{i = 1}^n \subseteq \mathcal{T}$, then
      $\bigcap_{i = 1}^n s_i \in \mathcal{T}$ (closure under finite intersection).
  \end{enumerate}
  A set with a topology, i.e. a pair
  $(X, \mathcal{T})$,
  is called a \emph{topological space}.
  Elements of $\mathcal{T}$ are called
  \emph{open sets}.
\end{definition}

\begin{example}
  The following are examples of topologies on a set $X$:
  \begin{itemize}
    \item The trivial topology: $\mathcal{T} = \{\varnothing, X\}$.
    \item The discrete topology: $\mathcal{T} = \mathcal{P}(X)$.\footnote{Note that the discrete topology is induced by the discrete metric.}
    \item If $(X, d)$ is a metric space, then
      $\mathcal{T} = \{\text{collection of metrically open sets}\}$
      is a topology on $X$.
  \end{itemize}
\end{example}

\begin{remark}
  Not every topology is induced by a metric.
  For instance consider the trivial topology
  on $\R$.
\end{remark}

\section{Basis for a Topology}
\begin{definition}
  A collection $\mathcal{B} \subseteq \mathcal{P}(X)$
  is called a \emph{basis} if
  \begin{enumerate}[(i)]
    \item $\bigcup_{b \in \mathcal{B}} b = X$, i.e.
      $\mathcal{B}$ is a covering of $X$,
    \item and if $x \in b_1 \cap b_2$ for any
      $b_1, b_2 \in B$, then there exists $b_3 \in \mathcal{B}$
      such that $x \in b_3$
      and $b_3 \subseteq b_1 \cap b_2$.
  \end{enumerate}
\end{definition}

\begin{theorem}
  Given a set $X$ and a basis $\mathcal{B}$, define
  \[
    \mathcal{T}_\mathcal{B} = \left\{\bigcup_{i \in I} s_i \mid \text{$I$ is any index set and $\{s_i\}_{i \in I} \subseteq \mathcal{B}$}\right\}.
  \]
  Then $\mathcal{T}_\mathcal{B}$ is a topology on $X$.
\end{theorem}

\begin{proof}
  First observe that $\varnothing, X \in \mathcal{T}_\mathcal{B}$:
  Picking $I = \varnothing$ gives
  $\bigcup_{i \in I} s_i = \varnothing \in \mathcal{T}_\mathcal{B}$
  and picking $I = \mathcal{B}$ gives
  $\bigcup_{b \in \mathcal{B}} b = X \in \mathcal{T}_\mathcal{B}$
  by the covering property of a basis.

  Now assume $\{s_i\}_{i \in I} \subseteq \mathcal{T}_\mathcal{B}$.
  For each $i \in I$, we have $s_i \in \mathcal{T}_\mathcal{B}$
  and so there exists an index set $J_i$
  such that $s_i = \bigcup_{j \in J_i} b_j$, where
  the $b_j \in \mathcal{B}$. Then
  \[
    \bigcup_{i \in I} s_i = \bigcup_{i \in I} \bigcup_{j \in J_i} b_j,
  \]
  which is a union of elements of
  $\mathcal{B}$ and hence is in
  $\mathcal{T}_{\mathcal{B}}$.

  Finally assume $\{s_i\}_{i = 1}^n \subseteq \mathcal{T}_\mathcal{B}$.
  Now as each $s_i \in \mathcal{T}_\mathcal{B}$,
  there exists $J_i$ such that
  $s_i = \bigcup_{j \in J_i} b_j$.
  Then
  \[
    \bigcap_{i = 1}^n s_i = \bigcap_{i = 1}^n \bigcup_{j \in J_i} b_j.
  \]
  Now assume $x \in \bigcap_{i = 1}^n s_i = \bigcap_{i = 1}^n \bigcup_{j \in J_i} b_j$.
  For each $1 \le i \le n$, there exists
  $j_i \in J_i$ such that $x \in b_{j_i}$.
  Hence $x \in \bigcap_{i = 1}^n b_{j_i}$.
  Now by induction on the intersection property of a basis,
  we can find $b_x \in \B$ with
  \[
    x \in b_x \subseteq \bigcap_{i = 1}^n b_{j_i}
  \]
  Also observe that
  \[\bigcap_{i = 1}^n b_{j_i} \subseteq \bigcap_{i = 1}^n \bigcup_{j \in J_i} b_j
    = \bigcap_{i = 1}^n s_i
  \]
  by construction,
  so we may write
  \[
    \bigcap_{i = 1}^n s_i = \bigcup_{x \in \bigcap_{i = 1}^n s_i} b_x \in \mathcal{T}_\mathcal{B}
  \]
  as a union of elements of $\mathcal{B}$.
\end{proof}

\begin{definition}
  A \emph{subbasis} $\mathcal{B} \subseteq \mathcal{P}(X)$ is a collection
  of sets such that
  $\bigcup_{b \in \mathcal{B}} b = X$.
\end{definition}

\begin{remark}
  One may define a basis $\widetilde{\mathcal{B}}$ from a
  subbasis $\mathcal{B}$ by adding all finite intersections
  of elements of $\mathcal{B}$.
  We get the covering property for free and adding
  the finite intersections gives us the
  intersection property of a basis.
\end{remark}

\begin{example}
  For $\R$ with the Euclidean metric, the following
  are bases for the standard topology:
  \begin{itemize}
    \item $\{B_R(x) \mid x \in \R, R > 0\}$.
    \item $\{B_R(x) \mid x \in \R, R > 0, R \in \Q\}$.
      For this use the fact that $\Q$ is dense in $\R$.
  \end{itemize}
  In particular this shows that a basis for a topology
  is not unique in general.
\end{example}

\section{Continuous Functions}

\begin{definition}
  Let $(X, \mathcal{T}_X)$ and $(Y, \mathcal{T}_Y)$
  be two topological spaces. A function $f : X \to Y$
  is called \emph{continuous} if for any
  $O \in \mathcal{T}_Y$, we have
  $f^{-1}(O) \in \mathcal{T}_X$, i.e. the
  preimage of an open set is open.\footnote{Recall that $f^{-1}(O) = \{x \in X \mid f(x) \in O\}$.}
\end{definition}

\begin{example}
  Let $X$ be equipped with the trivial
  topology $\{\varnothing, X\}$ and let
  $\R$ be equipped with the standard topology.
  Then the only continuous functions $f : X \to \R$
  are the constant functions $f : x \mapsto c$
  for fixed $c \in \R$. To see this, observe that
  \begin{itemize}
    \item $x \mapsto c$ is continuous since
      any open set in $\R$ either contains $c$ or does not,
      and so the preimage is either $X$ or $\varnothing$.
    \item Suppose $f(x_1) = y_1$ and $f(x_2) = y_2$.
      Let $\epsilon = |y_1 - y_2|$ and observe that
      $x_1 \in f^{-1}(B_\epsilon(y_1))$ while
      $x_2 \notin f^{-1}(B_\epsilon(y_1))$, so
      $f^{-1}(B_\epsilon(y_1))$ is not open
      in $X$ despite $B_\epsilon(y_1)$ being open in $\R$.
  \end{itemize}
\end{example}

\begin{example}
  Let $X$ have the discrete topology
  $\mathcal{T} = \mathcal{P}(X)$ and let
  $\R$ have the standard topology.
  Then all functions $X \to \R$ are continuous
  since any preimage is a subset of $X$ and thus
  in $\mathcal{P}(X)$.
\end{example}

\begin{remark}
  In a way, the trivial topology has too few
  open sets while the discrete topology has too many.
\end{remark}

\begin{definition}
  Two topological spaces $(X, \mathcal{T}_X)$ and
  $(Y, \mathcal{T}_Y)$ are
  \emph{topologically equivalent} or
  \emph{homeomorphic} if there exists a
  bijection $f : X \to Y$ such that
  $f$ and $f^{-1}$ are continuous.
\end{definition}

\begin{remark}
  A bijective function $f$ being continuous does
  not necessarily imply that its inverse $f^{-1}$ is.
\end{remark}

\begin{example}
  Consider $(-\pi / 2, \pi / 2)$ equipped with the
  Euclidean metric. This is homeomorphic to $\R$
  equipped with the Euclidean metric.\footnote{Note that $(-\pi / 2, \pi / 2)$ is bounded while $\R$ is not.} One
  homeomorphism is given by $\tan : (-\pi / 2, \pi / 2) \to \R$.
\end{example}
