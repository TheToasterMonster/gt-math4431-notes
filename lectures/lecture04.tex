\chapter{Aug.~29 --- Connectedness}

\section{Connected Spaces}
\begin{definition}
  A \emph{separation} of a topological space $X$
  is two open, nonempty sets $U, V \subseteq X$ such
  that $X = U \cup V$ and $U \cap V = \varnothing$.
  A space is called \emph{connected}
  if there is no separation of the space.
\end{definition}

\begin{prop}
  If $X$ is separated, i.e.
  $X = U \cup V$ with $U, V$ open and disjoint,
  then $U$ and $V$ are both open and closed.
\end{prop}

\begin{proof}
  Observe that $U$ is open by assumption, and we have
  \[
    U^c = X \setminus U = V,
  \]
  which is also open by assumption. Hence
  $U$ is closed. The case for $V$ is identical.
\end{proof}

\begin{example}
  Consider the following:
  \begin{itemize}
    \item The singleton space $\{x\}$ is connected. There
      are no two nonempty, disjoint open sets.
    \item Consider the space $X = \{0, 1\}$. This
      case depends on the choice of topology:
      \begin{enumerate}
        \item With the trivial topology
          $\mathcal{T} = \{\varnothing, X\}$, the
          space is connected.
        \item With the discrete topology
          $\mathcal{T} = \{\varnothing, X, \{1\}, \{0\}\}$,
          $X$ is disconnected since
          $X = \{0\} \cup \{1\}$.
        \item With the topology
          $\mathcal{T} = \{\varnothing, X, \{1\}\}$,
          the space is connected. The only
          nonempty sets $\{1\}, X$ are not disjoint
          and thus there can be no separation.
      \end{enumerate}
  \end{itemize}
\end{example}

\begin{theorem}
  A space $X$ is disconnected if and only if
  there exists a surjective map
  $f : X \to \{0, 1\}$ with the discrete topology.
\end{theorem}

\begin{proof}
  $(\Rightarrow)$
  If $X$ is disconnected, then we may write
  $X = U \cup V$ with $U, V$ open, disjoint, and
  nonempty. Then define
  \[
    f(x) = \begin{cases}
      0 & x \in U, \\
      1 & x \in V,
    \end{cases}
  \]
  which is surjective as $U, V$ are nonempty. To
  see that $f$ is continuous, observe that
  \[
    f^{-1}(\varnothing) = \varnothing, \quad
    f^{-1}(\{0, 1\}) = X, \quad
    f^{-1}(\{0\}) = U, \quad
    f^{-1}(\{1\}) = V,
  \]
  each of which are open. These are all of
  the open sets in the discrete topology, so $f$ is
  continuous.

  $(\Leftarrow)$ Assume there exists a surjective
  and continuous map $f : X \to \{0, 1\}$. Define
  \[
    U = f^{-1}(\{0\}) \quad \text{and} \quad
    V = f^{-1}(\{1\}),
  \]
  which are open since $f$ is continuous. Observe
  that $U, V \ne \varnothing$ since $f$ is surjective.
  Also $U \cap V = \varnothing$ since if there is
  any $x \in U \cap V$, then
  $f(x) = 0$ as $x \in U$ and $f(x) = 1$ as $x \in V$,
  a contradiction.
  Finally, $X = U \cup V$ since $f(x) = 0$ or
  $f(x) = 1$ for every $x \in X$, i.e.
  $x \in U$ or $x \in V$. So $X$ is disconnected.
\end{proof}

\section{Connected Sets}
\begin{definition}
  Let $X$ be a topological space and $S \subseteq X$.
  Then $S$ is called
  \emph{connected} if it is connected in the
  subspace topology.
\end{definition}

\begin{theorem}
  \label{thm:union-connected}
  If $A, B$ are connected sets and
  $A \cap B \ne \varnothing$, then $A \cup B$
  is connected.
\end{theorem}

\begin{proof}
  Assume not. Then there exists a continuous,
  surjective map
  $f : A \cup B \to \{0, 1\}$ with the discrete
  topology. Consider $f|_A : A \to \{0, 1\}$,
  which is continuous in the subspace topology.
  Notice that $f(A)$ cannot be $\{0, 1\}$ since
  otherwise $A$ is disconnected. Without
  loss of generality, assume $f(A) = \{0\}$ since
  $A$ is nonempty. Now consider
  $f|_B : B \to \{0, 1\}$, which is also
  continuous. Similarly, notice that
  $f(B)$ cannot be $\{0, 1\}$. But
  there exists
  $p \in A \cap B$, and $f(p) = 0$ as $p \in A$.
  Then since $p \in B$, we must have $f(B) = \{0\}$.
  But then we get that $f(A \cup B) = \{0\} \ne \{0, 1\}$,
  a contradiction to surjectivity.
\end{proof}

\begin{corollary}
  \label{cor:union-connected}
  A union of connected sets with
  ``common points'' is connected.
\end{corollary}

\begin{proof}
  Run induction (transfinite if the union is infinite)
  using the previous theorem.
\end{proof}

\begin{theorem}
  Closed intervals in $[a, b] \subseteq \R$
  with the metric topology are connected.
\end{theorem}

\begin{proof}
  Assume otherwise that $[a, b] = U \cup V$
  with $U, V$ disjoint, open, and nonempty.
  Assume without loss of generality that
  $a \in U$. Since $V$ is nonempty, there exists
  $c > a$ such that $c \in V$. Now consider
  $[a, c] \subseteq [a, b]$ with
  $U_1 = U \cap [a, c]$ and $V_1 = V \cap [a, c]$.
  By the least upper bound property of $\R$,
  since $U_1$ is nonempty and bounded from above,
  there exists $s = \sup U_1$ with
  $s \le c$. Now either $s \in U_1$ or $s \notin U_1$.

  If $s \in U_1$ (note this implies $s \ne c$), then $s$ is an interior point of
  $U_1$ since $U_1$ is open. So one may find
  a point $t$ such that $t > s$ and $t \in U_1$. But
  then $s$ is no longer an upper bound of $U_1$,
  a contradiction.

  Otherwise $s \notin U_1$. Since $U_1, V_1$
  cover $[a, c]$, we must then have $s \in V_1$ (note this implies $s \ne a$.
  Since $V_1$ is open, $s$ is an interior point
  of $V_1$, and thus there exists $t < s$
  such that $t \in V_1$ and $t$ is an upper bound for
  $U_1$. This contradicts $s$ being the
  least upper bound of $U_1$.

  Since both cases lead to contradictions, we
  conclue that $[a, b]$ must be connected.
\end{proof}

\begin{corollary}
  Open intervals in $\R$ are connected, and
  $\R$ itself is connected.
\end{corollary}

\begin{proof}
  For some $N_0 \ge 1$ (for instance choose
  $N_0 \ge 2 / (b - a)$) we can write
  \[
    (a, b) = \bigcup_{n = N_0}^\infty \left[a + \frac{1}{n}, b - \frac{1}{n}\right],
  \]
  Each of these closed intervals is connected by
  the previous theorem, and thus the union is connected
  by Corollary \ref{cor:union-connected} since
  they overlap.
  Similarly writing $\R = \bigcup_{n = 1}^\infty [-n, n]$
  shows that $\R$ is connected.
\end{proof}

\begin{corollary}[Intermediate value theorem]
  Let $f : [a, b] \to \R$ be a continuous function.
  Then
  for any $f(a) < t < f(b)$, there exists
  $c \in [a, b]$ such that $f(c) = t$.
\end{corollary}

\begin{proof}
  Assume not. We can consider the open sets
  $(-\infty, t)$ and $(t, \infty)$ in $\R$. Then
  $f^{-1}((-\infty, t))$ and $f^{-1}((t, \infty))$
  are open sets since $f$ is continuous. They
  are clearly disjoint (since $f$ must be
  well-defined),
  and also nonempty since
  $a \in f^{-1}((-\infty, t))$ and
  $b \in f^{-1}((t, \infty))$. Also since
  $f^{-1}(\{t\}) = \varnothing$ by assumption,
  \[
    [a, b] = f^{-1}((-\infty, t)) \cup f^{-1}((t, \infty)).
  \]
  But this is a separation of $[a, b]$, a
  contradiction since $[a, b]$ is connected.
\end{proof}

\begin{prop}
  The open interval $(0, 1)$ is not
  homeomorphic to the closed interval $[0, 1]$.
\end{prop}

\begin{proof}
  Removing any point from $(0, 1)$ disconnects
  it, but $[0, 1) = [0, 1] \setminus \{1\}$
  remains connected.\footnote{To see that $[0, 1)$ is connected, we can write $[0, 1) = \bigcup_{n = 2}^\infty [0, 1 - 1 / n]$.}
\end{proof}

\begin{prop}
  The real line $\R$ is not homeomorphic
  to the plane $\R^n$ for any $n \ge 2$.
\end{prop}

\begin{proof}
  Removing a point from $\R$ disconnects it but
  the same is not true for $\R^n$ when $n \ge 2$.
\end{proof}
