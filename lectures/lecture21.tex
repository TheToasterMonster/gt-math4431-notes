\chapter{Nov.~5 --- The Fundamental Group}

\section{Homotopy}
\begin{remark}
  We will always write $I = [0, 1]$, so that a
  homotopy is a map $H : X \times I \to Y$. We will
  also write $f \sim g$ to indicate that $f$ and
  $g$ are homotopic.
\end{remark}

\begin{remark}
  Homotopy is an equivalence relation on the set of
  continuous maps from $X$ to $Y$, and homotopy
  equivalence is an equivalence relation on the set
  of topological spaces.
\end{remark}

\begin{prop}
  Let $S \subseteq \R^n$ be a convex set with
  non-empty interior. Then $S$ is homotopic to the
  closed $n$-dimensional ball $\overline{B^n}$.
\end{prop}

\begin{proof}
  Since $S$ has non-empty interior, there exists
  an open ball $B_r^n$ inside it. We may assume
  $\overline{B_r^n}$ is contained inside $S$ (e.g. by
  halving the radius). Denote the center of $\overline{B_r^n}$
  by $x_0$. For any $y \in S$, by convexity the
  entire line segment from $x_0$ to $y$ is contained
  in $S$. Using this idea, we
  define $H : S \times I \to S$ by
  \[
    H(y, t) =
      x_0 + (y - x_0)g(y, t),
  \]
  where
  \[
    f(y) = \min\left\{\frac{r}{\|y - x_0\|}, 1\right\}
    \quad\text{and}\quad
    g(y, t) = \begin{cases}
      1 + t(f(y) - 1) / f(y) & 0 \le t \le f(y), \\
      f(y) & t \ge f(y).
    \end{cases}
  \]
  Note that these
  functions are continuous as a composition
  of continuous maps. We can also verify
  \[
    H(y, 0) = x_0 + (y - x_0) g(y, 0) = x_0 + (y - x_0) =
    y = \id_S(y)
  \]
  and
  \[
    H(y, 1) = x_0 + (y - x_0) g(y, 1)
    = x_0 + (y - x_0) \min\left\{1, \frac{r}{\|y - x_0\|}\right\}.
  \]
  If the minimum is $1$, then $r \ge \|y - x_0\|$
  and so $y \in \overline{B^n_r}$. Thus we get
  \[
    H(y, 1) = x_0 + (y - x_0) = y = \id_{\overline{B_r^n}}(y).
  \]
  Otherwise the minimum is $r / \|y - x_0\|$, so
  we get
  \[
    H(y, 1) = x_0 + (y - x_0) \frac{r}{\|y - x_0\|} \in \overline{B^n_r}.
  \]
  So $H$ is a homotopy between $S$ and $\overline{B_r^n}$, as desired.
\end{proof}

\begin{remark}
  A simpler way to show this is to argue using that
  homotopy is an equivalence relation. Shrink both
  $S$ and $\overline{B_r^n}$ to a point (both
  are convex, so simply shrink along the line segment
  $x_0 + (y - x_0)(1 - t)$), so they must be
  homotopic to each other by transitivity.
\end{remark}

\begin{definition}
  A space $X$ is called \emph{contractible} if
  $\id_X \sim \text{constant}$.
\end{definition}

\begin{example}
  The space $\R^n$ is contractible. Define
  $H(x, t) = (1 - t) x$ as a homotopy to the origin.
\end{example}

\begin{theorem}
  The unit circle $S^1 \subseteq \R^2$ is not
  contractible.
\end{theorem}

\begin{remark}
  This theorem
  is actually quite hard to show, we will do
  this eventually.
\end{remark}

\begin{remark}
  This theorem also leads to a
  proof of the \emph{Jordan curve theorem}, which
  states that any simple closed curve in $\R^2$
  divides the plane into two regions, one bounded
  and one unbounded. The idea is to first
  show that the
  image of any such curve in is
  homotopic to the circle (this itself is
  difficult, and is what Jordan originally proved),
  and then show the result only for the circle.
\end{remark}

\section{The Fundamental Group}
\begin{remark}
  We will be interested in homotopy of curves
  $\gamma : [0, 1] \to X$.
\end{remark}

\begin{definition}
  Given two curves $\gamma_1, \gamma_2 : [0, 1] \to X$
  with $\gamma_1(1) = \gamma_2(0)$, their
  \emph{concatenation} is\footnote{The concatenation is continuous by the pasting lemma since $\gamma_1(1) = \gamma_2(0)$.}
  \[
  \gamma_1 * \gamma_2(t) = \begin{cases}
    \gamma_1(2t) & 0 \le t \le 1/2, \\
    \gamma_2(2t - 1) & 1/2 \le t \le 1.
  \end{cases}
  \]
\end{definition}

\begin{definition}
  For a space $X$, its \emph{fundamental group (with base point $x_0 \in X$)}
  is
  \[
    \pi_1(X, x_0)
    = \{\text{continuous curves $\gamma : I \to X$ such that $\gamma(0) = \gamma(1) = x_0$}\} / {\sim},
  \]
  where $\sim$ is the homotopy of curves fixing
  endpoints.\footnote{Another way to ensure the fixed endpoint condition is to consider maps $\gamma : \R / \Z \to X$ which hit $x_0$.} The
  group operation on $\pi_1(X, x_0)$ is defined by
  \[
    [\gamma_1] * [\gamma_2] = [\gamma_1 * \gamma_2],
  \]
  where $[\gamma]$ denotes the equivalence class of
  $\gamma$ under $\sim$.
\end{definition}

\begin{prop}
  The group operation $*$ on $\pi_1(X, x_0)$ is well-defined.
\end{prop}

\begin{proof}
  Replace $\gamma_1$ with $\widetilde{\gamma_1}$
  and $\gamma_2$ with $\widetilde{\gamma_2}$, where
  $\gamma_1 \sim \widetilde{\gamma_1}$ with
  homotopy $H_1$ and
  $\gamma_2 \sim \widetilde{\gamma_2}$ with homotopy
  $H_2$. We want
  $\gamma_1 * \gamma_2 \sim \widetilde{\gamma_1} * \widetilde{\gamma_2}$.
  Define the homotopy $H : I \times I \to X$ by
  \[
    H(t, s) =
    \begin{cases}
      H_1(t, 2s) * \gamma_2 & 0 \le s \le 1/2, \\
      \widetilde{\gamma_1} * H_2(t, 2s - 1) & 1/2 \le s \le 1.
    \end{cases}
  \]
  The idea is that we take the homotopy $H_1$ first
  followed by the homotopy $H_2$. Note that
  for $s \ne 1 / 2$, the map $H$ is a composition
  of continuous maps, hence continuous. At
  $s = 1 / 2$, we have
  \[
    H_1(t, 1) * \gamma_2 =
    \widetilde{\gamma_1} * \gamma_2
    = \widetilde{\gamma_1} * H_2(t, 0)
  \]
  since $H_1$ is a homotopy from $\gamma_1$ to
  $\widetilde{\gamma_1}$ and $H_2$ is a homotopy
  from $\gamma_2$ to $\widetilde{\gamma_2}$.
  Now we check that
  \[
    H(t, 0) = H_1(t, 0) * \gamma_2
    = \gamma_1 * \gamma_2 \quad \text{and} \quad
    H(t, 1) = \widetilde{\gamma_1} * H_2(t, 1)
    = \widetilde{\gamma_1} * \widetilde{\gamma_2}.
  \]
  So $H$ is a homotopy from $\gamma_1 * \gamma_2$
  to $\widetilde{\gamma_1} * \widetilde{\gamma_2}$,
  and the group operation on
  $\pi_1(X, x_0)$ is well-defined.
\end{proof}
