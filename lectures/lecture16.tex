\chapter{Oct.~10 --- Quotient Spaces}

\section{Examples of Quotient Spaces}

\begin{example}[Line with two origins]
  Take two copies of $\R$, i.e.
  $X = \R \times \{0, 1\}$,
  and define the equivalence relation
  $(x, 0) \sim (x, 1)$ for all $x \ne 0$. Then
  $X / {\sim}$ is not Hausdorff (consider
  the points $x = (0, 0)$ and $y = (0, 1)$, any
  neighborhoods around $x, y$ must intersect), even
  though $X$ is.
\end{example}

\begin{remark}
  The above example shows that the quotient of a
  Hausdorff space need not be Hausdorff.
\end{remark}

\begin{example}[Cylinder]
  Take a square in $\R^2$ and identify the
  left and right sides (in the same direction). Then
  the quotient space is homeomorphic to a cylinder.
\end{example}

\begin{example}[M\"obius strip]
  Now take a square in $\R^2$ and identify the
  left and right sides in opposite directions. The
  resulting quotient space is still a 2D surface,
  but it is not homeomorphic to a cylinder. In
  particular, the M\"obius strip is not orientable.
\end{example}

\begin{example}[Torus]
  Take a square in $\R^2$ and identify the
  left and right sides, as well as the top and
  bottom sides (both in the same direction). The
  resulting quotient space is
  homeomorphic to a torus. In particular, the
  torus has no boundary.
\end{example}

\begin{example}[Klein bottle]
  Take the square in $\R^2$, identify the left
  and right (in the same direction), but identify
  the top and bottom in opposite directions. This
  quotient space is the Klein bottle, which cannot
  be embedded in $\R^3$ (but can embed in $\R^4$). The
  Klein bottle is also not orientable.
\end{example}

\begin{example}[Cone]
  Take the square in $\R^2$, identify the left
  and right sides (in the same direction), and
  collapse the bottom side to a single point.
  The quotient is a cone, which is homeomorphic
  to a disk.
\end{example}

\begin{remark}
  We can define the \emph{connected sum} $S_1 \con S_2$ of
  two surfaces $S_1, S_2$ by
  removing a disk from each surface, and then
  gluing the boundaries of these removed disks
  together.
\end{remark}

\section{Properties of Quotient Spaces}

\begin{theorem}
  Given a surjective map $p : X \to Y$,
  the set of $U \subseteq Y$ such that
  $p^{-1}(U)$ is open defines a topology on $Y$.
\end{theorem}

\begin{proof}
  First, for $\varnothing \in Y$, we have
  $p^{-1}(\varnothing) = \varnothing \in X$ is open,
  so $\varnothing \in Y$ is open. Similarly,
  $p^{-1}(Y) = X$ is open, so $Y$ is open.
  Thus $\varnothing, Y$ are both open.

  Now fix an arbitrary union $\bigcup_{\alpha \in I} U_{\alpha} \subseteq Y$ of open sets.
  Then we have
  \[
    p^{-1}\left(\bigcup_{\alpha \in I} U_{\alpha}\right)
    = \bigcup_{\alpha \in I} p^{-1}(U_{\alpha}).
  \]
  These sets are open in $X$ by assumption, so their
  union is open as well. Thus
  $\bigcup_{\alpha \in I} U_{\alpha}$ is also open.

  Finally, consider a finite intersection
  $\bigcap_{i = 1}^n U_i \subseteq Y$ of open sets.
  Similarlly we see that
  \[
    p^{-1}\left(\bigcap_{i = 1}^n U_i\right)
    = \bigcap_{i = 1}^n p^{-1}(U_i).
  \]
  Again these are open in $X$ by assumption, so their
  finite intersection is open. So
  $\bigcap_{i = 1}^n U_i$ is also open.
\end{proof}

\begin{corollary}
  If $X$ is compact, then $X / {\sim}$ with
  the quotient topology is compact.
\end{corollary}

\begin{proof}
  We have $X / {\sim} = p(X)$, so $X / {\sim}$ is
  the continuous image of a compact set.
\end{proof}

\begin{theorem}[Universal property of quotient spaces]
  Let $X, Y$ be topological spaces and $\sim$ an
  equivalence relation on $X$. Let $f : X \to Y$ be
  a continuous map and
  further suppose
  that $f(x) = f(y)$ whenever $x \sim y$. Then
  the following diagram commutes:
  \[
    \begin{tikzcd}
      X \ar[r, "f"] & Y \\
      X / {\sim} \ar[ur, "\widetilde{f}"', dashed] \ar[u, "p"] &
    \end{tikzcd}
  \]
  i.e. there exists a unique function
  $\widetilde{f} : X / {\sim} \to Y$
  such that $f = \widetilde{f} \circ p$.
\end{theorem}

\begin{proof}
  For $[x] \in X / {\sim}$, define
  $\widetilde{f}([x]) = f(x)$. This is well-defined
  by assumption since $f$ respects $\sim$. To
  show that $\widetilde{f}$ is continuous, consider
  an open set $U \subseteq Y$. Then
  \[
    \widetilde{f}^{-1}(U) = p(f^{-1}(U)),
  \]
  which is open since $f$ is continuous and
  $p$ is a quotient map (notice $f^{-1}$
  gives all of $[x]$ if it gives $x$).
\end{proof}

