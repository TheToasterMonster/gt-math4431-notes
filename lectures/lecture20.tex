\chapter{Oct.~31 --- Algebraic Topology}

\section{Introduction to Algebraic Topology}

\begin{remark}
  We previously showed, using connectedness, that
  $\R$ cannot be homeomorphic to $\R^n$ for any
  $n \ge 2$. How can we show that $\R^2$ and
  $\R^3$ are not homeomorphic?

  One attempt might be to use a similar idea and
  try removing a line from $\R^2$, thus disconnecting
  it. But the image of
  a line need not still be a line (unlike a point, whose
  image is always a point). For instance,
  space-filling curves exist, so it is not obvious
  that the image cannot also disconnect $\R^3$.

  Another example is the cylinder and the M\"obius
  strip. How can we show that these surfaces are
  not homeomorphic? They are both of dimension $2$,
  connected, compact, etc. One way in this case
  is to consider normal maps, and show that the cylinder
  is orientable whereas the M\"obius strip is not.

  Algebraic topology provides new algebraic invariants
  (e.g. the fundamental group)
  that allow us to distinguish spaces in these cases.
  Distinguishing these algebraic invariants is
  often a much easier task.
\end{remark}

\section{Homotopy}
\begin{definition}
  Two maps $f, g : X \to Y$ are
  \emph{homotopic} if there exists a \emph{homotopy}
  $H : X \times [0, 1] \to Y$ which satisfies
  $H(x, 0) = f(x)$ and $H(x, 1) = g(x)$ for all
  $x \in X$.
\end{definition}

\begin{remark}
  It is helpful to think of the parameter
  $t \in [0, 1]$ as time, and we move from $f$ to $g$.
\end{remark}

\begin{example}
  Here are some examples:
  \begin{enumerate}
    \item In $\R$, the maps $f(x) = x$ and
      $g(x) = 0$ are homotopic. Define $H : \R \times [0, 1] \to \R$
      by
      \[
        H(x, t) = (1 - t)f(x) + t g(x) = (1 - t)x.
      \]
      In fact, this method (linear interpolation) works for any continuous
      $f : \R \to \R$, since $\R$ is
      \emph{convex}.
    \item If $Y$ is path-connected, then any
      two constant maps $f_1, f_2 : X \to Y$ are
      homotopic. By path-connectedness, there is a
      curve $\gamma : [0, 1] \to Y$ with $\gamma(0) = f_1(x)$
      and $\gamma(1) = f_2(x)$, then
      \[
        H(x, t) = \gamma(t)
      \]
      is a homotopy between $f_1$ and $f_2$.
    \item Let $f : X \to X$ be the identity map,
      which we will use to describe the shape of $X$.
      Now we would like a \emph{skeleton} (like a
      wireframe) $g : X \to Y \subseteq X$ such that
      $g|_Y \equiv \id_Y$ and $g$ is homotopic
      to $f$. Note that skeletons need not be unique
      or homeormorphic
      (consider a two-holed donut in $\R^2$).
    \item A circle in $\R^2$ with a line segment
      protruding from it is homotopic to a normal circle.
      In some sense, the important feature of the
      circle from the perspective of homotopy is
      that it has a hole.
  \end{enumerate}
\end{example}

\begin{remark}
  Homotopy allows us to capture the idea of holes in
  a space.
\end{remark}

\begin{definition}
  Two spaces $X, Y$ are said to be \emph{homotopy equivalent}
  if there are continuous maps $f : X \to Y$ and
  $g : Y \to X$ such that $f \circ g$ is homotopic
  to $\id_Y$ and $g \circ f$ is homotopic to $\id_X$.
\end{definition}

\begin{remark}
  Homotopy equivalence is a strictly weaker
  (more general) notion than homeomorphism.
\end{remark}
