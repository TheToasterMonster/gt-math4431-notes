\chapter{Oct.~24 --- Urysohn Metrization, Part 2}

\section{Urysohn's Metrization Theorem, Continued}
\begin{lemma}[Urysohn's lemma]
  Let $X$ be a normal space, and $A, B \subseteq X$
  closed and disjoint. Then there exists
  a continuous map $f : X \to [0, 1]$ such that
  $f|_A \equiv 0$ and $f|_B \equiv 1$.
\end{lemma}

\begin{proof}
  We continue from last time. We had
  open sets $U_p$ for $p \in P = \Q \cap [0, 1]$ and
  $f : X \to \R$ given by
  \[
    f(x)  = \inf\{p \in P : x \in U_p\},
  \]
  and we must now show that $f$ is continuous. To
  do this, fix $x \in X$ and let $\epsilon > 0$. We
  must find an open set $U \ni x$ such that
  $f(U) \subseteq (f(x) - \epsilon, f(x) + \epsilon)$.
  Since $\Q$ is dense in $\R$, we can find
  $r, s \in \Q$ such that
  \[
    f(x) - \epsilon < r < f(x) < s < f(x) + \epsilon.
  \]
  Set $V = U_s \setminus \overline{U_r}$, which
  is open since $U_s$ is open and
  $\overline{U_r}$ is closed. Note that $x \in U_s$
  since $f(x) < s$. Now also
  $x \notin \overline{U_r}$ since otherwise we have
  $x \in \overline{U_r} \subseteq U_t$ for some
  $r < t < f(x)$, which is impossible since this
  implies $f(x) \le t < f(x)$. So we see that
  $x \in V$. Now pick any $y \in V$, and we show that
  \[
    f(x) - \epsilon < r \le f(y) \le s < f(x) + \epsilon.
  \]
  Since $y \in V = U_s \setminus \overline{U_r}$,
  we have $r \le f(y)$ since $y \notin \overline{U_r}$
  and $f(y) \le s$ since $y \in U_s$. This
  implies the desired inequality, so $V$ is
  the desired open neighborhood containing $x$. Thus
  $f$ is continuous.
\end{proof}

\begin{theorem}[Urysohn metrization theorem]
  Let $X$ be a normal space which is
  second-countable, i.e. $X$ has countable basis.
  Then $X$ is metrizable.
\end{theorem}

\begin{remark}
  The converse does not hold: Not all metric spaces
  are second-countable (although most are).
\end{remark}

\begin{theorem}[Tietze extension theorem]
  Let $X$ be a normal space, $A \subseteq X$
  closed, and $f : A \to \R$ continuous. Then
  one may extend $f$ to all of $X$ in a continuous
  fashion (preserving the same bound as $f$ if $f$ is
  bounded on $A$).
\end{theorem}

\begin{proof}
  Prove this as an exercise. Use Urysohn's lemma.
\end{proof}

\begin{remark}
  The Tietze extension theorem is a generalization
  of Urysohn's lemma: For disjoint closed sets
  $A, B \subseteq X$, define $f : A \cup B \to \R$
  by $f|_A \equiv 0$ and $f|_B \equiv 1$.
  Then Tietze extension defines $f$ on all of $X$.
\end{remark}

\begin{example}
  Consider Hilbert's cube $X = [0, 1]^\N$, which
  is compact by Tychonoff's theorem. We can define
  a metric on $X$ by
  $d_X(x, y) = \sum_{n = 1}^\infty 2^{-n} d(x_n, y_n)$,
  where $d$ is any metric on $[0, 1]$. Note that $d$
  is bounded since $d$ is continuous and
  $[0, 1] \times [0, 1]$ is compact. To
  see that $d_X$ is indeed a metric, note that:
  \begin{enumerate}[(i)]
    \item We clearly have $d_X(x, y) \ge 0$.
      If $d_X(x, y) = 0$, then $d(x_n, y_n) = 0$
      for every $n$, so $x = y$.
    \item Symmetry is clear from the definition of
      $d_X$.
    \item For the triangle inequality, fix $x, y, z \in X$ and we have
      \[
        d_X(x, z) = \sum_{n = 1}^\infty 2^{-n} d(x_n, z_n)
        \le \sum_{n = 1}^\infty 2^{-n} (d(x_n, y_n) + d(y_n, z_n)).
      \]
      Since $d_X(x, y)$ and $d_X(y, z)$ are finite ($d(x, y) \ge 0$ is bounded and $\sum_{n = 1}^\infty 2^{-n}$ converges),
      the above series converges absolutely and thus
      we can split the above series into the sum of
      two series:
      \[
        d_X(x, z) \le \sum_{n = 1}^\infty 2^{-n} d(x_n, y_n) + \sum_{n = 1}^\infty 2^{-n} d(y_n, z_n)
        = d_X(x, y) + d_X(y, z).
      \]
      This gives the triangle inequality.
  \end{enumerate}
  The proof of the Urysohn metrization theorem
  will utilize this observation.
\end{example}
