\chapter{Oct.~3 --- Tychonoff's Theorem, Part 2}

\section{Zorn's Lemma}

\begin{definition}
  A \emph{partially ordered set} is a set $X$ with
  a relation $\le$ such that
  \begin{enumerate}[(i)]
    \item (reflexivity) $x \le x$ for all
      $x \in X$,
    \item (anti-symmetry) for all $x, y \in X$,
      if $x \le y$ and $y \le x$, then $x = y$,
    \item and (transitivity) for all $x, y, z \in X$,
      if $x \le y$ and $y \le z$, then $x \le z$.
  \end{enumerate}
\end{definition}

\begin{definition}
  A \emph{chain} in a partial ordering is a
  completely ordered subset.
\end{definition}

\begin{definition}
  An \emph{upper bound} of a chain $C \subseteq X$ is
  an element $u \in X$ such that $a \le u$
  for every $a \in C$. A \emph{maximum}
  of $X$ is an element $z \in X$ such that there
  is no $x \in X$, $x \ne z$, with $x \le z$.
\end{definition}

\begin{lemma}[Zorn's lemma]
  Let $(X, \le)$ be a partially ordered set.
  If every chain in $X$ has an upper bound, then
  $X$ attains a maximum.
\end{lemma}

\begin{remark}
  Zorn's lemma is equivalent to the axiom of choice.
  Assuming Zorn's lemma, consider the set of
  partial choice functions on the index set $I$,
  ordered by inclusion of their domains of
  definition. Show an upper bound and obtain a full
  choice function as a maximal element by Zorn's
  lemma.
\end{remark}

\begin{remark}
  When applying Zorn's lemma,
  the partial ordering $\le$ is often taken to be
  inclusion or containment of sets.\footnote{One can easily verify that this satisfies the properties of a partial order.} In this case,
  taking the union over chains typically yields
  an upper bound.\footnote{In many cases, one shows
  that this is indeed an upper bound by projecting
  to some element of the chain.}
\end{remark}

\begin{example}
  Every vector space has a basis. Here a basis is a
  maximal linearly independent set. Let $X$ be the
  set of linearly independent sets, ordered by
  inclusion. Argue that the union over a chain in
  $X$ is again a linearly independent set and thus
  an upper bound. Finish using Zorn's lemma.
\end{example}

\begin{example}
  Every ring has maximal ideals. This can also be
  shown via Zorn's lemma.
\end{example}

\section{Proof of Tychonoff's Theorem}

\begin{remark}
  Recall from Homework 4 the following problem:
  A space $X$ is compact if and only if it has the
  \emph{finite intersection property}, i.e. for every
  collection $\{C_i\}_{i \in I}$ of closed sets
  such that $\bigcap_{i \in I_\text{fin}} \ne \varnothing$
  for every finite set $I_\text{fin} \subseteq I$,
  we have $\bigcap_{i \in I} C_i \ne \varnothing$. This
  is some description of compactness via closed
  sets instead of open sets.
  We will use this characterization of compactness
  for the proof of Tychonoff's theorem.
\end{remark}

\begin{theorem}[Tychonoff]
  Let $(X_i, \T_i)$ be compact for each $i \in I$.
  Then $\prod_{i \in I} X_i$ is compact.
\end{theorem}

\begin{proof}
  Let $\{C_\alpha\}_{\alpha \in J}$ be a collection
  of closed subsets of $\prod_{i \in I} X_i$
  such that for every finite subset
  $J_\text{fin} \subseteq J$, we have
  $\bigcap_{\alpha \in J_\text{fin}} C_\alpha \ne \varnothing$.
  We then wish to show that $\bigcap_{\alpha \in J} C_\alpha \ne \varnothing$.

  We say that a collection of sets of
  $\prod_{i \in I} X_i$ has the FI property if
  for every finite subcollection, the intersection is
  nonempty. Let $S$ be the collection of all
  FI-subcollections of the product space. Define
  a partial order over $S$ by containment. We want
  maximal elements in $S$, which we will find by
  applying Zorn's lemma. Let $R$ be a chain in $S$,
  and define $T = \bigcup_{r \in R} r$. Clearly
  $r \le T$ for every $r \in R$, so it suffices
  to show that $T$ satisfies FI. To do this, pick
  $J_\text{fin}$, some finite set of indices for
  elements of $T$. For each $\alpha \in J_\text{fin}$,
  there exists $r_\alpha \in R$ such that
  $T_\alpha \in r_\alpha$. Pick the largest
  $r_\alpha$ out of the $\alpha \in J_\text{fin}$
  (we can do this since $R$ is a chain and
  $J_\text{fin}$ is finite, so e.g. run a sorting
  algorithm). Let $r_\beta$ be this largest element.
  Then $T_\alpha \in r_\beta$ for all
  $\alpha \in J_\text{fin}$, and $r_\beta$ satisfies
  FI. Thus we obtain
  \[
    \bigcap_{\alpha \in J_\text{fin}} T_\alpha
    \ne \varnothing
  \]
  by the FI property of $r_\beta$. So $T$ also
  satisfies FI, and thus $T$ is an upper bound
  for $R$. Then by Zorn's lemma, there is
  a maximum $M$ in the set $S$ of
  FI-subcollections. We may assume
  $M$ contains $\{C_\alpha\}_{\alpha \in J}$.

  At this point, we make a few observations:
  \begin{enumerate}[(i)]
    \item If $\{S_j\}_{j = 1}^m \in M$, then
      $\bigcap_{j = 1}^m S_j \in M$.

      To see this, assume not. Then consider
      $M \cup \{\bigcap_{j = 1}^m S_j\}$.
      Any finite subset looks like
      \[
        m_1 \cap \cdots \cap m_k \cap S_{j_1} \cap \cdots \cap S_{j_\ell},
      \]
      which are all in $M$, so this intersection
      is nonempty. Thus $M \cup \{\bigcap_{j = 1}^m S_j\}$
      is a larger FI collection containing $M$,
      which contradicts the maximality of $M$.
    \item If $S \subseteq X$ such that
      $S \cap m \ne \varnothing$ for every $m \in M$,
      then $S \in M$.

      Similarly assume not. Then consider
      $M \cup \{S\}$. Any finite subset looks like
      \[
        m_1 \cap \cdots \cap m_k \cap S = S \cap \bigcap_{j = 1}^k m_j.
      \]
      By the previous observation,
      $\bigcap_{j = 1}^k m_j \in M$, so
      $S \cap \bigcap_{j = 1}^k m_j \ne \varnothing$.
      Thus $M \cup \{S\}$ is a larger FI collection
      containing $M$, which again contradicts the
      maximality of $M$.
  \end{enumerate}
  Now consider the $M$ that contained
  $\{C_\alpha\}_{\alpha \in J}$. For each $i \in I$,
  consider $\{\pi_i(C_\alpha)\}_{\alpha \in J}$. Note
  that $\pi_i(C_\alpha)$ need not be closed.
  So look at the closure $\overline{\pi_i(C_\alpha)} \subseteq X_i$
  instead. Then for any $J_\text{fin}$ finite,
  \[
    \bigcap_{\alpha \in J_\text{fin}} \overline{\pi_i(C_\alpha)}
    \ne \varnothing
  \]
  because we have that
  \[
    \varnothing \ne \pi_i\left(\bigcap_{\alpha \in J_\text{fin}} C_\alpha\right)
    \subseteq \bigcap_{\alpha \in J_\text{fin}} \overline{\pi_i(C_\alpha)}.
  \]
  Here we know $\bigcap_{\alpha \in J_\text{fin}} C_\alpha \ne \varnothing$
  by the FI property of $M$. Then by the compactness
  of the $X_i$, we have
  \[
    \bigcap_{\alpha \in J} \overline{\pi_i(C_\alpha)}
    \ne \varnothing.
  \]
  So choose $y_i \in \bigcap_{\alpha \in J} \overline{\pi_i(C_\alpha)}$, and
  set $y = (y_i)_{i \in I}$. Now we claim that
  \[
    y \in \bigcap_{\alpha \in J} C_\alpha
    \subseteq \prod_{i \in I} X_i,
  \]
  which immediately implies Tychonoff's theorem.
  To show this, we need the following facts:
  \begin{enumerate}
    \item For every $U \subseteq \prod_{i \in I} X_i$ open and
      $T \in M$, we have $U \cap T \ne \varnothing$.
    \item For every $i \in I$ and
      $U_i \subseteq X_i$ open, we have
      $\pi_i^{-1}(U_i) \in M$.
  \end{enumerate}
  We will finish the proof next class. Also, in
  the above argument, a slight issue is that
  we should work with all of $M$ instead of
  just with $\{C_\alpha\}_{\alpha \in J}$.
\end{proof}
