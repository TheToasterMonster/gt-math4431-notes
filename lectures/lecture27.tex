\chapter{Dec.~3 --- Final Exam Review}

\section{Review Problems}

\begin{exercise}
  Let $h : S^1 \to S^1$ be continuous and homotopic
  to a constant map. Prove that there exists
  a point $x \in S^1$ such that
  $\measuredangle(x, h(x)) = 90^\circ$.\footnote{Here $\measuredangle(x, y)$ is the \emph{signed angle}, measured with respect to the center of the circle.}
\end{exercise}

\begin{proof}
  If $h(x) = c$ is constant, simply take $x$ to be
  $c$ rotated by $90^\circ$ in either direction.

  So assume $h$ is non-constant. Define a
  map $f : S^1 \to S^1$ by
  $f(x) = \measuredangle(x, h(x))$, which is
  continuous. We claim that $f$ must change
  signs. To see this, suppose otherwise that
  $f$ is always positive or always negative. In
  this case, $h(x)$ is always on the same side
  of $x$ and thus completes a full loop around
  $S^1$ as $x$ travels around $S^1$. This
  implies that $h$ is not nullhomotopic, a
  contradiction. Thus $f$ must change signs.

  So there exists a point $x$ such that
  $\measuredangle(x, h(x)) = 0$. Argue now via dot
  products.
\end{proof}

\begin{exercise}
  Let $f : \R^2 \to \R^2$ be continuous and
  $\|f(p) - p\| < 1 / 2$. Show that $f$ is surjective.
\end{exercise}

\begin{proof}
  Assume not, so there exists $p_0$ such that
  $f(x) \ne p_0$ for any $x \in \R^2$. Then define
  $r : \R^2 \to S^1$ by
  \[
    r(x) = \frac{f(x) - p_0}{\|f(x) - p_0\|},
  \]
  which is continuous. In particular,
  consider $r|_{S^1 + p_0} : S^1 \to S^1$ and notice
  that the image of the points
  \[p_0 + (\pm 1 / \sqrt{2}, \pm 1 / \sqrt{2})\]
  under $f$ lie within a ball of radius
  $1 / 2$ centered at $p_0$. In particular, the
  image of these points under $r$ cannot leave
  their respective quadrants, so $r|_{S^1 + p_0}$
  is not nullhomotopic on $S^1$. Viewing this
  $S^1$ as lying in $\R^2$ and looking at
  the disk it encloses, we get a contradiction
  to the no-retract theorem.
\end{proof}

\begin{exercise}
  Consider $S^1 = \{z \in \C : |z| = 1\}$, and
  define $z_1 \sim z_2$ if
  $z_1 / z_2 = e^{2\pi i x}$ for some
  $x \in \Q$.
  \begin{enumerate*}[(1)]
    \item Is $S^1 / {\sim}$ Hausdorff?
    \item Is $S^1 / {\sim}$ compact?
  \end{enumerate*}
\end{exercise}

\begin{proof}
  (1) No. Take $[x_0] \ne [y_0] \in S^1 / {\sim}$,
  and consider open sets $U, V$ containing
  $[x_0], [y_0]$ in the quotient.
  Let $q$ be the
  quotient map, and fix $x \in q^{-1}(U)$.
  Note that this implies $x + \Q \in q^{-1}(U)$. Now
  $q^{-1}(V)$ is open, so $(x + \Q) \cap q^{-1}(V) \ne \varnothing$.
  This implies that $[x] \in q(q^{-1}(V)) = V$.
  So $[x] \in U \cap V$.

  (2) Yes, we see that $S^1 / {\sim}$ is a quotient of
  a compact space, hence it must be compact.
\end{proof}

\begin{remark}
  This equivalence relation is called the
  \emph{Vitali equivalence relation}, and is
  used in measure theory to construct a
  non-Lebesgue measurable set.
\end{remark}

\begin{exercise}
  Consider $\Q \subseteq \R$, and its
  one-point compactification $\Q^+$. Is $\Q^+$ a
  subspace of $\R$?
\end{exercise}

\begin{proof}
  One way to proceed is to show that $\Q^+$ is not
  Hausdorff and thus cannot be a subspace of $\R$.
\end{proof}

\begin{exercise}
  Consider $\R^2$ and define
  $(x_1, y_1) \sim (x_2, y_2)$ if
  $y_1 - x_1 = y_2 - x_2$. What
  is $\R^2 / {\sim}$?
\end{exercise}

\begin{proof}
  The quotient is homeomorphic $\R$ with the metric
  topology. Take the
  map $f : \R^2 / {\sim} \to \R$ given by
  \[
    [(x, y)] \mapsto y - x.
  \]
  Note that $(x, y) \mapsto y - x$ is continuous
  on $\R^2$ and respects $\sim$, so it descends
  to the map $f$ on the quotient. For surjectivity,
  simply not that for any $y \in \R$, we
  have $f([(0, y)]) = y$. For injectivity, suppose
  \[
    f([(x, y)]) = f([(z, w)]).
  \]
  Then $y - x = w - z$, so $[(x, y)] = [(z, w)]$
  as required. To see that the inverse is
  continuous, let $U$ be open in $\R^2 / {\sim}$.
  If $[(x_1, y_1)] \in U$, then the whole equivalence
  class of $(x_1, y_1)$ in $\R^2$ is contained
  in the open set $q^{-1}(U)$, where $q$ is
  the quotient map. So $q^{-1}(U)$ contains
  the entire line $y = x + (y_1 - x_1)$, and
  in particular a strip around it. Finish and show
  that $f^{-1}$ is continuous as an exercise.

  Then $f$ is a homeomorphism, so
  $\R^2 / {\sim}$ and $\R$ are homeomorphic.
\end{proof}
