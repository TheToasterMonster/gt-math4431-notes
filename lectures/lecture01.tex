\chapter{Aug.~20 --- Review of Metric Spaces}

\section{Metric Spaces}
Recall the definition of a \emph{metric space}:

\begin{definition}
  Given a set $X$, a function $d : X \times X \to \R$
  is called a \emph{metric} if
  \begin{enumerate}[(i)]
    \item (strong positivity)
      $d(x, y) \ge 0$ for all
      $x, y \in X$, and $d(x, y) = 0$ if and only if
      $x = y$,
    \item (symmetry) $d(x, y) = d(y, x)$,
    \item and (triangle inequality)
      $d(x, z) \le d(x, y) + d(y, z)$
      for all $x, y, z \in X$.
  \end{enumerate}
\end{definition}

\begin{example}
  For any set $X$, we can define the
  \emph{discrete metric}
  by
  \[
    d(x, y) =
    \begin{cases}
      1 & \text{if $x \ne y$}, \\
      0 & \text{otherwise}.
    \end{cases}
  \]
  Verify as an exercise that this satisfies
  the triangle inequality.
\end{example}

\begin{example}
  The Euclidean metric in $\R^n$ is
  \[
    d(\overline{x}, \overline{y})
    = \sqrt{\sum_{i = 1}^n (x_i - y_i)^2}
  \]
  where $\overline{x} = (x_1, \ldots, x_n)$
  and $\overline{y} = (y_1, \ldots, y_n)$.
\end{example}

\section{Open Sets}
\begin{definition}
  The \emph{open ball} of radius $R > 0$ around
  $x_0 \in X$ is
  \[
    B_R(x_0) = \{y \in X \mid d(x_0, y) < R\}
  .\]
  Given a set $S \subseteq X$, a point $x_0$
  is called an interior point of $S$ if there exists
  $r > 0$ such that $B_r(x_0) \subseteq S$.
  The set $S$ is called \emph{open} if all of its
  points are interior points.
\end{definition}

\begin{prop}
  The open ball $B_R(x)$ is open.
\end{prop}

\begin{proof}
  Fix an arbitrary $y \in B_R(x)$, and observe that
  it suffices to show that $y$ is an interior point.
  Take $r = R - d(x, y)$, and
  first note that $r > 0$ since
  $d(x, y) < R$. Now note that for all $z \in B_r(y)$,
  we have
  \[
    d(x, z) \le d(x, y) + d(y, z)
    < d(x, y) + (R - d(x, y))
    = R,
  \]
  so that $z \in B_R(x)$.
  Thus $B_r(y) \subseteq B_R(x)$, and so $y$ is an interior point.
\end{proof}

\begin{corollary}
  $B_R(x) = \bigcup_{y \in B_R(x)} B_{r_y}(y)$,
  where $r_y = R - d(x, y)$.
\end{corollary}

\begin{proof}
  We have $B_{r_y}(y) \subseteq B_R(x)$
  for each $y \in B_R(x)$,\footnote{Using the argument from the previous proposition.}
  and so
  $\bigcup_{y \in B_R(x)} B_{r_y}(y) \subseteq B_R(x)$.
  For the reverse inclusion simply observe that
  $y \in B_{r_y}(y) \subseteq \bigcup_{y \in B_R(x)} B_{r_y}(y)$ for
  each $y \in B_R(x)$.
\end{proof}

\begin{prop}
  In a metric space $(X, d)$, the following
  are true:
  \begin{enumerate}[(i)]
    \item $\varnothing, X$ are open,
    \item if $\{S_i\}_{i \in I}$ are open, then
      $\bigcup_{i \in I} S_i$ is open,
    \item and if $\{S_i\}_{i = 1}^n$ are open, then
      $\bigcap_{i = 1}^n S_i$ is open.
  \end{enumerate}
\end{prop}

\begin{proof}
  $(i)$ The empty set is open vacuously.
  To see that $X$ is open, simply take $R = 1$
  for any $x \in X$.

  $(ii)$ Fix $x \in \bigcup_{i \in I} S_i$
  arbitrary, so there exists $i_0 \in I$ with
  $x \in S_{i_0}$. Since $S_{i_0}$ is open,
  $x$ is an interior point and thus there exists
  $r > 0$ such that $B_r(x) \subseteq S_{i_0}$.
  But then $B_r(x) \subseteq S_{i_0} \subseteq \bigcup_{i \in I} S_i$,
  so $x$ is an interior point of
  $\bigcup_{i \in I} S_i$ also and thus
  $\bigcup_{i \in I} S_i$ is open.

  $(iii)$ Now assume $x \in \bigcap_{i = 1}^n S_i$.
  Then for each $1 \le i \le n$, there exists
  $r_i > 0$ such that $B_{r_i}(x) \subseteq S_i$.
  Then we can choose
  \[
    r = \min\{r_1, \ldots, r_n\} > 0,
  \]
  so that $B_r(x) \subseteq B_{r_i}(x) \subseteq S_i$
  for each $1 \le i \le n$. Thus
  $B_r(x) \subseteq \bigcap_{i = 1}^n S_i$
  and $\bigcap_{i = 1}^n S_i$ is open.
\end{proof}

\begin{remark}
  The above argument for the finite intersection
  property requires that there are only finitely
  many $r_i$. Otherwise it may very well be that
  $r = \inf\{r_i\} = 0$ and the argument fails.
\end{remark}
