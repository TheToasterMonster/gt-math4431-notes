\chapter{Sept.~24 --- Product Spaces}

\section{Product Spaces}
\begin{definition}
  For two sets $X, Y$, define their \emph{Cartesian product} as
  \[
    X \times Y = \{(x, y) \mid x \in X, y \in Y\}.
  \]
  Similarly we can define\footnote{One can also think of an element of the product as a function $I \to \bigcup_{i \in I} X_i$, where $f(i) \in X_i$.}
  \[
    \prod_{i \in I} X_i = \{(x) \mid (x)_i \in X_i\}.
  \]
  If each $X_i$ is a topological space
  with topology $\mathcal{T}_i$, then we define
  the following topologies on $\prod_{i \in I} X_i$:
  \begin{itemize}
    \item The \emph{box topology}: Take as a basis
      sets of the form $\prod_{i \in I} U_i$
      where $U_i \subseteq X_i$ are open sets.

      Suppose $B_1, B_2$ are basis sets and $x \in B_1 \cap B_2$.
      Then by definition $B_1 = \prod_{i \in I} U_i$ and
      $B_2 \in \prod_{i \in I} V_i$, where
      $U_i, V_i \subseteq X_i$ are open. Then
      set $B_3 = \prod_{i \in I} (U_i \cap V_i)$.
      Clearly $x \in B_3$ and $B_3 \subseteq B_1 \cap B_2$
      is a basis element (each $U_i \cap V_i$ is
      a finite intersection of open sets and thus
      open), so this is a basis.
    \item The \emph{product topology}:
      Take as a subbasis the sets
      $\pi_{i}^{-1}(U_i) \subseteq \prod_{i \in I} X_i$
      for each $i \in I$
      and $U_i \subseteq X_i$ open. Here
      $\pi_{i} : \prod_{j \in I} X_j \to X_i$
      is the projection onto the $i$th factor.

      The subbasis sets here are of the form
      $U_i \times \prod_{j \ne i} X_j$. The general
      basis sets will be finite intersections of
      these sets, i.e.
      \[
        U_{i_1} \times \dots \times U_{i_n}
        \times \prod_{j \ne i_1, \dots, i_n} X_j.
      \]
      These are called the \emph{cylindrical sets}.
      Think of this as having only finitely many
      restrictions on the factors, whereas we get
      to choose arbitrarily many restrictions with the
      box topology.
  \end{itemize}
\end{definition}

\begin{remark}
  For finite products, the box and product
  topologies coincide. They differ for infinite
  products: The box topology is finer than the
  product topology (so the box topology has
  more open sets).
\end{remark}

\begin{example}
  Consider the following product spaces:
  \begin{itemize}
    \item The power set $\mathcal{P}(X) = 2^X = \{0, 1\}^X$.
      Think of the elements as functions
      $X \to \{0, 1\}$, which pick whether or not
      to include each element of $X$ in the
      corresponding subset. Thus the power set
      comes with a natural topology (box or
      product) if $\{0, 1\}$
      is given the discrete topology.
    \item The space $\{0, 1\}^\N$ is the
      Cantor set, if $\{0, 1\}$ is given the
      discrete topology. The Cantor set with
      the metric topology inherited from $\R$ is
      homeomorphic to $\{0, 1\}^\N$ with the product
      topology. Think of the sequence $\{x_n\} \subseteq \{0, 1\}^\N$
      as choosing whether to pick the left or right
      third at each step.
    \item The space $[0, 1]^\N$ is called
      \emph{Hilbert's cube}. Note that
      $[0, 1] \times [0, 1]^\N \cong [0, 1]^\N$.
      For a homeomorphism, simply shift the sequence
      one to
      the right, putting a $0$ in the first slot.
      Then forget about the $0$.
  \end{itemize}
\end{example}

\begin{remark}
  Always assume $\prod_{i \in I} X_i$ is given
  the product topology, unless otherwise specified.
\end{remark}

\section{Properties of Product Spaces}
\begin{theorem}
  Assume $(X_i, \mathcal{T}_i)_{i \in I}$ are each
  $T_0$, each $T_1$, or each Hausdorff. Then the
  product
  $\prod_{i \in I} X_i$ is also $T_0$, $T_1$,
  or Hausdorff, respectively.
\end{theorem}

\begin{proof}
  We prove only the Hausdorff case. Let
  $(x), (y) \in \prod_{i \in I} X_i$ be distinct.
  As $(x) \ne (y)$, there is $i \in I$
  with $x_i \ne y_i$, where $x_i, y_i \in X_i$.
  Since $X_i$ is Hausdorff, there exist
  $A, B \subseteq X_i$ open, disjoint
  with $x_i \in A$ and $y_i \in B$.
  Then set
  \[
    U = A \times \prod_{j \ne i} X_j \quad \text{and} \quad
    V = B \times \prod_{j \ne i} X_j.
  \]
  These sets are open since they are cylindrical,
  and clearly $(x) \in U$, $(y) \in V$ since
  $x_i \in A$, $y_i \in B$. Also $U, V$ are
  disjoint since their $i$th components $A, B$
  are disjoint. So this is a separation of
  $(x)$ and $(y)$ by disjoint, open sets, and
  we conclude that $\prod_{i \in I} X_i$ is Hausdorff.
\end{proof}

\begin{corollary}
  Assume $(X_i, \mathcal{T}_i)_{i \in I}$ are each
  $T_0$, each $T_1$, or each Hausdorff. Then
  $\prod_{i \in I} X_i$ with the
  box topology is also $T_0$, $T_1$,
  or Hausdorff, respectively.
\end{corollary}

\begin{proof}
  The box topology is finer than the product topology.
\end{proof}

\begin{remark}
  In the product topology, the
  projections $\pi_i : \prod_{j \in I} X_j \to X_i$
  are continuous, onto, and open.
  The continuity and surjectivity of $\pi_i$ is
  essentially by construction of the product topology.
  To see that $\pi_i$ is open, consider a basis
  element of $\prod_{j \in I} X_j$, which is of the form
  \[
    U = U_{i_1} \times \dots \times U_{i_n}
    \times \prod_{j \ne i_1, \dots, i_n} X_j.
  \]
  Then $\pi_i(U)$ is either one of the
  $U_{i}$ or one of the $X_i$, which are both open.
\end{remark}

\begin{theorem}[Universal property of the product topology]
  The following diagram commutes:
  \[
    \begin{tikzcd}
      Z \ar[r, "f_i"] \ar[dr, "f", swap, dashed] & X_i \\
       & \prod_{i \in I} X_i \ar[u, "\pi_i"']
    \end{tikzcd}
  \]
  In particular, there exists a unique continuous
  map $f : Z \to \prod_{i \in I} X_i$ such that
  $f_i = \pi_i \circ f$ for each $i \in I$.
\end{theorem}

\begin{proof}
  Define $f : Z \to \prod_{i \in I} X_i$ on the
  set theory level by $(f(z))_i = f_i(z)$. Now we
  show the continuity of $f$. Fix a basis
  element
  \[
    B = U_{i_1} \times \dots \times U_{i_n} \times
    \prod_{j \ne i_1, \dots, i_n} X_j.
  \]
  Then we can write
  \[
    f^{-1}(B) = \{z \mid f(z) \in B\}.
  \]
  Now observe that $f(z) \in B = U_{i_1} \times \dots \times U_{i_n} \times \prod_{j \ne i_1, \dots, i_n} X_j$
  is equivalent to
  \[
    z \in \bigcap_{k = 1}^n f_{i_k}^{-1}(U_{i_k}).
  \]
  Since $U_{i_k}$ is open in $X_{i_k}$ and
  each $f_i$ is continuous, each
  $f_{i_k}^{-1}(U_{i_k})$ is open in $Z$.
  Then this is a finite intersection of open
  sets in $Z$, hence open. Thus $f$ is continuous.
\end{proof}

\begin{remark}
  If we had the box topology, this argument
  would not work. In particular, the intersection
  that we get could be infinite, which would
  not necessarily be open in $Z$.
\end{remark}

\begin{remark}
  This universal property formalizes the notion that
  we define functions into products by defining
  a function to each factor. Additionally, this
  generalizes the result from multivariable calculus
  that a vector-valued function is continuous
  precisely when each component function is continuous.
\end{remark}
