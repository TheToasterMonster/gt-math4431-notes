\chapter{Nov.~7 --- The Fundamental Group, Part 2}

\section{The Fundamental Group is a Group}
\begin{theorem}
  The fundamental group $\pi_1(X, x_0)$ with the
  operation $*$ is a group.
\end{theorem}

\begin{proof}
  We have already shown that $*$ is well-defined,
  so it suffices to check the group axioms. First
  clearly the concatenation of loops is still a
  loop, so we have closure under the group operation.

  The identity element is the constant
  curve $e(t) = x_0$ for all $t \in [0, 1]$. We
  must check that
  \[
    [\gamma] * [e] = [\gamma] = [e] * [\gamma].
  \]
  for any $\gamma \in \pi_1(X, x_0)$. It suffices
  to show $\gamma * e \sim \gamma$ and $e * \gamma \sim \gamma$.
  For the former, take the homotopy
  \[
    H(t, s) = \gamma(\min\{t(1 + s), 1\}).
  \]
  Observe that we have
  \begin{align*}
    H(t, 0) = \gamma(\min\{t, 1\}) = \gamma(t), \quad
    H(t, 1) = \gamma(\min\{2t, 1\}) = (\gamma * e)(t),
  \end{align*}
  and
  \[
    H(0, s) = \gamma(\min\{0, 1\}) = \gamma(0) = x_0, \quad
    H(1, s) = \gamma(\min\{1(1 + s), 1\}) = \gamma(1) = x_0
  .\]
  Thus $H$ is a homotopy from $\gamma$ to $\gamma * e$,
  so $\gamma \sim \gamma * e$. The proof that $e * \gamma \sim \gamma$
  is similar.

  Now we check associativity. Let
  $\gamma_1, \gamma_2, \gamma_3 \in \pi_1(X, x_0)$, so
  \[
    ((\gamma_1 * \gamma_2) * \gamma_3)(t) =
    \begin{cases}
      (\gamma_1 * \gamma_2)(2t) & t \le 1/2 \\
      \gamma_3(2t - 1) & t \ge 1 / 2
    \end{cases} =
    \begin{cases}
      \gamma_1(4t) & t \le 1/4, \\
      \gamma_2(4t - 1) & 1/4 \le t \le 1/2, \\
      \gamma_3(2t - 1) & t \ge 1/2.
    \end{cases}
  \]
  One can find a reparametrization
  into $\gamma_1 * (\gamma_2 * \gamma_3)$,
  which gives $(\gamma_1 * \gamma_2) * \gamma_3 \sim \gamma_1 * (\gamma_2 * \gamma_3)$.

  Now we check for inverses. Given
  $\gamma \in \pi_1(X, x_0)$, define
  $\delta : I \to X$ such that
  \[
    \delta(t) = \gamma(1 - t).
  \]
  We will show that $\gamma * \delta \sim e$ and
  $\delta * \gamma \sim e$. For the former, consider
  the homotopy
  \[
    H(t, s) =
    \begin{cases}
      \gamma(t) & t \le s / 2, \\
      \delta(1 - t) & s / 2 \le t \le s, \\
      x_0 & t \ge s
    \end{cases}
  \]
  from $e$ to $\gamma * \delta$, which is
  continuous since $\gamma(t) = \delta(1 - t)$.
  Thus $e \sim \gamma * \delta$. Note that we also
  get $\delta * \gamma \sim e$ for free since
  $\gamma(t) = \delta(1 - t)$, which lets us
  repeat the same argment.
  Thus $\pi_1(X, x_0)$ is a group.
\end{proof}

\section{Contractible Spaces and the Fundamental Group}

\begin{lemma}
  If $X$ is contractible, i.e. $\id_X$ is homotopic
  to constant map, then $\pi_1(X, x_0) = \{e\}$.
\end{lemma}

\begin{proof}
  Let $\gamma$ be a closed curve in $X$. Then
  $\id_X \circ \gamma$ is a curve in $X$. As
  $\id_X$ is homotopic to a constant map, say by
  $H(x, s)$,
  considering $H(\gamma(t), s)$ gives a homotopy
  of curves from $\gamma$ to a constant map. So
  $\gamma \sim e$, which means that
  $\pi_1(X, x_0) = \{e\}$ since this holds for
  every $\gamma \in \pi_1(X, x_0)$.
\end{proof}

\begin{remark}
  If $X$ is contractible, then $X$ is path-connected.
  For $x, y \in X$, contract both to a single common
  point, then join those two paths to get a path from
  $x$ to $y$.
\end{remark}

\begin{remark}
  Since $\R$, $\R^n$, and even $[0, 1]^\N$ are
  contractible, they have trivial fundamental groups.
\end{remark}

\begin{remark}
  The $2$-sphere is not contractible but its
  fundamental group is trivial.
\end{remark}
