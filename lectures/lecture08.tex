\chapter{Sept.~12 --- Separation Axioms}

\section{Separation Axioms}
\begin{definition}
  A topological space is said to satisfy the
  \emph{$T_0$ axiom} if the following holds:
  For every $a, b \in X$ with $a \ne b$, there exists
  $U$ open such that either $a \in U$, $b \notin U$
  or $b \in U$, $a \notin U$.
\end{definition}

\begin{remark}
  With the $T_0$ axiom, we cannot choose which
  point is in $U$ and which is not.
  For instance take $X = \{a, b\}$ with topology
  $\mathcal{T} = \{\varnothing, \{a\}, X\}$.
  This space is $T_1$, but we can only choose
  $U$ to contain $a$.
\end{remark}

\begin{definition}
  A space is said to satisfy the \emph{$T_1$ axiom}
  if for every $a, b \in X$ with $a \ne b$,
  there exist $U_a, U_b$ open such that
  $a \in U_a$, $b \notin U_a$ and
  $b \in U_b$, $a \notin U_b$.
\end{definition}

\begin{remark}
  With the $T_1$ axiom, $U_a$ and $U_b$
  need not be disjoint.
\end{remark}

\begin{definition}
  A space is said to be $T_2$ or \emph{Hausdorff}
  if the following holds: For every $a, b \in X$
  with $a \ne b$, there exist $U_a, U_b$ open
  such that $a \in U_a$, $b \in U_b$ and
  $U_a \cap U_b = \varnothing$.
\end{definition}

\begin{example}
  Metric spaces are Hausdorff. For any $a \ne b$,
  pick balls with radius $d(a, b) / 2$ around $a, b$.
\end{example}

\begin{theorem}
  We have the proper inclusion
  $T_2 \subsetneq T_1 \subsetneq T_0$.
\end{theorem}

\begin{proof}
  Come up with the examples as an exercise.
  Some of them were already discussed above.
\end{proof}

\begin{theorem}
  In a $T_1$ space, every singleton $\{x\}$ is
  closed.
\end{theorem}

\begin{proof}
  Fix $x \in X$. For every $y \ne x$, by the
  $T_1$ axiom we can find $U_y$ open such that
  $y \in U_y$ and $x \notin U_y$. In particular,
  this means that $U_y \subseteq \{x\}^c$. Then
  we can write
  \[
    \{x\}^c \subseteq \bigcup_{y \in \{x\}^c} U_y
    \subseteq \{x\}^c.
  \]
  So $\{x\}^c = \bigcup_{y \in \{x\}^c} U_y$,
  which is open as a union of open sets.
  Thus $\{x\}$ is closed.
\end{proof}

\section{Properties of Hausdorff Spaces}
\begin{theorem}
  In a Hausdorff space, a point
  $x$ is an accumulation point of
  a set $A$ if and only if every neighborhood
  of $x$ contains infinitely many elements of $A$.
\end{theorem}

\begin{proof}
  $(\Leftarrow)$ This is clear.

  $(\Rightarrow)$ Pick $x$ an accumulation point
  of $A$, and assume otherwise that there exists a
  neighborhood $U$ of $x$ with only finitely
  many elements of $A$, i.e.
  $|(U \setminus \{x\}) \cap A| < \infty$. Then
  we can write
  \[
    (U \setminus \{x\}) \cap A = \{a_1, \ldots, a_n\}
    = \bigcup_{i = 1}^n \{a_i\}.
  \]
  Since our space is Hausdorff and thus also $T_1$,
  these singletons $\{a_i\}$ are closed. Then
  $(U \setminus \{x\}) \cap A$ is closed as a
  finite union of closed sets. Now since
  our space is Hausdorff, for every
  $1 \le i \le n$ we can separate $a_i$ from $x$,
  i.e. there exists $U_{x_i}, U_{a_i}$ open
  such that $x \in U_{x_i}$, $a_i \in U_{a_i}$
  and $U_{x_i} \cap U_{a_i} = \varnothing$. Then
  \[
    U' = U \cap \bigcap_{i = 1}^n U_{x_i}
  \]
  is open as a finite intersection of open sets.
  Also $x \in U'$ since $x \in U$ and
  $x \in U_{x_i}$ for each $i$. But
  \[
    (U \setminus\{x\}) \cap A
    = \{a_1, \dots, a_n\}
    \subseteq \bigcup_{i = 1}^n U_{a_i}
  \]
  and $U_{a_j} \cap \bigcap_{i = 1}^n U_{x_i} = \varnothing$
  for all $j$, so
  $(U' \setminus \{x\}) \cap A = \varnothing$.
  Contradiction.
\end{proof}

\begin{remark}
  Maybe just the $T_1$ axiom is enough for this
  theorem. Think more about this.
\end{remark}

\begin{definition}
  A sequence $\{x_n\}_{n = 1}^\infty \subseteq (X, \mathcal{T})$
  \emph{converges} to a point $x \in X$, written
  $x_n \to x$, if for any open set $U$ containing $x$,
  there exists $N_0 \in \N$ such that
  $x_n \in U$ for every $n \ge N_0$.
\end{definition}

\begin{theorem}
  In a Hausdorff space, a convergent sequence has a
  unique limit.
\end{theorem}

\begin{proof}
  Assume otherwise that $x_n \to L_1$ and
  $x_n \to L_2$ with $L_1 \ne L_2$. Then since our
  space is Hausdorff, we can find
  $U_{L_1}, U_{L_2}$ open such that
  $L_1 \in U_{L_1}$, $L_2 \in U_{L_2}$ and
  $U_{L_1} \cap U_{L_2} = \varnothing$. Since
  $x_n \to L_1$, there exists $N_0 \in \N$ such
  that $x_n \in U_{L_1}$ for all $n \ge N_0$.
  Similarly we can find $N_0' \in \N$ with
  $x_n \in U_{L_2}$ for all $n \ge N_0'$ since
  $x_n \to L_2$. But then for $N = \max\{N_0, N_0'\}$,
  we have $x_N \in U_{L_1} \cap U_{L_2}$,
  a contradiction.
\end{proof}

\begin{theorem}
  In a Hausdorff space, every compact set is closed.
\end{theorem}

\begin{proof}
  Let $C \subseteq X$ be compact, and we show
  that $C^c$ is open. So fix $y \in C^c$.
  For any $x \in C$, since our space is Hausdorff,
  we can find $U_x, U_y$ open such that
  $x \in U_x$, $y \in U_y$ and $U_x \cap U_y = \varnothing$.
  Now consider $\bigcup_{x \in C} U_x$. This is
  an open cover of $C$, so we can find a finite
  subcover $C \subseteq \bigcup_{i = 1}^n U_{x_i}$
  since $C$ is compact. Then the finite intersection
  $\bigcap_{i = 1}^n U_{y_i}$ is an open set contain
  $y$, and it is disjoint from $C$ by construction
  since $U_{x_i} \cap U_{y_i} = \varnothing$ for each $i$.
  Now set $\widetilde{U}_y = \bigcap_{i = 1}^n U_{y_i}$, so that
  \[
    C^c \subseteq \bigcup_{y \in C^c} \widetilde{U}_y
    \subseteq C^c.
  \]
  Thus $C^c = \bigcup_{y \in C^c} \widetilde{U}_y$,
  which is open as the union of open sets, so
  we conclude that $C$ is closed.
\end{proof}
