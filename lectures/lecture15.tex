\chapter{Oct.~8 --- Tychonoff's Theorem, Part 3}

\section{Proof of Tychonoff's Theorem, Continued}
\begin{theorem}[Tychonoff]
  Let $(X_i, \T_i)$ be compact for each $i \in I$.
  Then $\prod_{i \in I} X_i$ is compact.
\end{theorem}

\begin{proof}
  We left off with the claim that
  \[
    x \in \bigcap_{m \in M} \overline{m}
    \subseteq \bigcap_{\alpha \in J} C_\alpha
    \subseteq \prod_{i \in I} X_i.
  \]
  To show this,
  we will show that $x \in \overline{m}$ for every
  $m \in M$. We need to show:
  \begin{itemize}
    \item For every open $U \subseteq \prod_{i \in I} X_i$ containing $x$ and
      $m \in M$, we have $U \cap m \ne \varnothing$.

      Pick $U$ a basis element (containing $x$), i.e.
      \begin{align*}
        U = U_{j_1} \times \cdots \times U_{j_n} \times
        \prod_{k \ne j_1, \ldots, j_n} X_k
        &= \left(U_{j_1} \times \prod_{r \ne j_1} X_r\right)
        \cap \cdots \cap \left(U_{j_n} \times \prod_{r' \ne j_n} X_{r'}\right) \\
        &= \pi_{j_1}^{-1}(U_{j_1}) \cap \cdots \cap \pi_{j_n}^{-1}(U_{j_n}).
      \end{align*}
      By property $(i)$ of $M$, it is enough to show
      that $\prod_{j_\ell}^{-1} U_{j_\ell} \in M$ for
      each $\ell$,
      since this implies that $U \in M$, so that
      $U \cap m \ne \varnothing$ for every $m \in M$
      by the FI property of
      $M$. Now consider $\pi_{j_\ell}^{-1}(U_{j_\ell})$.
      By property $(ii)$ of $M$, it suffices to show
      that
      $\pi_{j_\ell}^{-1}(U_{j_\ell}) \cap m \ne \varnothing$
      for every $m \in M$. This happens if
      and only if $U_{j_\ell} \cap \pi_{j_\ell}(m) \ne \varnothing$, so it is enough to show this.
      As $x \in U$, we have $x_{j_\ell} \in U_{j_\ell}$.
      But also
      \[
        x_{j_\ell} \in \bigcap_{m \in M} \overline{\pi_{j_\ell}(m)},
      \]
      so $x_{j_\ell} \in \overline{\pi_{j_\ell}(m)}$.
      Since $U_{j_\ell}$ is open and contains $x_{j_\ell}$,
      this means that $U_{j_\ell} \cap \pi_{j_\ell}(m) \ne \varnothing$.
  \end{itemize}
  Thus $\bigcap_{\alpha \in J} \ne \varnothing$,
  so $\prod_{i \in I} X_i$ has the finite intersection
  property, i.e. $\prod_{i \in I} X_i$ is compact.
\end{proof}

\begin{remark}
  If we have Tychonoff's theorem, why does Heine-Borel
  not work for a product of infinitely many factors
  $\R$? We run into issues with boundedness: It is
  nontrivial to even define a metric on an infinite
  product of metric spaces. An analogue of
  Heine-Borel for infinite dimensions is the
  Arzela-Ascoli theorem, which requires an additional
  property on top of closedness and boundedness
  (equicontinuity).
\end{remark}

\section{Quotient Spaces}

\begin{definition}
  A continuous and surjective function
  $p : X \to Y$ is called
  a \emph{quotient map} if $U \subseteq Y$ is
  open if and only if $p^{-1}(U) \subseteq X$ is open.
\end{definition}

\begin{example}
  If a continuous surjection $p$ is also an open
  map, then $p$ is a quotient map.
\end{example}

\begin{definition}
  An \emph{equivalence relation} on a set $X$ is a
  relation $\sim$ which is
  \begin{enumerate}[(i)]
    \item (reflexivity) $x \sim x$ for all $x \in X$,
    \item (symmetry) $x \sim y$ if and only if
      $y \sim x$,
    \item and (transitivity) if $x \sim y$ and
      $y \sim z$, then $x \sim z$.
  \end{enumerate}
  Define the \emph{equivalence class} of $x \in X$
  as $[x] = \{y \in X \mid y \sim x\}$, and
  $X / {\sim}$ to be the set of equivalence classes.
  There is a standard map $p : X \to X / {\sim}$
  given by $x \mapsto [x]$.\footnote{Note that $p$ is surjective by construction. For any equivalence class, pick any member of it as a preimage.}
\end{definition}

\begin{definition}
  We define the \emph{quotient topology} on $X / {\sim}$ by giving it the smallest topology such that
  $p : X \to X / {\sim}$ is a quotient map, i.e.
  we define $U \subseteq X / {\sim}$ to be open if
  $p^{-1}(U) \subseteq X$ is open.
\end{definition}

\begin{example}
  Consider the interval $X = [0, 1]$. Define $\sim$
  via the
  equivalence classes $[0] = \{0, 1\} = [1]$ and
  $[x] = \{x\}$ for $x \ne 0, 1$. Then the resulting
  quotient space $X / {\sim}$ is homeomorphic to
  the circle.\footnote{This an example of \emph{gluing}.}
\end{example}
