\chapter{Sept.~26 --- Products and Topological Properties}

\section{Connectedness and Path-Connectedness}
\begin{theorem}
  Assume that each $X_i$ is path-connected, then
  $\prod_{i \in I} X_i$ is path-connected.
\end{theorem}

\begin{proof}
  Fix $x, y \in \prod_{i \in I} X_i$, and define
  $x_i = \pi_i(x)$ and $y_i = \pi_i(y)$.
  Since $x_i, y_i \in X_i$ and $X_i$ is
  path-connected, there exists
  $\gamma_i : [0, 1] \to X_i$ continuous
  such that $\gamma_i(0) = x_i$ and
  $\gamma_i(1) = y_i$. By the universal
  property of products, there exists
  $\gamma : [0, 1] \to \prod_{i \in I} X_i$
  continuous such that $\pi_i \circ \gamma = \gamma_i$.
  This implies
  \[
    (\gamma(0))_i = \gamma_i(0) = x_i,
  \]
  so $\gamma(0) = x$. Similarly
  $\gamma(1) = y$, so $\gamma$ is a path
  from $x$ to $y$. This says that
  $\prod_{i \in I} X_i$ is path-connected.
\end{proof}

\begin{theorem}
  If $\{X_i\}$ are connected, then
  $\prod_{i \in I} X_i$ is connected.
\end{theorem}

\begin{proof}
  Suppose otherwise that $\prod_{i \in I} X_i$
  is not connected, i.e. there exists a
  separation of $\prod_{i \in I} X_i$.
  So let $U, V$ be two disjoint, nonempty, open
  subsets of $\prod_{i \in I} X_i$ such that
  $U \cup V = \prod_{i \in I} X_i$.

  First we claim that there exist
  $x \in U$ and $y \in V$ such $x_i \ne y_i$
  only at a single index $i$. To see this,
  note that $U$ and $V$ contain basis elements
  $U' \subseteq U$ and $V' \subseteq V$. Then
  these basis elements look like
  \[
    U' = U_{i_1} \times \dots \times U_{i_n}
    \times \prod_{j \ne i_1, \dots, i_n} X_j
    \quad \text{and} \quad
    V' = V_{j_1} \times \dots \times V_{j_m}
    \times \prod_{k \ne j_1, \dots, j_m} X_k.
  \]
  Then clearly we may choose $x \in U'$ and
  $y \in V'$ such that they differ in only
  finitely many coordinates. Then we have
  \[
    x = (x_1, \dots, x_n, \dots)
    \quad \text{and} \quad
    y = (y_1, \dots, y_n, \dots),
  \]
  where $x_i = y_i$ for $i > n$.
  Now consider $x_i, y_i$ for $1 \le i \le n$.
  If $x_i = y_i$, then do nothing. Otherwise
  if $x_i \ne y_i$, define
  \[
    y' = (x_1, \dots, x_{i-1}, y_i, x_{i+1}, \dots).
  \]
  Then $y' \in \prod_{i \in I} X_i = U \cup V$,
  so either $y' \in U$ or $y' \in V$ since
  $U \cap V = \varnothing$. If $y' \in V$,
  continue with $y = y'$, and if $y' \in U$,
  change $x = y'$. Do this for the finitely many
  $1 \le i < n$, and we obtain
  $x_i = y_i$ except for a single index $i$.
  Assume without loss of generality that $x_1 \ne y_1$.

  Now define a map $f : X_1 \to \prod_{i \in I} X_i$
  by
  $f(\widetilde{x}) = (\widetilde{x}, x_2, x_3, \dots)$.
  Note that $f$ is continuous by the universal
  property of products (the component maps
  $X_1 \to X_i$ are either the identity if
  $X_i = X_1$ or constant otherwise). Then
  $f(X_1)$ is connected since $X_1$ is connected
  and $f$ is continuous. But now $x \in f(X_1) \cap U$
  and $y \in f(X_1) \cap V$, so
  $U \cap V \ne \varnothing$. Contradiction.
\end{proof}
