\chapter{Sept.~19 --- Miscellaneous Topics}

\section{Local Compactness and One-Point Compactification}
\begin{definition}
  A map $f : X \to Y$ is called \emph{open}
  if for every $U \subseteq X$ open, $f(U)$ is open.
  A \emph{topological embedding} is an injective,
  continuous, and open map.
\end{definition}

\begin{remark}
  This ensures that $f$ with codomain restricted to
  $f(U)$ is a homeomorphism.
\end{remark}

\begin{example}
  Demanding only that $f$ is injective and continuous
  is not enough.
  Let $(X, \mathcal{T})$ be any topological space
  and let $X_{\text{trivial}}$ be $X$ with the
  trivial topology. Then the identity map
  $\id : X \to X_{\text{trivial}}$ is injective,
  continuous, but not open is general if $\mathcal{T}$
  is not trivial. So this is
  \emph{not} a topological embedding.
\end{example}

\begin{theorem}
  If $X$ is Hausdorff and locally compact, then
  $X^+ = X \cup \{\infty\}$ is Hausdorff.
\end{theorem}

\begin{proof}
  Pick $y, y' \in X^+$ with $y \ne y'$. If $y, y' \in X$,
  then since $X$ is Hausdorff, there are disjoint
  open sets $U, U' \subseteq X$ with $y \in U$
  and $y' \in U$. But then $U, U' \subseteq X^+$
  are still open and disjoint in $X^+$ since the identity
  map embeds $X$ into $X^+$. Thus $U, U'$ are
  disjoint open sets separating $y, y'$.

  Now assume without loss of generality that
  $y' = \infty$. By local compactness, there is a
  open set $U \subseteq X$ containing
  $y$ with compact closure. Then observe that
  $(\overline{U})^c \cup \{\infty\}$ is open in $X^+$
  since $\overline{U}$ is compact. Since $U$ is
  also open in $X^+$ and clearly disjoint from
  $(\overline{U})^c \cup \{\infty\}$, we have
  separated $y, \infty$.
\end{proof}

\section{Distance to a Closed Set}
\begin{definition}
  Let $(X, d)$ be a metric space. For $x \in X$
  and $A \subseteq X$, the \emph{distance from $x$
  to $A$} is
  \[
    d(x, A) = \inf\{d(x, a) : a \in A\}.
  \]
\end{definition}

\begin{remark}
  This infimum is achieved in $\R^n$, but not necessarily in a
  more general metric space.
\end{remark}

\begin{prop}
  For any $x \in \R^n$ and closed set $A \subseteq \R^n$, there
  exists $a \in A$ with $d(x, A) = d(x, a)$.
\end{prop}

\begin{proof}
  Pick $a_0 \in A$ and note that $d(x, a_0) < \infty$.
  Then set $R = 2 d(x, a_0) > 0$. Consider
  $\overline{B_R(x)} \cap A$, which is a closed
  and bounded set containing $a_0$, hence compact by
  Heine-Borel.
  Now apply the compact case to $\overline{B_R(x)} \cap A$
  to get $a_{\text{min}}$ with
  $d(x, a_{\text{min}}) = d(x, \overline{B_R(x)} \cap A)$.
  Now any $a \in A$ with
  $a \notin \overline{B_R(x)}$ satisfies
  \[
    d(x, a) \ge R > d(x, a_0) \ge d(x, a_{\text{min}}),
  \]
  and hence it cannot be the minimum. Thus
  we must have $d(x, a_{\text{min}}) = d(x, A)$.
\end{proof}

\begin{remark}
  Define
  \[
    \ell^2(\N) = \left\{\{a_n\}_{n = 1}^\infty \subseteq \R : \sum_{n = 1}^\infty a_n^2 < \infty\right\}.
  \]
  This is an inner product space over $\R$, and in particular
  we can induce a metric
  \[
    d(\{a_n\}, \{b_n\}) =
    \sqrt{
      \sum_{n = 1}^\infty |a_n - b_n|^2
    }
  \]
  to turn $(X, d)$ into a complete metric space.
  Also notice that $\{e_i\}_{i = 1}^\infty$, where
  $e_i$ is the sequence with $1$ in the $i$th position
  and $0$ everywhere else, is
  an orthonormal basis for $\ell^2(\N)$.
\end{remark}

\begin{lemma}
  Define the set
  \[
    A = \left\{\left(1 + \frac{1}{i}\right) e_i\right\}_{i = 1}^\infty.
  \]
  Then $d(\{0\}, A) = 1$. In particular, this is
  an example of a metric space where the infimum
  of the distance from a point to a closed set
  is not achieved.
\end{lemma}

\begin{proof}
  Observe that
  \[
    d\left(\{0\}, (1 + 1 / i) e_i\right) = 1 + \frac{1}{i},
  \]
  and so
  \[
    d(\{0\}, A) =
    \inf\{
      d(\{0\}, (1 + 1 / i) e_i : i \in \N
    \}
    = \inf\left\{1 + \frac{1}{i} : i \in \N\right\} = 1.
  \]
  In particular, this infimum is clearly not
  achieved since $d(\{0\}, (1 + 1 / i) e_i) = 1 + 1 / i > 1$
  for each $i \in \N$. Now
  we show that $A$ is closed by showing that it
  contains its limit points. For this, first observe that
  $d(a, a') \ge 1$ for any $a, a' \in A$ with $a \ne a'$. To see this, we can compute that
  \[
    d\left((1 + 1 / i) e_i, (1 + 1 / j) e_j\right) =
    \sqrt{
      \left(1 + \frac{1}{i}\right)^2 + \left(1 + \frac{1}{j}\right)^2
    } \ge \sqrt{2} \ge 1
  \]
  whenever $i \ne j$.
  Now assume we have $\{a_n\} \subseteq A$ with
  $a_n \to x$. In particular, $\{a_n\}$ must be
  a Cauchy sequence, and so for $\epsilon = 1 / 2$,
  there exists $N_0 \in \N$ such that for all
  $n, m \ge N_0$, we have $d(a_n, a_m) < 1 / 2$.
  But $d(a, a') \ge 1$, so the sequence must stabilize
  after $N_0$, and hence $x = a_n$ for $n \ge N_0$.
  In particular, $x \in A$, so we conclude that
  $A$ is closed. This finishes the example.
\end{proof}

\section{Nested Intersections in Compact Hausdorff Spaces}

\begin{prop}
  Let $X$ be a compact Hausdorff space, and $Y_i \subseteq X$ be closed and connected for $i \in I$.
  Assume the $\{Y_i\}$ are totally ordered,
  i.e. $Y_i \subseteq Y_j$ or $Y_j \subseteq Y_i$ for
  all $i, j \in I$.
  Then $\bigcap_{i \in I} Y_i$ is connected.
\end{prop}

\begin{proof}
  Let $U, V$ be a separation of
  $Y = \bigcap_{i \in I} Y_i$. Then $Y = U \cup V$
  and
  $U, V$ are open in $Y$, disjoint, and nonempty.
  In particular, we can find $U', V'$ open in $X$
  such that $U = U' \cap Y$ and $V = V' \cap Y$.
  However, $U', V'$ may no longer separate $Y$. This
  is why we need the Hausdorff condition. Use the
  next lemma to fix the proof from here, see more
  details in Homework 4.
\end{proof}

\begin{lemma}
  In a compact Hausdorff space, if $C_1, C_2$
  are two compact disjoint sets, then there
  exist $U_1, U_2$ open and disjoint such that
  $C_1 \subseteq U_1, C_2 \subseteq U_2$
  and $U_1 \cap U_2 = \varnothing$.
\end{lemma}

\begin{proof}
  First we show this in the case where
  $C_1 = \{x\}$ is a singleton and $C_2 = C$.
  For all $y \in C$, consider the pair $x$ and $y$.
  Then there exists $U_{x, y}, V_{x, y}$ open
  such that $x \in U_{x, y}$, $y \in V_{x, y}$,
  and $U_{x, y} \cap V_{x, y} = \varnothing$. Observe
  that $\bigcup_{y \in C} V_{x, y}$ is an open cover
  of $C$, so by compactness there exists
  a finite subcover $C \subseteq V = \bigcup_{i = 1}^n V_{x, y_i}$.
  Then $x \in U = \bigcap_{i = 1}^n U_{x, y_i}$,
  which is open as a finite intersection of
  open sets. Also each $U_{x, y_i}$ is disjoint
  from $V_{x, y_i}$, so $U$ is disjoint from $V$.
  Then $U_1 = U$ and $U_2 = V$ are the desired
  open sets.

  Now let $C_1, C_2$ be any two disjoint compact sets.
  For all $x \in C_1$, there are open sets
  $U_{x, C_2}, V_{x, C_2}$ such that $x \in U_{x, C_2}$,
  $C_2 \subseteq V_{x, C_2}$, and $U_{x, C_2} \cap V_{x, C_2} = \varnothing$.
  Then we make the same argument.
  We have
  \[
    C_1 \subseteq \bigcup_{x \in C_1} U_{x, C_2},
  \]
  an open cover of $C_1$, so by compactness there
  is a finite subcover
  $C_1 \subseteq U_1 = \bigcup_{i = 1}^m U_{x_i, C_2}$.
  Then
  \[C_2 \subseteq U_2 = \bigcap_{i = 1}^m V_{x_i, C_2},\]
  which is open as a finite intersection of
  open sets. As before, $U_1 \cap U_2 = \varnothing$
  since each $U_{x_i, C_2}$ is disjoint from
  $V_{x_i, C_2}$. Thus we get a separation
  by disjoint compact sets.
\end{proof}
