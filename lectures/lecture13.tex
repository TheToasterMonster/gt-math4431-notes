\chapter{Oct.~1 --- Tychonoff's Theorem}

\section{Revisiting the Cantor Set}

\begin{prop}
  We have the following properties of the Cantor set $C$:
  \begin{enumerate}[(a)]
    \item $C \ne \varnothing$,
    \item $|C| = \aleph$,
    \item $C$ is closed,
    \item $C$ is compact,
    \item and $C$ is nowhere dense.
  \end{enumerate}
\end{prop}

\begin{proof}
  This was done in Homework 3.
\end{proof}

\begin{prop}
  We have $C \cong \{0, 1\}^\N$, where $\{0, 1\}$ has
  the discrete topology.
\end{prop}

\begin{proof}
  Observe that to obtain $x \in C = \bigcap_{n=1}^\infty C_n$,
  at each step $I_n$ we must choose the left or right
  part of the current subinterval.
  So define the map $f : C \to \{0, 1\}^\N$ by
  \[
    (f(x))_i = \begin{cases}
      0 & \text{if $x$ belongs to the left third}, \\
      1 & \text{if $x$ belongs to the right third}.
    \end{cases}
  \]
  For injectivity, observe that if $a, b \in C$
  with $a \ne b$, then there exists some
  $N \in \N$ such that $3^{-N} < |a - b|$.
  In particular, this means that $a$ and $b$ cannot
  be in the same subinterval in $I_N$, so they
  must have taken different paths at some step. This
  implies that $f(a) \ne f(b)$.
  For surjectivity, given a string $\{0, 1\}^\N$,
  take the intersection of subintervals encoded
  by it. This intersection consists of a unique
  point and is a subset of $C$, so map the string to
  this point. So $f$ is bijective.

  Now we show that $f$ is continuous. Let $U$
  be an open set in $\{0, 1\}^\N$. Note that
  we may assume $U$ is a basis element, i.e.
  \[
    U = U_{\alpha_1} \times \cdots \times U_{\alpha_n} \times \prod_{i \ne \alpha_1, \ldots, \alpha_n} \{0, 1\},
  \]
  where the $U_{\alpha_i} \subseteq \{0, 1\}$ are
  open. Since there are only finitely
  many $U_{\alpha_i}$, we may assume that
  \[
    U = U_1 \times \dots \times U_m \times \{0, 1\}^\N,
  \]
  where $U_i = \{0\}$ or $U_i = \{1\}$. The general
  basis elements may be written as finite unions and
  intersections of sets of this form. Thus observe
  that $f^{-1}(U)$ is the subinterval obtained
  fixing the path given by the $U_i$,
  terminating at the $m$th step and taking all
  remaining choices. This gives some
  subinterval $C \cap I \subseteq C \cap I_{m + 1}$.
  Note that the subintervals of $I_{m + 1}$
  are separated by gaps of width $3^{-(n + 1)}$, so
  we can find an open set $V \subseteq \R$ containing $I$,
  disjoint from the other subintervals of $I_{m + 1}$.
  Then $C \cap I = C \cap V$ is open in $C$, so
  $f^{-1}(U)$ is open in $C$ and thus $f$ is
  continuous.

  Now $C$ is compact, $\{0, 1\}^\N$ is Hausdorff,
  and $f$ is a continuous bijection,
  so $f$ is a homeomorphism.
\end{proof}

\begin{prop}
  For the Cantor set $C$, we have $C^\N \cong C$.
\end{prop}

\begin{proof}
  This is a homework exercise. For the bijection,
  first show that
  $\N \times \N \cong \N$ (traverse $\N \times \N$
  in diagonals). For continuity, note that an open set
  in $C^\N$ has finitely many constraints, and an
  open set in each component of $C^\N$ has finitely
  many constraints. This is still finite many
  constraints in total.
\end{proof}

\section{Tychonoff's Theorem}

\begin{theorem}[Tychonoff]
  Let $(X_i, \mathcal{T}_i)$ be a compact space
  for each $i \in I$. Then the product space
  $\prod_{i \in I} X_i$, with the product
  topology, is compact.
\end{theorem}

\begin{corollary}[Heine-Borel]
  A set $C \subseteq \R^n$ is compact if and only
  if $C$ is closed and bounded.
\end{corollary}

\begin{proof}
  $(\Rightarrow)$ This direction is easy. Immediately
  $C$ is closed since $\R^n$ is Hausdorff. For
  boundedness, cover $C$ by open balls of radius $R$
  centered at the origin, then use compactness.

  $(\Leftarrow)$ We previously showed that
  $[a, b] \subseteq \R$ is compact. Since $C$ is
  bounded, there is some box in $\R^n$ with
  \[
    C \subseteq \prod_{i=1}^n [a_i, b_i],
  \]
  which is compact by Tychonoff's theorem. Thus
  $C$ is a closed subset of a compact set, so $C$ is
  compact.
\end{proof}

\section{Set Theory and the Axiom of Choice}

\begin{example}[Russell's paradox]
  Define the set
  \[
    S = \{\text{all sets $R$} \mid \text{$R$ does not contain itself}\}.
  \]
  Then consider whether $S$ contains itself.
  If $S \in S$, then $S$ cannot contain itself by
  the definition of $S$. But
  if $S \notin S$, then $S$ must contain itself
  by the definition of $S$ again. Both
  cases lead to contradictions.
\end{example}

\begin{remark}
  The above paradox led to the creation of
  the \emph{Zermelo-Fraenkel} ($\ZF$) axioms of set
  theory. More commonly in mathematics, we use $\ZFC$
  set theory, adding the \emph{axiom of choice} ($\AC$) to
  $\ZF$.
\end{remark}

\begin{axiom}[Axiom of choice]
  Let $X_i$ be a collection of nonempty sets. Then
  there exists
  a choice function $f : I \to \bigcup_{i \in I} X_i$
  such that $f(i) \in X_i$ for all $i \in I$.
\end{axiom}

\begin{theorem}[P. Cohen]
  The systems $\ZF \cup \AC$ and
  $\ZF \cup \neg \AC$ are both sound, i.e.
  they do not lead to contradictions. In other words,
  the axiom of choice is independent of $\ZF$.
\end{theorem}

\begin{theorem}
  Tychonoff's theorem is equivalent to the axiom of
  choice, i.e.
  they imply each other.
\end{theorem}
