\chapter{Sept.~10 --- More Compactness}

\section{The Cantor Set}
Define $I_0 = [0, 1]$ and remove the open middle-thirds
interval to get
\[
  I_1 = [0, 1] \setminus (1 / 3, 2 / 3) = [0, 1 / 3] \cup [2 / 3, 1].
\]
Continue by removing the middle thirds of each interval
to get $I_2, I_3, \ldots$. Then the
\emph{Cantor set} is defined to be
$K = \bigcap_{I \ge 0} I_n$. The Cantor set is
compact and uncountable. See more on Homework 3.

\section{The Heine-Borel Theorem}
\begin{theorem}[Heine-Borel]
  If $C \subseteq \R$, then $C$ is compact if
  and only if $C$ is closed and bounded.
\end{theorem}

\begin{proof}
  $(\Rightarrow)$ This direction is easy, see
  Homework 3 for details.

  $(\Leftarrow)$ First we show that
  $[a, b] \subseteq \R$ is compact. Let
  $\{U_\alpha\}_{\alpha \in I}$ be an open
  cover for $[a, b]$, i.e.
  $[a, b] \subseteq \bigcup_{\alpha \in I} U_\alpha$.
  Now define
  \[
    R = \{x \in [a, b] \mid \text{$[a, x]$ has a finite subcover}\}
  \]
  Clearly $a \in R$ since
  $a \in [a, b] \subseteq \bigcup_{\alpha \in I} U_\alpha$,
  so picking any single $U_\alpha$ with $a \in U_\alpha$
  gives a finite subcover for $[a, a] = \{a\}$. The
  goal is now to show that $b \in R$. Observe that
  $a \in R$ implies $R = \varnothing$, and $R \subseteq [a, b]$,
  so
  \[
    s = \sup R
  \]
  exists by the completeness of $\R$. We proceed
  to show that $s \in R$ and then $s = b$, which
  will show that $b \in R$. As $s \in [a, b] \subseteq \bigcup_{\alpha \in I} U_\alpha$,
  we can find $\alpha_s$ such that
  $s \in U_{\alpha_s}$. Since $U_{\alpha_s}$ is
  open, we can find $\delta > 0$ such that
  $(s - \delta, s + \delta) \subseteq U_{\alpha_s}$.
  Then since $s$ is a least upper bound of $R$,
  we can find $r \in R$ such that
  $s - \delta < r \le s$. Now since $r \in R$,
  $[a, r]$ admits a finite subcover
  $\{U_{\alpha_i}\}_{i = 1}^n$. Then
  \[
    [a, s] = [a, r] \cup (s - \delta, s]
    \subseteq \left(\bigcup_{i = 1}^n U_{\alpha_i}\right) \cup U_{\alpha_s}
  \]
  is a finite subcover for $[a, s]$, so $s \in R$.
  Now observe that we actually covered
  \[
    \left[a, s + \frac{\delta}{2}\right]
    = [a, r] \cup (s - \delta, s + \delta)
    \subseteq \left(\bigcup_{i = 1}^n U_{\alpha_i}\right) \cup U_{\alpha_s}
  \]
  in the previous construction. Then
  $s + \delta / 2 \in R$, which contradicts the
  minimality of $s$ unless $s = b$. Thus
  $b \in R$, so $[a, b]$ admits a finite subcover
  and thus $[a, b]$ is compact.

  Now let $C \subseteq \R$ be an arbitrary
  closed and bounded set. Since $C$ is bounded,
  there exists $I = (a, b)$ such that $C \subseteq I$.
  But then $C \subseteq I \subseteq \overline{I} = [a, b]$, so
  $C$ is a closed subset of a compact set, hence
  compact.
\end{proof}

\begin{remark}
  The Heine-Borel theorem also holds more
  generally in $\R^n$.
  A later theorem will
  say that the product of compact sets is compact in
  the product topology,
  and thus we can run the same argument as above but
  with boxes in $\R^n$ instead of intervals.
\end{remark}

\section{The Bolzano-Weierstrass Theorem}
\begin{definition}
  A point $x$ is an \emph{accumulation point}
  for a set $S$ if for all open sets $U$ containing
  $x$, we have
  $(U \setminus \{x\}) \cap S \ne \varnothing$.
\end{definition}

\begin{remark}
  We disallow constant sequences when talking about
  accumulation points.
\end{remark}

\begin{prop}
  Let $\mathrm{Acc}(A)$ be the set of accumulation
  points of a set $A$. Then $\overline{A} = A \cup \mathrm{Acc}(A)$.
\end{prop}

\begin{proof}
  We show that $A \cup \mathrm{Acc}(A)$ is closed,
  which will imply $\overline{A} \subseteq A \cup \mathrm{Acc}(A)$
  by the minimality of the closure. Write
  \[
    (A \cup \mathrm{Acc}(A))^c = A^c \cap \mathrm{Acc}(A)^c.
  \]
  Now assume $x \in A^c \cap \mathrm{Acc}(A)^c$.
  Since $x \notin \mathrm{Acc}(A)$, there exists
  $U_x$ open such that $x \in U_x$ and
  \[(A \setminus \{x\}) \cap U_x = \varnothing.\]
  But also $x \notin A$, so $A \setminus \{x\} = A$
  and $A \cap U_x = \varnothing$. Then we can write
  \[
    (A \cup \mathrm{Acc}(A))^c
    = A^c \cap \mathrm{Acc}(A)^c
    = \bigcup_{x \in A^c \cap \mathrm{Acc}(A)^c} U_x.
  \]
  This is a union of open sets, hence open, and
  so $A \cup \mathrm{Acc}(A)$ is closed.

  For the other direction, assume $x \in A \cup \mathrm{Acc}(A)$.
  If $x \in A$, we are done, so assume
  $x \in \mathrm{Acc}(A) \setminus A$. Now assume
  otherwise that $x \notin \overline{A}$.
  Then $x \in (\overline{A})^c$, which is open.
  Set $U = (\overline{A})^c$, so that
  \[
    U \cap (A \setminus \{x\}) = U \cap A = \varnothing.
  \]
  But then this says that $x$ is not an accumulation
  point, in contradiction.
\end{proof}

\begin{definition}
  We say that a topological space $X$ is \emph{sequentially compact} if
  every bounded sequence has a convergent
  subsequence.
\end{definition}

\begin{theorem}[Bolzano-Weierstrass]
  Any bounded infinite set $S \subseteq \R^n$ has an
  accumulation point.
\end{theorem}

\begin{proof}
  Since $S$ is bounded, find a compact set containing
  $S$. Then apply the later Theorem \ref{thm:compact-bw}.
\end{proof}

\begin{remark}
  In general,
  compactness is \emph{not} equivalent to sequential
  compactness, but both imply the Bolzano-Weierstrass
  theorem. However, in many spaces (including
  metric spaces, in particular), the two notions coincide (and are
  also equivalent to the Bolzano-Weierstrass theorem).
\end{remark}

\begin{theorem}
  A sequentially compact space has the Bolzano-Weierstrass
  property, namely that any bounded infinite set has an
  accumulation point.
\end{theorem}

\begin{proof}
  This is easy, pick a countable subset
  (i.e. a sequence)
  and apply sequential compactness.
\end{proof}

\begin{theorem}
  \label{thm:compact-bw}
  A compact space has the Bolzano-Weierstrass property, namely
  that any infinite set has an accumulation point.
\end{theorem}

\begin{proof}
  Let $A$ be an infinite set in $X$, where $X$ is compact.
  Assume otherwise that $A$ has no accumulation points in $X$.
  Then there is no accumulation point for $A$
  outside of $A$, so $\mathrm{Acc}(A) \subseteq A$. This
  gives
  \[
    \overline{A} = A \cup \mathrm{Acc}(A) = A,
  \]
  so $A$ is closed. Thus $A$ is a closed subset
  of a compact space, hence compact. Now for
  any $a \in A$, pick an open set $U_a$
  such that $a \in U_a$ and
  $U_a \cap (A \setminus \{a\}) = \varnothing$.
  Write $A \subseteq \bigcup_{a \in A} U_a$, and
  by compactness we can find a finite
  subcover $A \subseteq \bigcup_{i = 1}^n U_{a_i}$.
  Then observe that
  \[
    A = A \cap \bigcup_{i = 1}^n U_{a_i}
    = \bigcup_{i = 1}^n (A \cap U_{a_i})
    = \bigcup_{i = 1}^n \{a_i\}
    = \{a_1, \ldots, a_n\},
  \]
  This is in contradiction with $A$ being infinite.
\end{proof}

\begin{remark}
  Usually, this proof goes by showing that compactness
  implies sequential compactness, which then
  implies the Bolzano-Weierstrass property.
  But this proof avoids going through convergent
  sequences.
\end{remark}
