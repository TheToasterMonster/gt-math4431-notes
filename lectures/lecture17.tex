\chapter{Oct.~22 --- Urysohn Metrization}

\section{More Properties of Quotient Spaces}
Recall the universal property of the quotient space.
The following is a corollary:

\begin{corollary}
  Assume $g : X \to Y$ is continuous, and define
  $x \sim y$ if $g(x) = g(y)$. Then the following
  diagram commutes:
  \[
    \begin{tikzcd}
      X \ar[r, "g"] \ar[d, "\pi", swap] & Y \\
      X / {\sim} \ar[ur, "\widetilde{g}"', dashed]
    \end{tikzcd}
  \]
  i.e. there exists a unique continuous map
  $\widetilde{g} : X / {\sim} \to Y$ such that
  $g = \widetilde{g} \circ \pi$.
\end{corollary}

\begin{remark}
  Note that in the above corollary, $\widetilde{g}$
  is injective by construction.
\end{remark}

\section{More on Separation Axioms}

\begin{remark}
  For now we will assume all spaces are $T_1$, i.e.
  every singleton is closed.
\end{remark}

\begin{definition}
  A space is \emph{regular} if for every
  $x \in X$ and $A \subseteq X$ closed with
  $x \notin A$, there exists $U_x, V_A$
  open such that $x \in U_x$, $A \subseteq V_A$,
  and $U_x \cap V_A = \varnothing$.
\end{definition}

\begin{definition}
  A space is \emph{normal} if for
  every $A, B \subseteq X$ closed with $A \cap B = \varnothing$,
  there exist $U_A, V_B$ open such that
  $A \subseteq U_A$, $B \subseteq V_B$, and
  $U_A \cap V_B = \varnothing$.
\end{definition}

\begin{remark}
  Since we assume $T_1$, singletons are closed
  and thus we can think of $x$ as the closed set
  $\{x\}$. This says that normal is stronger than
  regular. Similarly, we can set $A = \{y\}$, which
  is closed by the $T_1$ condition.
  This says that regular is stronger
  than Hausdorff. All of these relationships are
  strict.
\end{remark}

\begin{prop}
  Every metric space is normal.
\end{prop}

\begin{proof}
  Take $A, B \subseteq (X, d)$ closed with
  $A \cap B = \varnothing$. Define $f : X \to \R$
  by
  \[
    f(x) = \frac{d(x, A)}{d(x, A) + d(x, B)}.
  \]
  Recall from Homework 4 that
  \begin{itemize}
    \item if $d(x, A) = 0$ and $A$ is closed, then
      $x \in A$,
    \item and if $x \in A$, $B$ is closed, and
      $A \cap B = \varnothing$, then
      $d(x, B) > 0$.
  \end{itemize}
  So for all $x \in A$, we have
  \[
    f(x) = \frac{d(x, A)}{d(x, A) + d(x, B)}
    = \frac{0}{d(x, B)} = 0
  \]
  since $d(x, A) = 0$ and
  $d(x, B) > 0$. On the other hand,
  if $x \in B$, then
  \[
    f(x) = \frac{d(x, A)}{d(x, A) + d(x, B)}
    = \frac{d(x, A)}{d(x, A)} = 1
  \]
  since $d(x, B) = 0$ and $d(x, A) > 0$. Now note that
  we actually have $f : X \to [0, 1]$, where
  $f|_A \equiv 0$ and $f|_B \equiv 1$. Also note
  that $f$ is continuous since $d(x, A)$ and $d(x, B)$
  are continuous and $d(x, A)$ and $d(x, B)$ are
  never simultaneously zero. So define
  $U = f^{-1}([0, 1 / 4))$ and $V = f^{-1}((3 / 4, 1])$,
  which are open since $f$ is continuous.
  Then $A \subseteq U$ and $B \subseteq V$,
  and $U \cap V = \varnothing$ since they are
  preimages of disjoint sets.
\end{proof}

\section{Urysohn's Metrization Theorem}
\begin{definition}
  A space $X$ has \emph{locally countable basis}
  if for every $x \in X$, there exists
  a countable sequence of open neighborhoods
  $\{U_n\}$, each with $x \in U_n$, such that
  for any basis element $B$ containing $x$, there
  exists $U_n$ such that $x \in U_n \subseteq B$. A
  space $X$ has \emph{countable basis} if $X$ has
  a countable basis.
\end{definition}

\begin{remark}
  Metric spaces have locally countable bases, e.g.
  take $U_n = B(x, 1 / n)$.
\end{remark}

\begin{remark}
  A space with locally countable basis is
  sometimes called \emph{first countable}, and a space
  with countable basis (not local) is sometimes
  called \emph{second countable}. Note that
  $\R^n$ is second countable (for instance
  take open balls with rational centers and
  rational radii).
\end{remark}

\begin{theorem}[Urysohn metrization theorem]
  If $X$ is a normal topological space with
  countable basis, then $X$ is metrizable,
  i.e. there exists a metric $d : X \times X \to \R$
  such that the induced metric topology coincides
  with the given topology on $X$.
\end{theorem}

\begin{lemma}[Urysohn's lemma]
  Let $X$ be a normal topological space. Let
  $A, B \subseteq X$ be closed with $A \cap B = \varnothing$.
  Then there exists $f : X \to [0, 1]$ continuous
  such that $f|_A \equiv 0$ and $f|_B \equiv 1$.\footnote{A function $f$ satisfying these properties is called an \emph{Urysohn function}.}
\end{lemma}

\begin{proof}
  Define $P = \Q \cap [0, 1] = \{p_0, p_1, \dots\}$
  such that $p_0 = 1$ and $p_0 = 0$ (note that
  $P$ is countable, so we may enumerate it).
  We want to find open sets $U_p \subseteq X$
  for $p \in P$ such that $A \subseteq U_p$ and
  \begin{enumerate}
    \item $U_1 = U_{p_0} = B^c$,
    \item and for all $p < q$ in $P$, we have
      $\overline{U_p} \subseteq U_q$.
  \end{enumerate}
  We define these sets via induction. First
  set $U_1 = U_{p_0} = B^c$ as a base case, which
  is open since $B$ is closed and contains $A$
  since $A \cap B = \varnothing$. Now
  for the induction step, assume we have
  $U_{p_0}, \dots, U_{p_n}$ open such that
  $\overline{U_{p_i}} \subseteq U_{p_j}$
  for $i < j$. Consider $p_{n + 1}$. We have
  the following cases:
  \begin{itemize}
    \item Suppose $p_{n + 1} > p = \max\{p_0, \dots, p_n\}$.
      In this case, note that
      $\overline{U_p}$ and $B$ are closed and disjoint
      ($B$ is separated from $U_p$ by an open set,
      so $B$ cannot intersect the closure of $U_p$).
      Then normality guarantees open sets
      $U \supseteq \overline{U_p}$ and
      $V \supseteq B$ such that $U \cap V = \varnothing$.
      Set $U_{p_{n + 1}} = U$.
    \item Now suppose $p_{n + 1} < p = \min\{p_0, \dots, p_n\}$.
      Here note that $U_p^c$ is closed, so
      $A, U_p^c$ are closed and disjoint. Use the
      same normality argument as above to find
      $U_{p_{n + 1}}$.
    \item Otherwise $p_i < p_{n + 1} < p_j$
      for some $0 \le i, j \le n$ (we have finitely
      many $p_k$ at each step, so we can order them
      and find the right location for $p_{n + 1}$).
      Then use the normality argument on
      $\overline{U_{p_i}}$ and $U_{p_j}^c$.
  \end{itemize}
  After the induction, we obtain the countable
  collection of open sets $U_p$ for $p \in P$.
  Also for $p \in \Q \setminus P$, define
  $U_p = \varnothing$
  if $p < 0$ and $U_p = X$ if $p > 1$. This
  defines $U_p$ for all $p \in \Q$. Now define
  $f : X \to \R$ by
  \[
    f(x) = \inf\{p \in \Q \mid x \in U_p\}.
  \]
  First note that if $x \in A$, then
  $x \in A \subseteq U_p$ for every $p \in P$,
  so $f(x) = \inf\{p \in P\} = 0$.
  On the other hand, if $x \in B$, then
  $x \in B = U_1^c \subseteq U_p^c$ for every
  $p \le 1$, i.e. $x \notin U_p$ for $p \le 1$.
  But $x \in U_p = X$ for $p > 1$, so
  $f(x) = \inf\{q \in \Q \mid q > 1\} = 1$.
  So $f$ satisfies the desired properties of an
  Urysohn function.

  It only remains to show that $f$ is continuous,
  which we will do next class.
\end{proof}
