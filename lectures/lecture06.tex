\chapter{Sept.~5 --- Compactness}

\section{Note on the Subspace Topology}
\begin{remark}
  Let $X$ be a topological space with topology
  $\T_X$, and let $Y \subseteq X$ be a subset
  endowed with a topology $\tau$. Suppose that
  for any continuous $f : X \to Z$, there exists
  a continuous $\widetilde{f} : Y \to Z$
  such that the following diagram commutes,
  \[
    \begin{tikzcd}
      X \ar[r, "f"] & Z \\
      Y \ar[ur, "\widetilde{f}"', dashed] \ar[u, "i"] &
    \end{tikzcd}
  \]
  where $i : Y \to X$ is the inclusion map.\footnote{Note that at least set-theoretically, this immediately defines $\widetilde{f} = f|_Y$. But a priori we do not know that $\widetilde{f}$ is continuous.}
  Then in Homework 2 we showed that
  $\T_Y \subseteq \tau$. We can see this as a
  \emph{universal property} for the subspace topology.
\end{remark}

\section{Compactness}
\begin{definition}
  A set $C \subseteq X$ is called \emph{compact}
  if for any \emph{open cover}
  \[
    C \subseteq \bigcup_{\alpha \in I} U_\alpha,
    \quad
    \text{each $U_\alpha$ is open},
  \]
  there exists a finite subcover $C \subseteq \bigcup_{i = 1}^n U_{\alpha_i}$.
\end{definition}

\begin{example} Consider the following:
  \begin{itemize}
    \item In a finite topology, any set is compact.
      This is because any open cover is already finite.
    \item In a discrete space, i.e. $\mathcal{T} = \mathcal{P}(X)$,
      compact sets are precisely the finite sets.
      It is clear that finite sets are compact,
      for each $x$ choose a single open set in the cover
      containing $x$. Conversely, if a set is
      compact, we can pick our open
      cover to contain only singletons, and the
      existence of a
      finite subcover means that the set has
      only finitely many elements.
  \end{itemize}
\end{example}

\begin{theorem}[Heine-Borel]
  Let $C \subseteq \R^n$ be a subset, where
  $\R^n$ is given the metric topology. Then
  $C$ is compact if and only if $C$ is closed and bounded.
\end{theorem}

\begin{proof}
  We postpone this proof until later.
\end{proof}

\begin{lemma}
  Let $X$ be compact. If $Y \subseteq X$ is closed,
  then $Y$ is compact.
\end{lemma}

\begin{proof}
  Let $Y \subseteq X$ closed be given, and assume that
  $Y \subseteq \bigcup_{\alpha \in I} U_\alpha$
  an open cover. Since $Y$ is closed,
  its complement $Y^c$ is open. Then
  \[
    Y^c \cup \bigcup_{\alpha \in I} U_\alpha
  \]
  is an open cover of $X$ since $X = Y \cup Y^c$.
  Since $X$ is compact, there exists a finite subcover
  \[
    X \subseteq Y^c \cup \bigcup_{i = 1}^n U_{\alpha_i}.
  \]
  Now observe that $Y \subseteq X$ and
  $Y \cap Y^c = \varnothing$,
  so actually $Y \subseteq \bigcup_{i = 1}^n U_{\alpha_i}$,
  which is a finite subcover.
\end{proof}

\begin{theorem}
  Let $X$ be compact and $f : X \to Y$
  continuous. Then $f(X)$ is compact.
\end{theorem}

\begin{proof}
  Consider $f(X) \subseteq Y$ and let
  $f(X) \subseteq \bigcup_{\alpha \in I} V_\alpha$, an
  open cover in $Y$. Notice that
  \[
    X = f^{-1}(f(X))
    \subseteq f^{-1}\left(\bigcup_{\alpha \in I} V_\alpha\right)
    = \bigcup_{\alpha \in I} f^{-1}(V_\alpha).
  \]
  Note that each $f^{-1}(V_\alpha)$ is open in $X$ since $f$ is
  continuous and $V_\alpha$ is open in $Y$, so
  this is in fact an open cover of $X$. Thus since $X$ is
  compact, we may extract a finite subcover
  \[
    X \subseteq \bigcup_{i = 1}^n f^{-1}(V_{\alpha_i}).
  \]
  Then we see that
  \[
    f(X) \subseteq f\left(\bigcup_{i = 1}^n f^{-1}(V_{\alpha_i})\right)
    \subseteq \bigcup_{i = 1}^n V_{\alpha_i},
  \]
  which is a finite subcover of $f(X)$. Therefore
  $f(X)$ is compact.
\end{proof}

\begin{theorem}
  Assume $\{C_j\}_{j = 1}^m$ are compact subsets
  of $X$. Then $\bigcup_{j = 1}^m C_j$ is compact.
\end{theorem}

\begin{proof}
  Assume $\bigcup_{j = 1}^m C_j \subseteq \bigcup_{\alpha \in I} U_\alpha$,
  an open cover. Observe this is also an open
  cover of $C_j$ for each $1 \le j \le m$, so
  we can extract a finite subcover, i.e. we can
  find $\alpha_{j, 1}, \dots, \alpha_{j, n_j}$
  with
  \[
    C_j \subseteq \bigcup_{i = 1}^{n_j} U_{\alpha_{j, i}}.
  \]
  Then we see that
  \[
    \bigcup_{j = 1}^m C_j
    \subseteq \bigcup_{j = 1}^m \bigcup_{i = 1}^{n_j} U_{\alpha_{j, i}},
  \]
  which is still a finite union. This is then a finite
  subcover of $\bigcup_{j = 1}^m C_j$, so
  $\bigcup_{j = 1}^m C_j$ is compact.
\end{proof}

\begin{theorem}[Weierstrass]
  Let $f : [a, b] \to \R$ be continuous. Then
  $f([a, b])$ is bounded, and moreover
  there exist $x_{\mathrm{max}}, x_{\mathrm{min}} \in [a, b]$
  such that $f(x_{\mathrm{max}}) \ge f(x) \ge f(x_{\mathrm{min}})$
  for all $x \in [a, b]$.
\end{theorem}

\begin{proof}
  Since $f$ is continuous and $[a, b]$ is compact
  (by Heine-Borel), $f([a, b]) \subseteq \R$ is
  compact. Thus by Heine-Borel, $f([a, b])$ is
  bounded. In particular, we can find $M, m$
  such that
  \[
    m \le f(x) \le M \quad \text{for all } x \in [a, b].
  \]
  For the second part, observe that $f([a, b])$
  is bounded and nonempty, so $s = \sup f([a, b])$.
  Since this is the supremum, there must exist
  $y_i \in f([a, b])$ such that $y_i \to s$
  as $i \to \infty$. Now observe that
  $f([a, b])$ is closed by Heine-Borel, and in
  particular it contains its limit points. Thus
  we obtain $s \in f([a, b])$. Then pick
  $x_{\mathrm{max}} \in f^{-1}(\{s\}) \subseteq [a, b]$,
  which will satisfy
  $f(x_{\mathrm{max}}) = s \ge f(x)$ for all $x \in [a, b]$.
  by construction.

  The argument for finding
  $x_{\mathrm{min}} \in [a, b]$ is similar.
\end{proof}

\begin{theorem}
  Let $X$ be a topological space, $K \subseteq X$
  compact, and $f : K \to \R$ a continuous function.
  Then $f$ is bounded over $K$
  and attains its minimum and maximum on $K$.
\end{theorem}

\begin{proof}
  The same argument goes through, replacing
  $[a, b]$ by the compact set $K$.
\end{proof}
