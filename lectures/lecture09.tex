\chapter{Sept.~17 --- Compactification}

\section{Motivation for Compactification}
Let $(X, \mathcal{T})$ be a topological space which is not compact.
Usually we assume $X$ is Hausdorff, and the goal is
to find a compact space which looks like $X$, i.e.
compactify $X$.

\begin{remark}
  The naive idea is to take the trivial topology
  on $X$ in place of $\mathcal{T}$, getting
  $X_{\text{trivial}}$. This is compact, the identity map
  $\mathrm{id} : X \to X_{\text{trivial}}$ is continuous and
  bijective, but it is not a homeomorphism.
  This is bad because we forget all the topological
  structure on $X$, for instance every sequence
  converges to every point in
  $X_{\text{trivial}}$. We would like to
  compactify $X$ while keeping as much structure
  as possible.
\end{remark}

\begin{example}
  Let $X = (0, 1)$ with the metric topology.
  Take $Y = [0, 1]$ with the metric topology, so
  $X$ embeds into $Y$ by the inclusion map.\footnote{By $X$ \emph{embeds into} $Y$, we mean that there is a continuous injection from $X$ to $Y$.} Note
  that $Y$ is compact by Heine-Borel.
\end{example}

\begin{example}
  Let $X = (0, 1)$ with the metric topology. Take
  $Y = \mathbb{S}^1 \subseteq \R^2$ to be the unit circle,
  where $\R^2$ has the metric topology. Then
  $X$ embeds into $Y$ by the
  stereographic projection (technically $\R$ is
  embedded but $\R$ is homeomorphic to $(0, 1)$
  by the arctangent) by adding one point
  at the north pole.
\end{example}

\begin{example}
  For the open unit disk in $\R^2$, we can add its
  boundary to get the closed unit disk as a
  compactification (the closed unit disk is
  compact by Heine-Borel). This adds uncountably
  many points. An alternative is to add only a
  single point at infinity, and identify the boundary
  with this point.
\end{example}

\section{One-Point Compactification}
\begin{definition}
  For a topological space $X$, the \emph{one-point compactification}
  (or \emph{Alexandroff compactification}) of $X$ is
  the set $X^+ = X \cup \{\infty\}$ with topology
  generated by the basis
  \[
    \{\text{$U \subseteq X$ open}\}
    \cup \{K^c \mid K \subseteq X \text{ is compact}\}
  .\]
\end{definition}

\begin{remark}
  One way to think about the
  one-point compactification is that we are forcing
  all unbounded sequences in $X$ to converge to
  to the new point $\infty$.
\end{remark}

\begin{theorem}
  For a Hausdorff topological space $X$, its
  one-point compactification $X^+$ is compact.
\end{theorem}

\begin{proof}
  Let $\{O_\alpha\}$ be an open cover of $X^+$, i.e.
  $X^+ = \bigcup_{\alpha \in I} O_\alpha$. Some
  open set must contain the point $\infty$, and
  each open set contains a basis element, so
  there exists $\alpha' \in I$ such that
  $O_{\alpha'}$ contains $K^c$, for some
  $K \subseteq X$ compact. Now since $K \subseteq X \subseteq X^+$,
  we see that $K \subseteq \bigcup_{\alpha \in I} (O_\alpha \cap X)$ is an open cover of $K$ (note that the Hausdorff
  condition implies that every compact $K$ is closed
  in $X$, and thus the $K^c$ basis elements are open in $X$).
  So by compactness, there exists a finite subcover
  $K \subseteq \bigcup_{\alpha \in I_{\text{finite}}} (O_{\alpha} \cap X)$.
  Then
  \[
    X^+ = O_{\alpha'} \cup K
    \subseteq O_{\alpha'} \cup \bigcup_{\alpha \in I_{\text{finite}}} (O_{\alpha} \cap X)
    \subseteq \bigcup_{\alpha \in I_{\text{finite}}} O_{\alpha}
  \]
  since $K^c \subseteq O_{\alpha'}$. This is a
  finite subcover, so $X$ is compact.
\end{proof}

\begin{remark}
  In analysis, we sometimes speak of a sequence
  diverging to $\infty$, i.e. the sequence
  eventually escapes any compact set. This is precisely
  convergence to $\infty$ in the one-point
  compactification.
\end{remark}

\begin{theorem}
  Assume $X$ is a Hausdorff, locally compact but not
  compact
  topological space.\footnote{A space $X$ is \emph{locally compact} if for every $x \in X$, there exists $U$ open with $x \in U$ such that $\overline{U}$ is compact.} Then the inclusion map
  $\mathrm{id} : X \to X^+$ is a dense,
  continuous embedding.\footnote{The embedding is \emph{dense} if $\overline{X} = X^+$.}
\end{theorem}

\begin{proof}
  First clearly $\mathrm{id} : X \to \mathrm{id}(X) \subseteq X^+$ is
  injective. Now we show that $\id$ is continuous.
  It is enough to show that the preimage of basis
  elements of $X^+$ is open in $X$. If
  $U \subseteq X$ is open, then $\id^{-1}(U) = U \subseteq X$ is
  clearly open. Otherwise consider
  $K^c \cup \{\infty\}$ for $K \subseteq X$ compact.
  Then we have
  \[
    \id^{-1}(K^c \cup \{\infty\})
    = K^c \subseteq X.
  \]
  Since $X$ is Hausdorff, the compact set $K$ is
  closed, and so $K^c$ is open. Thus $\id$
  is continuous.

  Finally we show density, i.e. $\overline{X} = X^+$,
  where the closure is taken in $X^+$. To do this, suppose
  otherwise that $\overline{X} \ne X^+$. Clearly $X \subseteq \overline{X}$,
  so if $\overline{X} \ne X^+$, we must have
  $\overline{X} = X$ (since $X^+ = X \cup \{\infty\}$).
  But then $X$ is closed in $X^+$, which is
  compact, so $X$ is also compact in $X^+$ since
  $X$ and thus $X^+$ is Hausdorff (since
  $X$ is locally compact).
  This implies (exercise) that $X$ itself
  is compact.
  Contradiction.
\end{proof}

\begin{remark}
  If $X$ is already compact, then $X^c = \{\infty\}$
  is open in $X^+$. Obviously $X^+$ is compact since
  $X$ is and every sequence converging to $\infty$
  must eventually be constant. In particular,
  $X^+$ must be disconnected, and the extra point
  $\infty$ just sits there to the side.
\end{remark}

\begin{remark}
  Due to the above, we must assume that
  $X$ is not compact in order to get a dense embedding.
\end{remark}

\begin{example}
  Consider the space $X = [0, 1)$. Then the
  one-point compactification of $X$ is $[0, 1]$.
\end{example}
