\chapter{Nov.~19 --- Fundamental Group, Part 3}

\section{Simply-Connectedness}
\begin{remark}
  When working with the fundamental group, we
  often assume that $X$ is path-connected. If this
  is the case, the fundamental group is independent
  of the basepoint $x_0$. This is because any
  loop at another basepoint $y_0$ can be converted
  to a loop at based at $x_0$ by concatenating with
  a path from $x_0$ to $y_0$ and back, which exists
  if we assume that $X$ is path-connected.
  Thus we can just write $\pi_1(X)$.
\end{remark}

\begin{definition}
  A space $X$ is \emph{simply-connected} if
  $\pi_1(X) = 0$.
\end{definition}

\begin{remark}
  There are simply-connected spaces
  which are not contractible, e.g. $S^n$ for
  $n \ge 2$, where
  \[
    S^n = \left\{(x_1, \dots, x_{n + 1}) \in \R^{n + 1} : \sum_{i = 1}^{n + 1} x_i^2 = 1\right\}.
  \]
  Any loop in $S^n$ can be contracted to a point,
  e.g. the north pole, but all of $S^n$ cannot be.
  Note that a loop into $S^n$ must miss at least
  a single point, we can assume this is the south
  pole. Then $S^n$ punctured at the south pole is
  homeomorphic to a disk and thus contractible, so
  we can contract loops.
\end{remark}

\begin{prop}
  The circle $S^1$ is not simply-connected. In
  particular, $\pi_1(S^1) = \Z$.
\end{prop}

\begin{remark}
  The homotopy class of a loop in the fundamental
  group measures its \emph{index} or
  \emph{winding number} around a hole. This is
  of use, for instance, in multivariable calculus
  or complex analysis.
\end{remark}

\begin{remark}
  If $\gamma : [0, 1] \to \C$ is a differentiable
  curve, then we can compute the index via
  \[
    \oint_{\gamma} \frac{1}{z}\, dz
    = 2\pi i \cdot \mathrm{Ind}(\gamma, 0)
  \]
  by the residue theorem. This method only works for
  very nice curves $\gamma$ (namely differentiable).
  The winding number can be calculated for much
  more general continuous curves via the fundamental
  group.
\end{remark}

\begin{example}
  Consider the vector field $F$ on
  $\R^2 \setminus \{0\}$ given by
  \[
    F(x, y) =
    \left(-\frac{y}{x^2 + y^2}, \frac{x}{x^2 + y^2}\right).
  \]
  One can compute that
  \[
    \curl F = \frac{\partial F_1}{\partial y} - \frac{\partial F_2}{\partial x} = 0,
  \]
  so $F$ is \emph{locally conservative}. Recall that
  $F$ is \emph{conservative} if $F = \nabla \phi$
  for some potential function $\phi$, and in this
  case the integral of $F$ along any closed loop is
  $0$. One can check
  that conservative implies locally conservative,
  but the converse is not true. One can check that
  \[
    \int_{\gamma} F \cdot dr \ne 0
  \]
  where $\gamma(t) = (\cos t, \sin t)$ for
  $t \in [0, 2\pi]$, so $F$ is not conservative.
  The issue is that $F$ is not defined at the origin.
  In vector analysis, \emph{Poincar\'e's lemma}
  says that a locally conservative vector field is
  globally conservative when its domain
  is simply-connected.
\end{example}

\section{Applications of the Fundamental Group}
\begin{definition}
  A \emph{retract} of $X$ onto $A \subseteq X$ is a
  map $r : A \to X$ such that $r(X) \subseteq A$
  and $r|_A = \id_A$.
\end{definition}

\begin{remark}
  Let $i : A \to X$ be the inclusion map. Then
  we have the following diagram:
  \[
    \begin{tikzcd}
      A \ar[r, "i"] \ar[rr, bend right=20, "\id_A", swap]& X \ar[r, "r"] & A
    \end{tikzcd}
  \]
  i.e. we have $r \circ i = \id_A$. Note
  that $r$ is a projection in the sense that
  $r \circ r = r$.
\end{remark}

\begin{example}[Functoriality]
  Let $X$ be a topological space and
  $\pi_0(X)$ be the set of path-connected
  components in $X$. We can view $\pi_0$ as a
  map from the category of topological spaces to
  the category of sets which sends
  $X \mapsto \pi_0(X)$. Note that what makes
  $\pi_0$ a \emph{functor} is the following
  diagram:
  \[
    \begin{tikzcd}
      X \ar[r, "f"] \ar[d, "\pi_0"'] & Y \ar[d, "\pi_0"] \\
      \pi_0(X) \ar[r, "f_*"', dashed] & \pi_0(Y)
    \end{tikzcd}
  \]
  i.e. a continuous map $f : X \to Y$ on topological
  spaces
  induces a function $f_* : \pi_0(X) \to \pi_0(Y)$
  on sets. Furthermore, if $f$ is a homeomorphism,
  then $f$ is a bijection (an isomorphism of sets).
\end{example}

\begin{remark}
  The \emph{Galois group} of a field extension is
  another example of a functor.
\end{remark}

\begin{example}
  Note that if $f : X \to Y$ is continuous, then
  $\gamma \mapsto f \circ \gamma$ is a map which
  preserves the homotopy classes of loops. In
  particular, $f$ induces a map $f_* : \pi_1(X) \to \pi_1(Y)$,
  giving us:
  \[
    \begin{tikzcd}
      X \ar[r, "f"] \ar[d, "\pi_1"'] & Y \ar[d, "\pi_1"] \\
      \pi_1(X) \ar[r, "f_*"', dashed] & \pi_1(Y)
    \end{tikzcd}
  \]
  Now recall the retract and inclusion maps.
  We have the following diagram by using
  functoriality twice:
  \[
    \begin{tikzcd}
      A \ar[r, "i", swap] \ar[rr, bend left=20, "\id_A"] \ar[d, "\pi_1"'] & X \ar[d, "\pi_1"] \ar[r, "r", swap] & A \ar[d, "\pi_1"] \\
      \pi_1(A) \ar[r, "i_*"', dashed, swap] \ar[rr, bend right=20, "\id_{\pi_1(A)}", dashed, swap] & \pi_1(X) \ar[r, "r_*", dashed] & \pi_1(A)
    \end{tikzcd}
  \]
\end{example}

\begin{theorem}[No retract]
  There is no retract from the closed disk
  $B^2$ to $S^1$.
\end{theorem}

\begin{proof}
  Assume not, i.e. there exists a retract $r : B^2 \to S^1$.
  Then we have the diagram:
  \[
    \begin{tikzcd}
      S^1 \ar[r, "i"] \ar[d, "\pi_1"'] & B^2 \ar[d, "\pi_1"] \ar[r, "r"] & S^1 \ar[d, "\pi_1"] \\
      \pi_1(S^1) = \Z \ar[r, "i_*"', dashed] & \pi_1(B^2) = 0 \ar[r, "r_*", swap, dashed] & \pi_1(S^1) = \Z
    \end{tikzcd}
  \]
  Since $r \circ i = \id_{S^1}$, we must
  have $(r \circ i)_* = \id_\Z$. But
  $r_* \circ i_*$ factors through $0$, so it can
  only have a single point in
  the image since $i_*(\Z) = \{0\}$. This
  says that $(r \circ i)_* \ne r_* \circ i_*$,
  a contradiction.
\end{proof}

\begin{theorem}[Brouwer's fixed point theorem]
  Any continuous map $f : B^2 \to B^2$ must
  have a fixed point, i.e. there exists $x_0 \in B^2$
  such that $f(x_0) = x_0$.
\end{theorem}

\begin{proof}
  Assume not. Then we can define the
  map $\widetilde{f} : B^2 \to S^1$ by sending
  the $x$ to the intersection of the ray from
  $f(x)$ to $x$ with $S^1$.\footnote{Note that we cannot defined $\widetilde{f}$ if there is some $x_0$ with $f(x_0) = x_0$. In this case, the ray is not well-defined.} One can show that
  $\widetilde{f}$ is continuous, and that
  $\widetilde{f}|_{S^1}$ is the identity. So
  $\widetilde{f}$ is a retract from $B^2$ to $S^1$.
  This cannot happen by the previous no retract
  theorem, in contradiction.
\end{proof}

\begin{remark}
  The same result immediately follows for any
  shape homotopy equivalent to the disk $B^2$.
\end{remark}
