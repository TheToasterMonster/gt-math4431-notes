\chapter{Aug.~27 --- Closed Sets, Continuity, the Subspace Topology}

\section{Closed Sets}
\begin{definition}
  A set $S \subseteq X$ is called a \emph{closed set}
  if $S^c = X \setminus S$ is open.
\end{definition}

\begin{example}
  In $\R$, observe that
  $[a, b]^c = (-\infty, a) \cup (b, \infty)$,
  which is a union of open sets and thus open.
  Thus the closed intervals $[a, b] \subseteq \R$ are closed.
\end{example}

\begin{remark}
  This is not a dichotomy. Sets can be
  both open and closed (\emph{clopen}), or even neither.
  Trivially, if $X$ is any topological space,
  then $\varnothing$ and $X$ are both open and closed.
\end{remark}

\begin{example}
  Let $X = \{0, 1\}$ and $\mathcal{T} = \mathcal{P}(X)$.
  Then $\{0\}$ is both open and closed.
\end{example}

\begin{example}
  Let $X = \{1, 2, 3\}$ and
  $\mathcal{T} = \{\varnothing, X, \{1\}, \{1, 2\}\}$.
  Then $\{2\}$ is neither open nor closed.
\end{example}

Recall the following De Morgan's laws from set theory:
\begin{prop}[De Morgan's laws]
  Let $I$ be an index set and $\{A_i\}_{i \in I}$
  be sets. Then
  \[\left( \bigcup_{i \in I} A_i \right)^c = \bigcap_{i \in I} A_i^c
    \quad \text{and} \quad
    \left( \bigcap_{i \in I} A_i \right)^c = \bigcup_{i \in I} A_i^c.
  \]
\end{prop}

\begin{corollary}
  In a topological space $(X, \mathcal{T})$, we have:
  \begin{enumerate}[(i)]
    \item $\varnothing, X$ are closed.
    \item if $\{A_i\}_{i \in I}$ are closed,
      then $\bigcap_{i \in I} A_i$ is closed,
    \item and if $\{A_i\}_{i = 1}^n$ are closed,
      then so is $\bigcup_{i = 1}^n A_i$.
  \end{enumerate}
  This gives a dual characterization of a topology.
\end{corollary}

\begin{proof}
  $(i)$ We have $\varnothing^c = X \in \mathcal{T}$
  and $X^c = \varnothing \in \mathcal{T}$.

  $(ii)$ Note that
  \[
    \left( \bigcap_{i \in I} A_i \right)^c
    = \bigcup_{i \in I} A_i^c.
  \]
  As each $A_i$ is closed, we have
  $A_i^c \in \mathcal{T}$ is open, and hence
  $\bigcup_{i \in I} A_i^c \in \mathcal{T}$ is open.
  So $\bigcap_{i \in I} A_i$ is closed.

  $(iii)$ Observe that
  \[
    \left( \bigcup_{i = 1}^n A_i \right)^c
    = \bigcap_{i = 1}^n A_i^c.
  \]
  Each $A_i$ is closed, so $A_i^c$ is open.
  Thus $\bigcap_{i = 1}^n A_i^c$ is open,
  and so $\bigcup_{i = 1}^n A_i$ is closed.
\end{proof}

\section{Properties of Continuity}
Recall that a function $f : (X, \mathcal{T}_X) \to (Y, \mathcal{T}_Y)$
is continuous if for every
$O \in \mathcal{T}_Y$, we have $f^{-1}(O) \in \mathcal{T}_X$.

\begin{theorem}
  A function $f : X \to Y$ is continuous if and
  only if for every $C$ closed in $Y$,
  $f^{-1}(C)$ is closed in $X$.
\end{theorem}

\begin{proof}
  $(\Rightarrow)$ Let $C \subseteq Y$ be closed.
  Note that
  \[
    f^{-1}(C) = \{x \in X \mid f(x) \in C\},
  \]
  so we have
  \[
    (f^{-1}(C))^c = \{x \in X \mid f(x) \notin C\}
    = \{x \in X \mid f(x) \in C^c\}
    = f^{-1}(C^c)
  .\]
  Since $C$ is closed, $C^c$ is open and so
  $f^{-1}(C^c) = (f^{-1}(C))^c$ is open. Thus $f^{-1}(C)$ is closed.

  $(\Leftarrow)$ Assume $S \subseteq Y$ is open.
  Note that
  \[
    (f^{-1}(S))^c
    = \{x \in X \mid f(x) \in S\}^c
    = \{x \in X \mid f(x) \notin S\}
    = \{x \in X \mid f(x) \in S^c\}
    = f^{-1}(S^c).
  \]
  Since $S$ is open, $S^c$ is closed and
  so $f^{-1}(S^c) = (f^{-1}(S))^c$ is closed by
  assumption.
  Thus $f^{-1}(S)$ is open, and so we see that
  $f$ is continuous.
\end{proof}

\begin{theorem}[Composition theorem]
  Let $(X, \mathcal{T}_X)$, $(Y, \mathcal{T}_Y)$,
  and $(Z, \mathcal{T}_Z)$ be topological spaces.
  Let
  \[f : X \to Y \quad \text{and} \quad g : Y \to Z\]
  be continuous
  functions. Then $g \circ f : X \to Z$
  is continuous.
\end{theorem}

\begin{proof}
  Let $S \subseteq Z$ be open. It suffices to
  show that $(g \circ f)^{-1}(S) \subseteq X$ is open.
  Note that
  \begin{align*}
    (g \circ f)^{-1}(S)
    = \{x \in X \mid (g \circ f)(x) \in S\}
    &= \{x \in X \mid f(x) \in g^{-1}(S)\} \\
    &= \{x \in X \mid x \in f^{-1}(g^{-1}(S))\}
    = f^{-1}(g^{-1}(S)).
  \end{align*}
  Now as $g$ is continuous, $g^{-1}(S)$ is open in $Y$.
  Finally as $f$ is continuous, $f^{-1}(g^{-1}(S))$ is
  open in $X$.
\end{proof}

\begin{theorem}
  Assume $X = \bigcup_{\alpha \in I} U_\alpha$
  for open sets $U_\alpha$ and let $f : X \to Y$.
  Assume that $f|_{U_\alpha} : U_\alpha \to Y$
  is continuous for each $\alpha \in I$.
  Then $f$ is continuous.
\end{theorem}

\begin{proof}
  Let $S \subseteq Y$ be open, and it suffices
  to show that $f^{-1}(S)$ is open. Observe that
  \[
    f^{-1}(S) = f^{-1}(S) \cap X
    = f^{-1}(S) \cap \left( \bigcup_{\alpha \in I} U_\alpha \right)
    = \bigcup_{\alpha \in I} (f^{-1}(S) \cap U_\alpha)
    = \bigcup_{\alpha \in I} f|_{U_\alpha}^{-1}(S).
  \]
  The $f|_{U_\alpha}$ are continuous,
  so each $f|_{U_\alpha}^{-1}(S)$ is open. Thus
  $f^{-1}(S)$ is open as a union of open sets.
\end{proof}

\begin{theorem}[Gluing]
  Assume $X, Y$ are topological spaces and
  $A, B \subseteq X$ are open. Suppose $f_1 : A \to Y$
  and $f_2 : B \to Y$ are continuous, and that
  $f_1 \equiv f_2$ on $A \cap B$. Then
  $f : A \cup B \to Y$ defined by
  \[
    f(x) =
    \begin{cases}
      f_1(x) & \text{if } x \in A, \\
      f_2(x) & \text{if } x \in B
    \end{cases}
  \]
  is continuous.
\end{theorem}

\begin{proof}
  Let $S \subseteq Y$ be open, it suffices
  to show that $f^{-1}(S)$ is open. Observe that
  \[
    f^{-1}(S) = f_1^{-1}(S) \cup f_2^{-1}(S).
  \]
  Both $f_1^{-1}(S)$ and $f_2^{-1}(S)$ are open
  since $f_1$ and $f_2$ are continuous, so
  $f^{-1}(S)$ is open as their union.
\end{proof}

\section{Subspace Topology}

\begin{definition}
  Let $(X, \mathcal{T}_X)$ be a topological space
  and $S \subseteq X$ a set. The \emph{subspace topology}
  on $S$ is defined as follows:
  $O \subseteq S$ is open if there exists $U \subseteq X$
  open in $X$ such that $U = O \cap S$.
\end{definition}

\begin{example}
  Let $\R$ be given the metric topology and
  $S = [0, 1]$.
  \begin{itemize}
    \item The set $[0, 1]$ is not open
      in $\R$, but it is open in the subspace
      topology on $S$ since
      $[0, 1] = S \cap (-1, 2)$.
    \item The set $[0, 1 / 2)$ is neither open nor closed
      in $\R$, but
      $[0, 1 / 2) = S \cap (-1 / 2, 1 / 2)$,
      so it is open in $S$.
  \end{itemize}
\end{example}

\begin{theorem}
  The subspace topology is indeed a topology.
\end{theorem}

\begin{proof}
  Let $(X, \mathcal{T}_X)$ be a topological
  space and $S \subseteq X$ be given the subspace
  topology.

  $(i)$ We have $S = S \cap X$ and
  $\varnothing = \varnothing \cap X$, so
  $S, \varnothing$ are open in $S$.

  $(ii)$ Let $\{U_\alpha\}_{\alpha \in I}$
  be open in the subspace topology. Then
  for every $\alpha \in I$, there exists
  $O_\alpha \in \mathcal{T}$
  such that $U_\alpha = S \cap O_\alpha$. Then
  \[
    \bigcup_{\alpha \in I} U_\alpha
    = \bigcup_{\alpha \in I} (S \cap O_\alpha)
    = S \cap \left( \bigcup_{\alpha \in I} O_\alpha \right).
  \]
  The $\{O_\alpha\}_{\alpha \in I}$ are
  open in $X$, so their union is open in $X$.
  Thus $\bigcup_{\alpha \in I} U_\alpha$ is open
  in the subspace topology.

  $(iii)$ Let $\{U_i\}_{i = 1}^n $ be open in the
  subspace topology. Then there are
  $O_i$ for $1 \le i \le n$ with
  $U_i = S \cap O_i$. Then we have
  \[
    \bigcap_{i = 1}^n U_i
    = \bigcap_{i = 1}^n (S \cap O_i)
    = S \cap \left( \bigcap_{i = 1}^n O_i \right).
  \]
  As the $O_i \in \mathcal{T}$ are open,
  $\bigcap_{i = 1}^n O_i$ is open in $X$.
  Thus $\bigcap_{i = 1}^n U_i$ is open in the
  subspace topology.
\end{proof}

\begin{theorem}
  Assume $f : X \to Y$ is a continuous function and
  $S \subseteq X$ a subspace. Then
  $f|_S : S \to Y$ is continuous, where
  $S$ is equipped with the subspace topology.
\end{theorem}

\begin{proof}
  Let $O \subseteq Y$ be an open set, it suffices
  to show that $f|_S^{-1}(O)$ is open in the subspace
  topology. But observing that
  $f|_S^{-1}(O) = f^{-1}(O) \cap S$
  immediately shows that $f|_S^{-1}(O)$ is open in $S$
  since $f^{-1}(O)$ is open in $X$ due to the
  continuity of $f$.
\end{proof}

\begin{remark}
  The subspace topology is the smallest topology
  on $S$ such that the inclusion map
  $i : S \to X$ given by
  $i(s) = s$ is a continuous function.
\end{remark}

\begin{remark}
  Let $X$ be a topological space with subspaces
  $Y \subseteq X$ and $Z \subseteq Y$. Then
  the subspace topology on $Z$ induced by the
  subspace $Y$ is the same as the
  subspace topology on $Z$ induced directly by $X$.
\end{remark}

\begin{remark}
  A topological space can have a subspace
  homeomorphic to itself. For instance, consider
  $\R$ and $(-\pi / 2, \pi / 2)$ with
  a homemorphism given by the tangent function.
\end{remark}
