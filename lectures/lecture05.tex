\chapter{Sept.~3 --- Path-Connectedness}

\section{More on Connectedness}
\begin{remark}
  The intervals $[a, b] \subseteq \R$ are homeomorphic
  to $[0, 1]$ for any $a < b$. We can take
  $f : [a, b] \to [0, 1]$ defined by
  \[
    f(x) = \frac{1}{b - a}(x - a)
  \]
  for instance as a homemorphism.
\end{remark}

\begin{lemma}
  \label{lem:image-connected}
  If $X$ is connected and $f : X \to Y$
  is continuous, then $f(X)$ is connected.
\end{lemma}

\begin{proof}
  This is part of Homework 2.
\end{proof}

\begin{corollary}
  The plane $\R^2$ is connected.
\end{corollary}

\begin{proof}
  Express $\R^2$ as the union of
  horizontal and vertical lines. Each line is the
  image of $\R$ and is thus connected by
  Lemma $\ref{lem:image-connected}$.
  Also any pair of
  horizontal and vertical lines must intersect,
  so we can use Corollary \ref{cor:union-connected}
  to conclude that the union $\R^2$ is connected.
\end{proof}

\begin{remark}
  We can extend this to $\R^3$ by embedding
  planes (copies of $\R^2$), and similarly for $\R^n$.
\end{remark}

\begin{prop}
  The unit circle $\mathbb{S}^1 \subseteq \R^2$
  is connected.
\end{prop}

\begin{proof}
  Define $\gamma : [0, 2\pi] \to \R^2$
  by $\gamma(t) = (\cos t, \sin t)$.
  The image of $\gamma$ is precisely $\mathbb{S}^1$.
\end{proof}

\begin{prop}
  \label{prop:equivalence-connected}
  Define a relation $\sim$ on $X$ by $x \sim y$
  if there exists a connected subset
  $S \subseteq X$ such that $x, y \in S$.
  Then $\sim$ is an equivalence relation.
\end{prop}

\begin{proof}
  For reflexivity, fix $x \in X$ and let
  $S$ be the largest connected set containing
  $x$ (this exists since we know at least
  $\{x\}$ must be
  connected). Then $x \in S$, so $x \sim x$.

  For symmetry, fix $x, y \in X$. If $x \sim y$,
  then there exists a connected set $S$ such that
  $x, y \in S$. But then $y, x \in S$, so we see that
  $y \sim x$.

  For transitivity, assume that $x \sim y$
  and $y \sim z$. Then there exists $S_1$
  connected such that $x, y \in S_1$ and $S_2$
  connected such that $y, z \in S_2$. Notice
  that $S_1 \cap S_2 \ne \varnothing$ since
  $y \in S_1 \cap S_2$. Then $S_1 \cup S_2$ is
  connected by Theorem \ref{thm:union-connected}
  and $x, y, z \in S_1 \cap S_2$. In particular,
  $x, z \in S_1 \cap S_2$ and thus $x \sim z$.

  So we see that $\sim$ is an equivalence relation.
\end{proof}

\begin{definition}
  Let the equivalence relation $\sim$ be defined
  on $X$ as in Proposition \ref{prop:equivalence-connected}.
  Then we can write $X$ as the disjoint union of
  the equivalence classes of $\sim$.
  These
  equivalence classes are called
  the \emph{connected components} of $X$.
\end{definition}

\begin{remark}
  The connected components of a space are defined
  solely
  via topologies, so they must be invariant under
  homeomorphism.
\end{remark}

\begin{example}
  The letter $S$, sitting in $\R^2$, is not
  homeomorphic to the letter $T$. There is a point
  we can remove from $T$ to give three connected
  components, but removing any point from $S$ gives
  at most two such connected components.
\end{example}

\section{Path-Connectedness}

\begin{remark}
  Connectedness is usually a very difficult
  property to verify. This motivates
  \emph{path-connectedness}.
\end{remark}

\begin{definition}
  A set $S$ is \emph{path-connected} if
  for all $x, y \in S$, there exists a continuous
  map $\gamma : [0, 1] \to S$ such that
  $\gamma(0) = x$ and $\gamma(1) = y$. Here
  $[0, 1]$ is given the usual metric topology.
\end{definition}

\begin{lemma}
  If $S$ is path-connected, then $S$ is connected.
\end{lemma}

\begin{proof}
  This is part of Homework 2.
\end{proof}

\begin{remark}
  Unlike connectedness, it is immediately
  obvious that $\R^n$ is path-connected. Simply
  take the line segment between any two points.
  Then we can conclude connectedness by the previous
  lemma.
\end{remark}

\begin{example}
  There are spaces which are connected but not
  path-connected.
  \begin{itemize}
    \item Consider the
      \emph{topologist's sine curve}, given by
      the union of the vertical segment
      $\{(0, y) \mid -1 \le y \le 1\}$ and the image of
      $(0, \infty)$ under $x \mapsto (x, \sin(1/x))$, is
      an example of such a space. See Homework 2
      for more details.
    \item Consider the \emph{cone} $C$ in
      $\R^2$ defined by ($(0, 1)$ denotes an
      open interval unless otherwise specified)
      \[
        C = ([0, 1] \times \{0\})
        \cup (K \times [0, 1])
        \cup (\{0\} \times [0, 1]),
      \]
      where $K = \{1 / n : n \in \N\}$.
      Note that $C$ is clearly path-connected
      and hence also connected.
      Then define the space
      \[
        D = C \setminus (\{0\} \times (0, 1)),
      \]
      which is now not path-connected (consider
      the point $(0, 1) \in D$)
      but still connected.
  \end{itemize}
\end{example}

\begin{remark}
  Observe the following:
  \begin{itemize}
    \item One can define \emph{path-connected components} in a similar
      manner as connected components.
    \item A continuous image of a path-connected
      space is path-connected. Simply compose
      the curve with the continuous map, which is
      now a path in the image.
    \item The union of path-connected spaces
      sharing a point is path-connected.
      Take two curves to the common point
      and concatenate them using the pasting lemma.
    \item In $\R^n$, connectedness is equivalent
      to path-connectedness. In general, this
      holds if you can get a basis of only connected
      sets.
  \end{itemize}
\end{remark}

\begin{remark}
  Recall from homework that if $f : [0, 1] \to [0, 1]$
  is continuous, then $f$ has a fixed point,
  i.e. there exists $c \in [0, 1]$ with $f(c) = c$.
  This follows from a clever use of the intermediate
  value theorem. Now consider a more topological
  perspective. Consider the diagonal
  $\{(x, x) \mid x \in [0, 1]\}$ and look at the
  graph of $f$, which is contained in the closed
  unit square. This graph is path-connected as the
  image of a path-connected set and so
  there is a path between the points $(0, f(0))$
  and $(1, f(1))$. But then this path must
  intersect the diagonal at some point, which
  gives a fixed point.
\end{remark}

\begin{theorem}(Brouwer fixed point theorem)
  Let $K$ be a closed, bounded, and convex
  set in $\R$. Then any continuous map $f : K \to K$
  has a fixed point, i.e. there exists $c \in K$
  such that $f(c) = c$.
\end{theorem}

\begin{remark}
  One can see the existence of the Nash equilibrium
  as a consequence of this theorem.
\end{remark}

\begin{remark}
  In $\R^2$, this theorem follows from the
  following claim. Let $X = \maps(\sphere^1, \sphere^1)$ be the set of all
  continuous maps from $\mathbb{S}^1$ to itself.
  Then Brouwer's fixed point theorem in $\R^2$
  follows from the following:
\end{remark}

\begin{theorem}
  The space $\maps(\sphere^1, \sphere^1)$ is not
  path connected.
\end{theorem}
