\chapter{Nov.~26 --- The Van Kampen Theorem}

\section{The Borsuk-Ulam Theorem}

\begin{theorem}[Borsuk-Ulam]
  For any continuous function $f : S^2 \to \R^2$,
  there exist two antipodal points
  $x, -x \in S^2$ such that $f(x) = f(-x)$.
\end{theorem}

\begin{proof}
  Assume not, so that $f(x) - f(-x) \ne 0$ for all $x \in S^2$.
  Then we can define a map
  $g : S^2 \to S^1$ by
  \[
    g(x) = \frac{f(x) - f(-x)}{\|f(x) - f(-x)\|}.
  \]
  Note that $g(-x) = -g(x)$. Consider the
  equator in $S^2$, which is homeomorphic to $S^1$.
  By considering the restriction to the
  equator we obtain a closed loop
  $g|_{S^1} : S^1 \to S^1$. Fix $p \in S^1$. Then
  $g(p)$ and $g(-p)$ are antipodal points in $S^1$
  by the observation. By connectedness,
  $g(S^1)$ contains the entire arc between
  $g(p)$ and $g(-p)$. Also by the reflection property,
  moving past $g(-p)$ along this arc to $g(p)$ must
  complete the entire loop.\footnote{In particular, the arc between $g(p)$ and $g(-p)$ completely determines the continued arc from $g(-p)$ to $g(p)$.} In particular,
  $g(S^1) = S^1$ and so $g(S^1)$ is a
  non-nullhomotopic loop in $S^1$.

  However, $S^2$ is simply connected,\footnote{Even simpler, since we are specifically working with the equator, it suffices to consider the upper hemisphere which is contractible. This allows us to avoid the need for the computation $\pi_1(S^2) = 0$.} so
  there is a homotopy from the equator
  $S^1 \subseteq S^2$ to a constant loop, and
  composing this homotopy with $g$ gives a homotopy
  from $g(S^1)$ to a constant loop.
  Contradiction.
\end{proof}

\begin{remark}
  The Borsuk-Ulam theorem holds more generally for
  continuous maps $f : S^{2n} \to \R^{2n}$.
\end{remark}

\section{The Van Kampen Theorem}

\begin{example}
  Consider the \emph{wedge sum} $S^1 \lor S^1$,
  which we form by fixing a single point in each
  copy of $S^1$ and identifying the pair. Let $\alpha, \beta$
  be the loops obtained by traversing each copy of
  $S^1$ once. Then any loop in
  $S^1 \lor S^1$ can be written as a word in
  $\{\alpha, \beta, \alpha^{-1}, \beta^{-1}\}$, e.g.
  \[
    \alpha^{-1} * \beta * \beta * \alpha.
  \]
  This is an example of an element of the
  free group on $\alpha, \beta$. In particular,
  $\pi_1(S^1 \lor S^1) = F_2 = \Z * \Z$.
\end{example}

\begin{definition}
  The \emph{free group} on $n$ generators
  $x_1, \dots, x_n$, denoted $F_n$, is the collection
  of finite words on $x_1^{\pm 1}, \dots, x_n^{\pm 1}$
  with the operation of concatenation, where
  words are identified up to relaxation, i.e.
  \[
    x_k^{\alpha_1} x_k^{\alpha_2} = x_k^{\alpha_1 + \alpha_2}
  \]
  for a fixed $k$ and $\alpha_1, \alpha_2 \in \{\pm 1\}$.
\end{definition}

\begin{remark}
  In general, the free groups
  are not abelian when there is more than one
  generator.
\end{remark}

\begin{definition}
  The \emph{free product} of groups $G, H$, denoted
  $G * H$, is the free group on their generators,
  modulo the relations of the generators within
  $G$ and $H$.
\end{definition}

\begin{theorem}[Van Kampen]
  Let $A, B$ be path-connected topological spaces
  and suppose that $A \cap B$ is also path-connected.
  Then if $X = A \cup B$ and $x_0 \in A \cap B$,
  we have
  \[
    \pi_1(X, x_0) = 
    \pi_1(A, x_0) *_{\pi_1(A \cap B, x_0)} \pi_1(B, x_0)
  \]
  where the above product is the
  \emph{almalgamated free product}.
\end{theorem}

\begin{example}
  Recall the space $S^1 \lor S^1$. If
  $A, B \subseteq S^1 \cap S^1$ are the
  two circles, then note that
  $A \cap B$ consists of a single point, and
  thus $\pi_1(A \cap B) = 0$. Thus
  we simply have $\pi_1(S^1 \lor S^1) = \Z * \Z$.
\end{example}

\begin{example}
  Let $A, B$ be path-connected with
  $A \cap B$ path-connected, and let $X = A \cup B$.
  Let
  \begin{align*}
    \pi_1(A)
    &= \langle u_1, \dots, u_k \mid \alpha_1, \dots, \alpha_\ell\rangle, \\
    \pi_1(B)
    &= \langle v_1, \dots, v_m \mid \beta_1, \dots, \beta_n\rangle, \\
    \pi_1(A \cap B)
    &= \langle w_1, \dots, w_r \mid \gamma_1, \dots, \gamma_q\rangle.
  \end{align*}
  Note that by functoriality, the inclusions
  $I : A \cap B \to A$ and $J : A \cap B \to B$
  induce homomorphisms
  $I_* : \pi_1(A \cap B) \to \pi_1(A)$ and
  $J_* : \pi_1(A \cap B) \to \pi_1(B)$. Then
  the van Kampen theorem asserts that
  \[
    \pi_1(X)
    = \langle u_1, \dots, u_k, v_1, \dots, v_m \mid
    \alpha_1, \dots, \alpha_\ell, \beta_1, \dots, \beta_n,
    I_*(w_1) J_*^{-1}(w_1), \dots, I_*(w_r) J_*^{-1}(w_r)\rangle.
  \]
  In particular, we are identifying
  $I_*(w_i)$ and $J_*(w_i)$ since they represent
  the same loop in $A \cap B$.
\end{example}

\begin{example}
  Recall that $\pi_1(\mathbb{T}^2) = \Z \times \Z$.
  We can also compute this via the van Kampen theorem.
  Consider the fundamental polygon of
  $\mathbb{T}^2$,
  let $B$ be an interior circle contained in the
  square and
  $A = \mathbb{T}^2 \setminus B$. Then $B$ is
  simply-connected, so $\pi_1(B) = 0$. On the
  other hand, $A$ is homotopic to the
  boundary of the square, which is homeomorphic to
  $S^1 \lor S^1$. Thus $\pi_1(A) = \Z * \Z$ from
  the previous example.

  Now notice that the intersection $A \cap B$ is
  a circle, and thus $I_*(A \cap B) = \alpha \beta \alpha^{-1} \beta^{-1}$,
  where $\alpha, \beta$ are curves
  represent the sides of the square. Thus by
  the van Kampen theorem, we see that
  \[
    \pi_1(X) = (\Z * \Z) / {\langle \alpha \beta \alpha^{-1} \beta^{-1}\rangle} = \Z \times \Z,
  \]
  since the relation $\alpha \beta = \beta \alpha$
  lets us collect the like terms together, forming
  two copies of $\Z$.
\end{example}

\begin{remark}
  The van Kampen theorem yields a relatively
  easy proof of the \emph{Nielson-Schreier theorem},
  which says that any subgroup of a free group is
  also free. The idea is to realize the free group
  as the fundamental group of a \emph{bouquet}
  (wedge sum) of circles, and compute the subgroup
  via van Kampen.
\end{remark}
