\documentclass[12pt, letterpaper, oneside]{book}
\usepackage[margin={0.6in, 0.75in}]{geometry}
\usepackage{microtype}
% \usepackage{kpfonts}
\usepackage{amsmath, amssymb, amsthm}
\usepackage{parskip}
\usepackage[many]{tcolorbox}
\usepackage{footnote}
\usepackage{cancel}
\usepackage{titlesec}
\usepackage{pgffor}
\usepackage[shortlabels]{enumitem}
\usepackage{hyperref}

\usepackage[overload]{textcase}

\renewcommand{\chaptername}{Lecture}
\newtheorem{axiom}{Axiom}[chapter]
\newtheorem{theorem}{Theorem}[chapter]
\newtheorem{prop}{Proposition}[chapter]
\newtheorem{corollary}{Corollary}[theorem]
\newtheorem{lemma}{Lemma}[chapter]
\theoremstyle{definition}
\newtheorem{definition}{Definition}[chapter]
\newtheorem{exercise}{Exercise}[chapter]
\newtheorem{example}{Example}[definition]
\newtheorem*{remark}{Remark}

\tcbset{sharp corners, breakable, enhanced, parbox=false}
\newtcolorbox{mybox}[3][]
{
  colframe = #2!150,
  colback  = #2!5,
  coltitle = #2!0!white,  
  title    = {#3},
  #1,
}

\titleformat{\chapter}[display]
    {\normalfont\huge\bfseries}{\chaptertitlename\ \thechapter}{20pt}{\Huge}
\titlespacing*{\chapter}{0pt}{0pt}{40pt}

\newcommand{\R}{\mathbb{R}}
\newcommand{\N}{\mathbb{N}}
\newcommand{\Z}{\mathbb{Z}}
\newcommand{\C}{\mathbb{C}}
\newcommand{\Q}{\mathbb{Q}}
\newcommand{\F}{\mathbb{F}}

\newcommand{\T}{\mathcal{T}}
\newcommand{\B}{\mathcal{B}}

\DeclareMathOperator{\Vol}{Vol}
\DeclareMathOperator{\Int}{int}
\DeclareMathOperator{\area}{area}
\DeclareMathOperator{\curl}{curl}

\title{MATH 4431: Introduction to Topology}
\author{Frank Qiang\\Instructor: Asaf Katz}
\date{Georgia Institute of Technology\\Fall 2024}

\begin{document}
  \maketitle

  \begingroup
  \let\cleardoublepage\clearpage
  \tableofcontents
  \endgroup

  % \foreach \i in {00, 01, 02, 03, 04, ..., 50} {%
  %   \edef\FileName{lectures/lecture\i.tex}%     The % here are necessary to eliminate any
  %   \IfFileExists{\FileName}{%  spurious spaces that may get inserted
  %      \input{\FileName}%       at these points
  %   }
  % }
  \chapter{Aug.~20 --- Review of Metric Spaces}

\section{Metric Spaces}
Recall the definition of a \emph{metric space}:

\begin{definition}
  Given a set $X$, a function $d : X \times X \to \R$
  is called a \emph{metric} if
  \begin{enumerate}[(i)]
    \item (strong positivity)
      $d(x, y) \ge 0$ for all
      $x, y \in X$, and $d(x, y) = 0$ if and only if
      $x = y$,
    \item (symmetry) $d(x, y) = d(y, x)$,
    \item and (triangle inequality)
      $d(x, z) \le d(x, y) + d(y, z)$
      for all $x, y, z \in X$.
  \end{enumerate}
\end{definition}

\begin{example}
  For any set $X$, we can define the
  \emph{discrete metric}
  by
  \[
    d(x, y) =
    \begin{cases}
      1 & \text{if $x \ne y$}, \\
      0 & \text{otherwise}.
    \end{cases}
  \]
  Verify as an exercise that this satisfies
  the triangle inequality.
\end{example}

\begin{example}
  The Euclidean metric in $\R^n$ is
  \[
    d(\overline{x}, \overline{y})
    = \sqrt{\sum_{i = 1}^n (x_i - y_i)^2}
  \]
  where $\overline{x} = (x_1, \ldots, x_n)$
  and $\overline{y} = (y_1, \ldots, y_n)$.
\end{example}

\section{Open Sets}
\begin{definition}
  The \emph{open ball} of radius $R > 0$ around
  $x_0 \in X$ is
  \[
    B_R(x_0) = \{y \in X \mid d(x_0, y) < R\}
  .\]
  Given a set $S \subseteq X$, a point $x_0$
  is called an interior point of $S$ if there exists
  $r > 0$ such that $B_r(x_0) \subseteq S$.
  The set $S$ is called \emph{open} if all of its
  points are interior points.
\end{definition}

\begin{prop}
  The open ball $B_R(x)$ is open.
\end{prop}

\begin{proof}
  Fix an arbitrary $y \in B_R(x)$, and observe that
  it suffices to show that $y$ is an interior point.
  Take $r = R - d(x, y)$, and
  first note that $r > 0$ since
  $d(x, y) < R$. Now note that for all $z \in B_r(y)$,
  we have
  \[
    d(x, z) \le d(x, y) + d(y, z)
    < d(x, y) + (R - d(x, y))
    = R,
  \]
  so that $z \in B_R(x)$.
  Thus $B_r(y) \subseteq B_R(x)$, and so $y$ is an interior point.
\end{proof}

\begin{corollary}
  $B_R(x) = \bigcup_{y \in B_R(x)} B_{r_y}(y)$,
  where $r_y = R - d(x, y)$.
\end{corollary}

\begin{proof}
  We have $B_{r_y}(y) \subseteq B_R(x)$
  for each $y \in B_R(x)$,\footnote{Using the argument from the previous proposition.}
  and so
  $\bigcup_{y \in B_R(x)} B_{r_y}(y) \subseteq B_R(x)$.
  For the reverse inclusion simply observe that
  $y \in B_{r_y}(y) \subseteq \bigcup_{y \in B_R(x)} B_{r_y}(y)$ for
  each $y \in B_R(x)$.
\end{proof}

\begin{prop}
  In a metric space $(X, d)$, the following
  are true:
  \begin{enumerate}[(i)]
    \item $\varnothing, X$ are open,
    \item if $\{S_i\}_{i \in I}$ are open, then
      $\bigcup_{i \in I} S_i$ is open,
    \item and if $\{S_i\}_{i = 1}^n$ are open, then
      $\bigcap_{i = 1}^n S_i$ is open.
  \end{enumerate}
\end{prop}

\begin{proof}
  $(i)$ The empty set is open vacuously.
  To see that $X$ is open, simply take $R = 1$
  for any $x \in X$.

  $(ii)$ Fix $x \in \bigcup_{i \in I} S_i$
  arbitrary, so there exists $i_0 \in I$ with
  $x \in S_{i_0}$. Since $S_{i_0}$ is open,
  $x$ is an interior point and thus there exists
  $r > 0$ such that $B_r(x) \subseteq S_{i_0}$.
  But then $B_r(x) \subseteq S_{i_0} \subseteq \bigcup_{i \in I} S_i$,
  so $x$ is an interior point of
  $\bigcup_{i \in I} S_i$ also and thus
  $\bigcup_{i \in I} S_i$ is open.

  $(iii)$ Now assume $x \in \bigcap_{i = 1}^n S_i$.
  Then for each $1 \le i \le n$, there exists
  $r_i > 0$ such that $B_{r_i}(x) \subseteq S_i$.
  Then we can choose
  \[
    r = \min\{r_1, \ldots, r_n\} > 0,
  \]
  so that $B_r(x) \subseteq B_{r_i}(x) \subseteq S_i$
  for each $1 \le i \le n$. Thus
  $B_r(x) \subseteq \bigcap_{i = 1}^n S_i$
  and $\bigcap_{i = 1}^n S_i$ is open.
\end{proof}

\begin{remark}
  The above argument for the finite intersection
  property requires that there are only finitely
  many $r_i$. Otherwise it may very well be that
  $r = \inf\{r_i\} = 0$ and the argument fails.
\end{remark}

  \chapter{Aug.~22 --- Topology, Basis, Continuity}

\section{Topological Spaces}
\begin{definition}
  A \emph{topology} $\mathcal{T} \subseteq \mathcal{P}(X)$ is a
  collection of sets such that
  \begin{enumerate}[(i)]
    \item $\varnothing, X \in \mathcal{T}$,
    \item for any index set $I$, if $\{s_i\}_{i \in I} \subseteq \mathcal{T}$,
      then $\bigcup_{i \in I} s_i \in \mathcal{T}$ (closure under arbitrary union),
    \item and if $\{s_i\}_{i = 1}^n \subseteq \mathcal{T}$, then
      $\bigcap_{i = 1}^n s_i \in \mathcal{T}$ (closure under finite intersection).
  \end{enumerate}
  A set with a topology, i.e. a pair
  $(X, \mathcal{T})$,
  is called a \emph{topological space}.
  Elements of $\mathcal{T}$ are called
  \emph{open sets}.
\end{definition}

\begin{example}
  The following are examples of topologies on a set $X$:
  \begin{itemize}
    \item The trivial topology: $\mathcal{T} = \{\varnothing, X\}$.
    \item The discrete topology: $\mathcal{T} = \mathcal{P}(X)$.\footnote{Note that the discrete topology is induced by the discrete metric.}
    \item If $(X, d)$ is a metric space, then
      $\mathcal{T} = \{\text{collection of metrically open sets}\}$
      is a topology on $X$.
  \end{itemize}
\end{example}

\begin{remark}
  Not every topology is induced by a metric.
  For instance consider the trivial topology
  on $\R$.
\end{remark}

\section{Basis for a Topology}
\begin{definition}
  A collection $\mathcal{B} \subseteq \mathcal{P}(X)$
  is called a \emph{basis} if
  \begin{enumerate}[(i)]
    \item $\bigcup_{b \in \mathcal{B}} b = X$, i.e.
      $\mathcal{B}$ is a covering of $X$,
    \item and if $x \in b_1 \cap b_2$ for any
      $b_1, b_2 \in B$, then there exists $b_3 \in \mathcal{B}$
      such that $x \in b_3$
      and $b_3 \subseteq b_1 \cap b_2$.
  \end{enumerate}
\end{definition}

\begin{theorem}
  Given a set $X$ and a basis $\mathcal{B}$, define
  \[
    \mathcal{T}_\mathcal{B} = \left\{\bigcup_{i \in I} s_i \mid \text{$I$ is any index set and $\{s_i\}_{i \in I} \subseteq \mathcal{B}$}\right\}.
  \]
  Then $\mathcal{T}_\mathcal{B}$ is a topology on $X$.
\end{theorem}

\begin{proof}
  First observe that $\varnothing, X \in \mathcal{T}_\mathcal{B}$:
  Picking $I = \varnothing$ gives
  $\bigcup_{i \in I} s_i = \varnothing \in \mathcal{T}_\mathcal{B}$
  and picking $I = \mathcal{B}$ gives
  $\bigcup_{b \in \mathcal{B}} b = X \in \mathcal{T}_\mathcal{B}$
  by the covering property of a basis.

  Now assume $\{s_i\}_{i \in I} \subseteq \mathcal{T}_\mathcal{B}$.
  For each $i \in I$, we have $s_i \in \mathcal{T}_\mathcal{B}$
  and so there exists an index set $J_i$
  such that $s_i = \bigcup_{j \in J_i} b_j$, where
  the $b_j \in \mathcal{B}$. Then
  \[
    \bigcup_{i \in I} s_i = \bigcup_{i \in I} \bigcup_{j \in J_i} b_j,
  \]
  which is a union of elements of
  $\mathcal{B}$ and hence is in
  $\mathcal{T}_{\mathcal{B}}$.

  Finally assume $\{s_i\}_{i = 1}^n \subseteq \mathcal{T}_\mathcal{B}$.
  Now as each $s_i \in \mathcal{T}_\mathcal{B}$,
  there exists $J_i$ such that
  $s_i = \bigcup_{j \in J_i} b_j$.
  Then
  \[
    \bigcap_{i = 1}^n s_i = \bigcap_{i = 1}^n \bigcup_{j \in J_i} b_j.
  \]
  Now assume $x \in \bigcap_{i = 1}^n s_i = \bigcap_{i = 1}^n \bigcup_{j \in J_i} b_j$.
  For each $1 \le i \le n$, there exists
  $j_i \in J_i$ such that $x \in b_{j_i}$.
  Hence $x \in \bigcap_{i = 1}^n b_{j_i}$.
  Now by induction on the intersection property of a basis,
  we can find $b_x \in \B$ with
  \[
    x \in b_x \subseteq \bigcap_{i = 1}^n b_{j_i}
  \]
  Also observe that
  \[\bigcap_{i = 1}^n b_{j_i} \subseteq \bigcap_{i = 1}^n \bigcup_{j \in J_i} b_j
    = \bigcap_{i = 1}^n s_i
  \]
  by construction,
  so we may write
  \[
    \bigcap_{i = 1}^n s_i = \bigcup_{x \in \bigcap_{i = 1}^n s_i} b_x \in \mathcal{T}_\mathcal{B}
  \]
  as a union of elements of $\mathcal{B}$.
\end{proof}

\begin{definition}
  A \emph{subbasis} $\mathcal{B} \subseteq \mathcal{P}(X)$ is a collection
  of sets such that
  $\bigcup_{b \in \mathcal{B}} b = X$.
\end{definition}

\begin{remark}
  One may define a basis $\widetilde{\mathcal{B}}$ from a
  subbasis $\mathcal{B}$ by adding all finite intersections
  of elements of $\mathcal{B}$.
  We get the covering property for free and adding
  the finite intersections gives us the
  intersection property of a basis.
\end{remark}

\begin{example}
  For $\R$ with the Euclidean metric, the following
  are bases for the standard topology:
  \begin{itemize}
    \item $\{B_R(x) \mid x \in \R, R > 0\}$.
    \item $\{B_R(x) \mid x \in \R, R > 0, R \in \Q\}$.
      For this use the fact that $\Q$ is dense in $\R$.
  \end{itemize}
  In particular this shows that a basis for a topology
  is not unique in general.
\end{example}

\section{Continuous Functions}

\begin{definition}
  Let $(X, \mathcal{T}_X)$ and $(Y, \mathcal{T}_Y)$
  be two topological spaces. A function $f : X \to Y$
  is called \emph{continuous} if for any
  $O \in \mathcal{T}_Y$, we have
  $f^{-1}(O) \in \mathcal{T}_X$, i.e. the
  preimage of an open set is open.\footnote{Recall that $f^{-1}(O) = \{x \in X \mid f(x) \in O\}$.}
\end{definition}

\begin{example}
  Let $X$ be equipped with the trivial
  topology $\{\varnothing, X\}$ and let
  $\R$ be equipped with the standard topology.
  Then the only continuous functions $f : X \to \R$
  are the constant functions $f : x \mapsto c$
  for fixed $c \in \R$. To see this, observe that
  \begin{itemize}
    \item $x \mapsto c$ is continuous since
      any open set in $\R$ either contains $c$ or does not,
      and so the preimage is either $X$ or $\varnothing$.
    \item Suppose $f(x_1) = y_1$ and $f(x_2) = y_2$.
      Let $\epsilon = |y_1 - y_2|$ and observe that
      $x_1 \in f^{-1}(B_\epsilon(y_1))$ while
      $x_2 \notin f^{-1}(B_\epsilon(y_1))$, so
      $f^{-1}(B_\epsilon(y_1))$ is not open
      in $X$ despite $B_\epsilon(y_1)$ being open in $\R$.
  \end{itemize}
\end{example}

\begin{example}
  Let $X$ have the discrete topology
  $\mathcal{T} = \mathcal{P}(X)$ and let
  $\R$ have the standard topology.
  Then all functions $X \to \R$ are continuous
  since any preimage is a subset of $X$ and thus
  in $\mathcal{P}(X)$.
\end{example}

\begin{remark}
  In a way, the trivial topology has too few
  open sets while the discrete topology has too many.
\end{remark}

\begin{definition}
  Two topological spaces $(X, \mathcal{T}_X)$ and
  $(Y, \mathcal{T}_Y)$ are
  \emph{topologically equivalent} or
  \emph{homeomorphic} if there exists a
  bijection $f : X \to Y$ such that
  $f$ and $f^{-1}$ are continuous.
\end{definition}

\begin{remark}
  A bijective function $f$ being continuous does
  not necessarily imply that its inverse $f^{-1}$ is.
\end{remark}

\begin{example}
  Consider $(-\pi / 2, \pi / 2)$ equipped with the
  Euclidean metric. This is homeomorphic to $\R$
  equipped with the Euclidean metric.\footnote{Note that $(-\pi / 2, \pi / 2)$ is bounded while $\R$ is not.} One
  homeomorphism is given by $\tan : (-\pi / 2, \pi / 2) \to \R$.
\end{example}

  \chapter{Aug.~27 --- Closed Sets, Continuity, the Subspace Topology}

\section{Closed Sets}
\begin{definition}
  A set $S \subseteq X$ is called a \emph{closed set}
  if $S^c = X \setminus S$ is open.
\end{definition}

\begin{example}
  In $\R$, observe that
  $[a, b]^c = (-\infty, a) \cup (b, \infty)$,
  which is a union of open sets and thus open.
  Thus the closed intervals $[a, b] \subseteq \R$ are closed.
\end{example}

\begin{remark}
  This is not a dichotomy. Sets can be
  both open and closed (\emph{clopen}), or even neither.
  Trivially, if $X$ is any topological space,
  then $\varnothing$ and $X$ are both open and closed.
\end{remark}

\begin{example}
  Let $X = \{0, 1\}$ and $\mathcal{T} = \mathcal{P}(X)$.
  Then $\{0\}$ is both open and closed.
\end{example}

\begin{example}
  Let $X = \{1, 2, 3\}$ and
  $\mathcal{T} = \{\varnothing, X, \{1\}, \{1, 2\}\}$.
  Then $\{2\}$ is neither open nor closed.
\end{example}

Recall the following De Morgan's laws from set theory:
\begin{prop}[De Morgan's laws]
  Let $I$ be an index set and $\{A_i\}_{i \in I}$
  be sets. Then
  \[\left( \bigcup_{i \in I} A_i \right)^c = \bigcap_{i \in I} A_i^c
    \quad \text{and} \quad
    \left( \bigcap_{i \in I} A_i \right)^c = \bigcup_{i \in I} A_i^c.
  \]
\end{prop}

\begin{corollary}
  In a topological space $(X, \mathcal{T})$, we have:
  \begin{enumerate}[(i)]
    \item $\varnothing, X$ are closed.
    \item if $\{A_i\}_{i \in I}$ are closed,
      then $\bigcap_{i \in I} A_i$ is closed,
    \item and if $\{A_i\}_{i = 1}^n$ are closed,
      then so is $\bigcup_{i = 1}^n A_i$.
  \end{enumerate}
  This gives a dual characterization of a topology.
\end{corollary}

\begin{proof}
  $(i)$ We have $\varnothing^c = X \in \mathcal{T}$
  and $X^c = \varnothing \in \mathcal{T}$.

  $(ii)$ Note that
  \[
    \left( \bigcap_{i \in I} A_i \right)^c
    = \bigcup_{i \in I} A_i^c.
  \]
  As each $A_i$ is closed, we have
  $A_i^c \in \mathcal{T}$ is open, and hence
  $\bigcup_{i \in I} A_i^c \in \mathcal{T}$ is open.
  So $\bigcap_{i \in I} A_i$ is closed.

  $(iii)$ Observe that
  \[
    \left( \bigcup_{i = 1}^n A_i \right)^c
    = \bigcap_{i = 1}^n A_i^c.
  \]
  Each $A_i$ is closed, so $A_i^c$ is open.
  Thus $\bigcap_{i = 1}^n A_i^c$ is open,
  and so $\bigcup_{i = 1}^n A_i$ is closed.
\end{proof}

\section{Properties of Continuity}
Recall that a function $f : (X, \mathcal{T}_X) \to (Y, \mathcal{T}_Y)$
is continuous if for every
$O \in \mathcal{T}_Y$, we have $f^{-1}(O) \in \mathcal{T}_X$.

\begin{theorem}
  A function $f : X \to Y$ is continuous if and
  only if for every $C$ closed in $Y$,
  $f^{-1}(C)$ is closed in $X$.
\end{theorem}

\begin{proof}
  $(\Rightarrow)$ Let $C \subseteq Y$ be closed.
  Note that
  \[
    f^{-1}(C) = \{x \in X \mid f(x) \in C\},
  \]
  so we have
  \[
    (f^{-1}(C))^c = \{x \in X \mid f(x) \notin C\}
    = \{x \in X \mid f(x) \in C^c\}
    = f^{-1}(C^c)
  .\]
  Since $C$ is closed, $C^c$ is open and so
  $f^{-1}(C^c) = (f^{-1}(C))^c$ is open. Thus $f^{-1}(C)$ is closed.

  $(\Leftarrow)$ Assume $S \subseteq Y$ is open.
  Note that
  \[
    (f^{-1}(S))^c
    = \{x \in X \mid f(x) \in S\}^c
    = \{x \in X \mid f(x) \notin S\}
    = \{x \in X \mid f(x) \in S^c\}
    = f^{-1}(S^c).
  \]
  Since $S$ is open, $S^c$ is closed and
  so $f^{-1}(S^c) = (f^{-1}(S))^c$ is closed by
  assumption.
  Thus $f^{-1}(S)$ is open, and so we see that
  $f$ is continuous.
\end{proof}

\begin{theorem}[Composition theorem]
  Let $(X, \mathcal{T}_X)$, $(Y, \mathcal{T}_Y)$,
  and $(Z, \mathcal{T}_Z)$ be topological spaces.
  Let
  \[f : X \to Y \quad \text{and} \quad g : Y \to Z\]
  be continuous
  functions. Then $g \circ f : X \to Z$
  is continuous.
\end{theorem}

\begin{proof}
  Let $S \subseteq Z$ be open. It suffices to
  show that $(g \circ f)^{-1}(S) \subseteq X$ is open.
  Note that
  \begin{align*}
    (g \circ f)^{-1}(S)
    = \{x \in X \mid (g \circ f)(x) \in S\}
    &= \{x \in X \mid f(x) \in g^{-1}(S)\} \\
    &= \{x \in X \mid x \in f^{-1}(g^{-1}(S))\}
    = f^{-1}(g^{-1}(S)).
  \end{align*}
  Now as $g$ is continuous, $g^{-1}(S)$ is open in $Y$.
  Finally as $f$ is continuous, $f^{-1}(g^{-1}(S))$ is
  open in $X$.
\end{proof}

\begin{theorem}
  Assume $X = \bigcup_{\alpha \in I} U_\alpha$
  for open sets $U_\alpha$ and let $f : X \to Y$.
  Assume that $f|_{U_\alpha} : U_\alpha \to Y$
  is continuous for each $\alpha \in I$.
  Then $f$ is continuous.
\end{theorem}

\begin{proof}
  Let $S \subseteq Y$ be open, and it suffices
  to show that $f^{-1}(S)$ is open. Observe that
  \[
    f^{-1}(S) = f^{-1}(S) \cap X
    = f^{-1}(S) \cap \left( \bigcup_{\alpha \in I} U_\alpha \right)
    = \bigcup_{\alpha \in I} (f^{-1}(S) \cap U_\alpha)
    = \bigcup_{\alpha \in I} f|_{U_\alpha}^{-1}(S).
  \]
  The $f|_{U_\alpha}$ are continuous,
  so each $f|_{U_\alpha}^{-1}(S)$ is open. Thus
  $f^{-1}(S)$ is open as a union of open sets.
\end{proof}

\begin{theorem}[Pasting lemma]
  Assume $X, Y$ are topological spaces and
  $A, B \subseteq X$ are open. Suppose $f_1 : A \to Y$
  and $f_2 : B \to Y$ are continuous, and that
  $f_1 \equiv f_2$ on $A \cap B$. Then
  $f : A \cup B \to Y$ defined by
  \[
    f(x) =
    \begin{cases}
      f_1(x) & \text{if } x \in A, \\
      f_2(x) & \text{if } x \in B
    \end{cases}
  \]
  is continuous.
\end{theorem}

\begin{proof}
  Let $S \subseteq Y$ be open, it suffices
  to show that $f^{-1}(S)$ is open. Observe that
  \[
    f^{-1}(S) = f_1^{-1}(S) \cup f_2^{-1}(S).
  \]
  Both $f_1^{-1}(S)$ and $f_2^{-1}(S)$ are open
  since $f_1$ and $f_2$ are continuous, so
  $f^{-1}(S)$ is open as their union.
\end{proof}

\section{Subspace Topology}

\begin{definition}
  Let $(X, \mathcal{T}_X)$ be a topological space
  and $S \subseteq X$ a set. The \emph{subspace topology}
  on $S$ is defined as follows:
  $O \subseteq S$ is open if there exists $U \subseteq X$
  open in $X$ such that $U = O \cap S$.
\end{definition}

\begin{example}
  Let $\R$ be given the metric topology and
  $S = [0, 1]$.
  \begin{itemize}
    \item The set $[0, 1]$ is not open
      in $\R$, but it is open in the subspace
      topology on $S$ since
      $[0, 1] = S \cap (-1, 2)$.
    \item The set $[0, 1 / 2)$ is neither open nor closed
      in $\R$, but
      $[0, 1 / 2) = S \cap (-1 / 2, 1 / 2)$,
      so it is open in $S$.
  \end{itemize}
\end{example}

\begin{theorem}
  The subspace topology is indeed a topology.
\end{theorem}

\begin{proof}
  Let $(X, \mathcal{T}_X)$ be a topological
  space and $S \subseteq X$ be given the subspace
  topology.

  $(i)$ We have $S = S \cap X$ and
  $\varnothing = \varnothing \cap X$, so
  $S, \varnothing$ are open in $S$.

  $(ii)$ Let $\{U_\alpha\}_{\alpha \in I}$
  be open in the subspace topology. Then
  for every $\alpha \in I$, there exists
  $O_\alpha \in \mathcal{T}$
  such that $U_\alpha = S \cap O_\alpha$. Then
  \[
    \bigcup_{\alpha \in I} U_\alpha
    = \bigcup_{\alpha \in I} (S \cap O_\alpha)
    = S \cap \left( \bigcup_{\alpha \in I} O_\alpha \right).
  \]
  The $\{O_\alpha\}_{\alpha \in I}$ are
  open in $X$, so their union is open in $X$.
  Thus $\bigcup_{\alpha \in I} U_\alpha$ is open
  in the subspace topology.

  $(iii)$ Let $\{U_i\}_{i = 1}^n $ be open in the
  subspace topology. Then there are
  $O_i$ for $1 \le i \le n$ with
  $U_i = S \cap O_i$. Then we have
  \[
    \bigcap_{i = 1}^n U_i
    = \bigcap_{i = 1}^n (S \cap O_i)
    = S \cap \left( \bigcap_{i = 1}^n O_i \right).
  \]
  As the $O_i \in \mathcal{T}$ are open,
  $\bigcap_{i = 1}^n O_i$ is open in $X$.
  Thus $\bigcap_{i = 1}^n U_i$ is open in the
  subspace topology.
\end{proof}

\begin{theorem}
  Assume $f : X \to Y$ is a continuous function and
  $S \subseteq X$ a subspace. Then
  $f|_S : S \to Y$ is continuous, where
  $S$ is equipped with the subspace topology.
\end{theorem}

\begin{proof}
  Let $O \subseteq Y$ be an open set, it suffices
  to show that $f|_S^{-1}(O)$ is open in the subspace
  topology. But observing that
  $f|_S^{-1}(O) = f^{-1}(O) \cap S$
  immediately shows that $f|_S^{-1}(O)$ is open in $S$
  since $f^{-1}(O)$ is open in $X$ due to the
  continuity of $f$.
\end{proof}

\begin{remark}
  The subspace topology is the smallest topology
  on $S$ such that the inclusion map
  $i : S \to X$ given by
  $i(s) = s$ is a continuous function.
\end{remark}

\begin{remark}
  Let $X$ be a topological space with subspaces
  $Y \subseteq X$ and $Z \subseteq Y$. Then
  the subspace topology on $Z$ induced by the
  subspace $Y$ is the same as the
  subspace topology on $Z$ induced directly by $X$.
\end{remark}

\begin{remark}
  A topological space can have a subspace
  homeomorphic to itself. For instance, consider
  $\R$ and $(-\pi / 2, \pi / 2)$ with
  a homemorphism given by the tangent function.
\end{remark}

\end{document}
