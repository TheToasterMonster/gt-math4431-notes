\documentclass[12pt, letterpaper, oneside]{book}
\usepackage[margin={0.6in, 0.75in}]{geometry}
\usepackage{microtype}
% \usepackage{kpfonts}
\usepackage{amsmath, amssymb, amsthm}
\usepackage{parskip}
\usepackage[many]{tcolorbox}
\usepackage{footnote}
\usepackage{cancel}
\usepackage{titlesec}
\usepackage{pgffor}
\usepackage[shortlabels]{enumitem}
\usepackage{hyperref}
\usepackage{tikz-cd}

\usepackage[overload]{textcase}

\renewcommand{\chaptername}{Lecture}
\newtheorem{axiom}{Axiom}[chapter]
\newtheorem{theorem}{Theorem}[chapter]
\newtheorem{prop}{Proposition}[chapter]
\newtheorem{corollary}{Corollary}[theorem]
\newtheorem{lemma}{Lemma}[chapter]
\theoremstyle{definition}
\newtheorem{definition}{Definition}[chapter]
\newtheorem{exercise}{Exercise}[chapter]
\newtheorem{example}{Example}[definition]
\newtheorem*{remark}{Remark}

\tcbset{sharp corners, breakable, enhanced, parbox=false}
\newtcolorbox{mybox}[3][]
{
  colframe = #2!150,
  colback  = #2!5,
  coltitle = #2!0!white,  
  title    = {#3},
  #1,
}

\titleformat{\chapter}[display]
    {\normalfont\huge\bfseries}{\chaptertitlename\ \thechapter}{20pt}{\Huge}
\titlespacing*{\chapter}{0pt}{0pt}{40pt}

\newcommand{\R}{\mathbb{R}}
\newcommand{\N}{\mathbb{N}}
\newcommand{\Z}{\mathbb{Z}}
\newcommand{\C}{\mathbb{C}}
\newcommand{\Q}{\mathbb{Q}}
\newcommand{\F}{\mathbb{F}}
\newcommand{\sphere}{\mathbb{S}}

\newcommand{\T}{\mathcal{T}}
\newcommand{\B}{\mathcal{B}}

\DeclareMathOperator{\Vol}{Vol}
\DeclareMathOperator{\Int}{int}
\DeclareMathOperator{\area}{area}
\DeclareMathOperator{\curl}{curl}
\DeclareMathOperator{\maps}{maps}

\title{MATH 4431: Introduction to Topology}
\author{Frank Qiang\\Instructor: Asaf Katz}
\date{Georgia Institute of Technology\\Fall 2024}

\begin{document}
  \maketitle

  \begingroup
  \let\cleardoublepage\clearpage
  \tableofcontents
  \endgroup

  % \foreach \i in {00, 01, 02, 03, 04, ..., 50} {%
  %   \edef\FileName{lectures/lecture\i.tex}%     The % here are necessary to eliminate any
  %   \IfFileExists{\FileName}{%  spurious spaces that may get inserted
  %      \input{\FileName}%       at these points
  %   }
  % }
  \chapter{Aug.~20 --- Review of Metric Spaces}

\section{Metric Spaces}
Recall the definition of a \emph{metric space}:

\begin{definition}
  Given a set $X$, a function $d : X \times X \to \R$
  is called a \emph{metric} if
  \begin{enumerate}[(i)]
    \item (strong positivity)
      $d(x, y) \ge 0$ for all
      $x, y \in X$, and $d(x, y) = 0$ if and only if
      $x = y$,
    \item (symmetry) $d(x, y) = d(y, x)$,
    \item and (triangle inequality)
      $d(x, z) \le d(x, y) + d(y, z)$
      for all $x, y, z \in X$.
  \end{enumerate}
\end{definition}

\begin{example}
  For any set $X$, we can define the
  \emph{discrete metric}
  by
  \[
    d(x, y) =
    \begin{cases}
      1 & \text{if $x \ne y$}, \\
      0 & \text{otherwise}.
    \end{cases}
  \]
  Verify as an exercise that this satisfies
  the triangle inequality.
\end{example}

\begin{example}
  The Euclidean metric in $\R^n$ is
  \[
    d(\overline{x}, \overline{y})
    = \sqrt{\sum_{i = 1}^n (x_i - y_i)^2}
  \]
  where $\overline{x} = (x_1, \ldots, x_n)$
  and $\overline{y} = (y_1, \ldots, y_n)$.
\end{example}

\section{Open Sets}
\begin{definition}
  The \emph{open ball} of radius $R > 0$ around
  $x_0 \in X$ is
  \[
    B_R(x_0) = \{y \in X \mid d(x_0, y) < R\}
  .\]
  Given a set $S \subseteq X$, a point $x_0$
  is called an interior point of $S$ if there exists
  $r > 0$ such that $B_r(x_0) \subseteq S$.
  The set $S$ is called \emph{open} if all of its
  points are interior points.
\end{definition}

\begin{prop}
  The open ball $B_R(x)$ is open.
\end{prop}

\begin{proof}
  Fix an arbitrary $y \in B_R(x)$, and observe that
  it suffices to show that $y$ is an interior point.
  Take $r = R - d(x, y)$, and
  first note that $r > 0$ since
  $d(x, y) < R$. Now note that for all $z \in B_r(y)$,
  we have
  \[
    d(x, z) \le d(x, y) + d(y, z)
    < d(x, y) + (R - d(x, y))
    = R,
  \]
  so that $z \in B_R(x)$.
  Thus $B_r(y) \subseteq B_R(x)$, and so $y$ is an interior point.
\end{proof}

\begin{corollary}
  $B_R(x) = \bigcup_{y \in B_R(x)} B_{r_y}(y)$,
  where $r_y = R - d(x, y)$.
\end{corollary}

\begin{proof}
  We have $B_{r_y}(y) \subseteq B_R(x)$
  for each $y \in B_R(x)$,\footnote{Using the argument from the previous proposition.}
  and so
  $\bigcup_{y \in B_R(x)} B_{r_y}(y) \subseteq B_R(x)$.
  For the reverse inclusion simply observe that
  $y \in B_{r_y}(y) \subseteq \bigcup_{y \in B_R(x)} B_{r_y}(y)$ for
  each $y \in B_R(x)$.
\end{proof}

\begin{prop}
  In a metric space $(X, d)$, the following
  are true:
  \begin{enumerate}[(i)]
    \item $\varnothing, X$ are open,
    \item if $\{S_i\}_{i \in I}$ are open, then
      $\bigcup_{i \in I} S_i$ is open,
    \item and if $\{S_i\}_{i = 1}^n$ are open, then
      $\bigcap_{i = 1}^n S_i$ is open.
  \end{enumerate}
\end{prop}

\begin{proof}
  $(i)$ The empty set is open vacuously.
  To see that $X$ is open, simply take $R = 1$
  for any $x \in X$.

  $(ii)$ Fix $x \in \bigcup_{i \in I} S_i$
  arbitrary, so there exists $i_0 \in I$ with
  $x \in S_{i_0}$. Since $S_{i_0}$ is open,
  $x$ is an interior point and thus there exists
  $r > 0$ such that $B_r(x) \subseteq S_{i_0}$.
  But then $B_r(x) \subseteq S_{i_0} \subseteq \bigcup_{i \in I} S_i$,
  so $x$ is an interior point of
  $\bigcup_{i \in I} S_i$ also and thus
  $\bigcup_{i \in I} S_i$ is open.

  $(iii)$ Now assume $x \in \bigcap_{i = 1}^n S_i$.
  Then for each $1 \le i \le n$, there exists
  $r_i > 0$ such that $B_{r_i}(x) \subseteq S_i$.
  Then we can choose
  \[
    r = \min\{r_1, \ldots, r_n\} > 0,
  \]
  so that $B_r(x) \subseteq B_{r_i}(x) \subseteq S_i$
  for each $1 \le i \le n$. Thus
  $B_r(x) \subseteq \bigcap_{i = 1}^n S_i$
  and $\bigcap_{i = 1}^n S_i$ is open.
\end{proof}

\begin{remark}
  The above argument for the finite intersection
  property requires that there are only finitely
  many $r_i$. Otherwise it may very well be that
  $r = \inf\{r_i\} = 0$ and the argument fails.
\end{remark}

  \chapter{Aug.~22 --- Topology, Basis, Continuity}

\section{Topological Spaces}
\begin{definition}
  A \emph{topology} $\mathcal{T} \subseteq \mathcal{P}(X)$ is a
  collection of sets such that
  \begin{enumerate}[(i)]
    \item $\varnothing, X \in \mathcal{T}$,
    \item for any index set $I$, if $\{s_i\}_{i \in I} \subseteq \mathcal{T}$,
      then $\bigcup_{i \in I} s_i \in \mathcal{T}$ (closure under arbitrary union),
    \item and if $\{s_i\}_{i = 1}^n \subseteq \mathcal{T}$, then
      $\bigcap_{i = 1}^n s_i \in \mathcal{T}$ (closure under finite intersection).
  \end{enumerate}
  A set with a topology, i.e. a pair
  $(X, \mathcal{T})$,
  is called a \emph{topological space}.
  Elements of $\mathcal{T}$ are called
  \emph{open sets}.
\end{definition}

\begin{example}
  The following are examples of topologies on a set $X$:
  \begin{itemize}
    \item The trivial topology: $\mathcal{T} = \{\varnothing, X\}$.
    \item The discrete topology: $\mathcal{T} = \mathcal{P}(X)$.\footnote{Note that the discrete topology is induced by the discrete metric.}
    \item If $(X, d)$ is a metric space, then
      $\mathcal{T} = \{\text{collection of metrically open sets}\}$
      is a topology on $X$.
  \end{itemize}
\end{example}

\begin{remark}
  Not every topology is induced by a metric.
  For instance consider the trivial topology
  on $\R$.
\end{remark}

\section{Basis for a Topology}
\begin{definition}
  A collection $\mathcal{B} \subseteq \mathcal{P}(X)$
  is called a \emph{basis} if
  \begin{enumerate}[(i)]
    \item $\bigcup_{b \in \mathcal{B}} b = X$, i.e.
      $\mathcal{B}$ is a covering of $X$,
    \item and if $x \in b_1 \cap b_2$ for any
      $b_1, b_2 \in B$, then there exists $b_3 \in \mathcal{B}$
      such that $x \in b_3$
      and $b_3 \subseteq b_1 \cap b_2$.
  \end{enumerate}
\end{definition}

\begin{theorem}
  Given a set $X$ and a basis $\mathcal{B}$, define
  \[
    \mathcal{T}_\mathcal{B} = \left\{\bigcup_{i \in I} s_i \mid \text{$I$ is any index set and $\{s_i\}_{i \in I} \subseteq \mathcal{B}$}\right\}.
  \]
  Then $\mathcal{T}_\mathcal{B}$ is a topology on $X$.
\end{theorem}

\begin{proof}
  First observe that $\varnothing, X \in \mathcal{T}_\mathcal{B}$:
  Picking $I = \varnothing$ gives
  $\bigcup_{i \in I} s_i = \varnothing \in \mathcal{T}_\mathcal{B}$
  and picking $I = \mathcal{B}$ gives
  $\bigcup_{b \in \mathcal{B}} b = X \in \mathcal{T}_\mathcal{B}$
  by the covering property of a basis.

  Now assume $\{s_i\}_{i \in I} \subseteq \mathcal{T}_\mathcal{B}$.
  For each $i \in I$, we have $s_i \in \mathcal{T}_\mathcal{B}$
  and so there exists an index set $J_i$
  such that $s_i = \bigcup_{j \in J_i} b_j$, where
  the $b_j \in \mathcal{B}$. Then
  \[
    \bigcup_{i \in I} s_i = \bigcup_{i \in I} \bigcup_{j \in J_i} b_j,
  \]
  which is a union of elements of
  $\mathcal{B}$ and hence is in
  $\mathcal{T}_{\mathcal{B}}$.

  Finally assume $\{s_i\}_{i = 1}^n \subseteq \mathcal{T}_\mathcal{B}$.
  Now as each $s_i \in \mathcal{T}_\mathcal{B}$,
  there exists $J_i$ such that
  $s_i = \bigcup_{j \in J_i} b_j$.
  Then
  \[
    \bigcap_{i = 1}^n s_i = \bigcap_{i = 1}^n \bigcup_{j \in J_i} b_j.
  \]
  Now assume $x \in \bigcap_{i = 1}^n s_i = \bigcap_{i = 1}^n \bigcup_{j \in J_i} b_j$.
  For each $1 \le i \le n$, there exists
  $j_i \in J_i$ such that $x \in b_{j_i}$.
  Hence $x \in \bigcap_{i = 1}^n b_{j_i}$.
  Now by induction on the intersection property of a basis,
  we can find $b_x \in \B$ with
  \[
    x \in b_x \subseteq \bigcap_{i = 1}^n b_{j_i}
  \]
  Also observe that
  \[\bigcap_{i = 1}^n b_{j_i} \subseteq \bigcap_{i = 1}^n \bigcup_{j \in J_i} b_j
    = \bigcap_{i = 1}^n s_i
  \]
  by construction,
  so we may write
  \[
    \bigcap_{i = 1}^n s_i = \bigcup_{x \in \bigcap_{i = 1}^n s_i} b_x \in \mathcal{T}_\mathcal{B}
  \]
  as a union of elements of $\mathcal{B}$.
\end{proof}

\begin{definition}
  A \emph{subbasis} $\mathcal{B} \subseteq \mathcal{P}(X)$ is a collection
  of sets such that
  $\bigcup_{b \in \mathcal{B}} b = X$.
\end{definition}

\begin{remark}
  One may define a basis $\widetilde{\mathcal{B}}$ from a
  subbasis $\mathcal{B}$ by adding all finite intersections
  of elements of $\mathcal{B}$.
  We get the covering property for free and adding
  the finite intersections gives us the
  intersection property of a basis.
\end{remark}

\begin{example}
  For $\R$ with the Euclidean metric, the following
  are bases for the standard topology:
  \begin{itemize}
    \item $\{B_R(x) \mid x \in \R, R > 0\}$.
    \item $\{B_R(x) \mid x \in \R, R > 0, R \in \Q\}$.
      For this use the fact that $\Q$ is dense in $\R$.
  \end{itemize}
  In particular this shows that a basis for a topology
  is not unique in general.
\end{example}

\section{Continuous Functions}

\begin{definition}
  Let $(X, \mathcal{T}_X)$ and $(Y, \mathcal{T}_Y)$
  be two topological spaces. A function $f : X \to Y$
  is called \emph{continuous} if for any
  $O \in \mathcal{T}_Y$, we have
  $f^{-1}(O) \in \mathcal{T}_X$, i.e. the
  preimage of an open set is open.\footnote{Recall that $f^{-1}(O) = \{x \in X \mid f(x) \in O\}$.}
\end{definition}

\begin{example}
  Let $X$ be equipped with the trivial
  topology $\{\varnothing, X\}$ and let
  $\R$ be equipped with the standard topology.
  Then the only continuous functions $f : X \to \R$
  are the constant functions $f : x \mapsto c$
  for fixed $c \in \R$. To see this, observe that
  \begin{itemize}
    \item $x \mapsto c$ is continuous since
      any open set in $\R$ either contains $c$ or does not,
      and so the preimage is either $X$ or $\varnothing$.
    \item Suppose $f(x_1) = y_1$ and $f(x_2) = y_2$.
      Let $\epsilon = |y_1 - y_2|$ and observe that
      $x_1 \in f^{-1}(B_\epsilon(y_1))$ while
      $x_2 \notin f^{-1}(B_\epsilon(y_1))$, so
      $f^{-1}(B_\epsilon(y_1))$ is not open
      in $X$ despite $B_\epsilon(y_1)$ being open in $\R$.
  \end{itemize}
\end{example}

\begin{example}
  Let $X$ have the discrete topology
  $\mathcal{T} = \mathcal{P}(X)$ and let
  $\R$ have the standard topology.
  Then all functions $X \to \R$ are continuous
  since any preimage is a subset of $X$ and thus
  in $\mathcal{P}(X)$.
\end{example}

\begin{remark}
  In a way, the trivial topology has too few
  open sets while the discrete topology has too many.
\end{remark}

\begin{definition}
  Two topological spaces $(X, \mathcal{T}_X)$ and
  $(Y, \mathcal{T}_Y)$ are
  \emph{topologically equivalent} or
  \emph{homeomorphic} if there exists a
  bijection $f : X \to Y$ such that
  $f$ and $f^{-1}$ are continuous.
\end{definition}

\begin{remark}
  A bijective function $f$ being continuous does
  not necessarily imply that its inverse $f^{-1}$ is.
\end{remark}

\begin{example}
  Consider $(-\pi / 2, \pi / 2)$ equipped with the
  Euclidean metric. This is homeomorphic to $\R$
  equipped with the Euclidean metric.\footnote{Note that $(-\pi / 2, \pi / 2)$ is bounded while $\R$ is not.} One
  homeomorphism is given by $\tan : (-\pi / 2, \pi / 2) \to \R$.
\end{example}

  \chapter{Aug.~27 --- Closed Sets, Continuity, the Subspace Topology}

\section{Closed Sets}
\begin{definition}
  A set $S \subseteq X$ is called a \emph{closed set}
  if $S^c = X \setminus S$ is open.
\end{definition}

\begin{example}
  In $\R$, observe that
  $[a, b]^c = (-\infty, a) \cup (b, \infty)$,
  which is a union of open sets and thus open.
  Thus the closed intervals $[a, b] \subseteq \R$ are closed.
\end{example}

\begin{remark}
  This is not a dichotomy. Sets can be
  both open and closed (\emph{clopen}), or even neither.
  Trivially, if $X$ is any topological space,
  then $\varnothing$ and $X$ are both open and closed.
\end{remark}

\begin{example}
  Let $X = \{0, 1\}$ and $\mathcal{T} = \mathcal{P}(X)$.
  Then $\{0\}$ is both open and closed.
\end{example}

\begin{example}
  Let $X = \{1, 2, 3\}$ and
  $\mathcal{T} = \{\varnothing, X, \{1\}, \{1, 2\}\}$.
  Then $\{2\}$ is neither open nor closed.
\end{example}

Recall the following De Morgan's laws from set theory:
\begin{prop}[De Morgan's laws]
  Let $I$ be an index set and $\{A_i\}_{i \in I}$
  be sets. Then
  \[\left( \bigcup_{i \in I} A_i \right)^c = \bigcap_{i \in I} A_i^c
    \quad \text{and} \quad
    \left( \bigcap_{i \in I} A_i \right)^c = \bigcup_{i \in I} A_i^c.
  \]
\end{prop}

\begin{corollary}
  In a topological space $(X, \mathcal{T})$, we have:
  \begin{enumerate}[(i)]
    \item $\varnothing, X$ are closed.
    \item if $\{A_i\}_{i \in I}$ are closed,
      then $\bigcap_{i \in I} A_i$ is closed,
    \item and if $\{A_i\}_{i = 1}^n$ are closed,
      then so is $\bigcup_{i = 1}^n A_i$.
  \end{enumerate}
  This gives a dual characterization of a topology.
\end{corollary}

\begin{proof}
  $(i)$ We have $\varnothing^c = X \in \mathcal{T}$
  and $X^c = \varnothing \in \mathcal{T}$.

  $(ii)$ Note that
  \[
    \left( \bigcap_{i \in I} A_i \right)^c
    = \bigcup_{i \in I} A_i^c.
  \]
  As each $A_i$ is closed, we have
  $A_i^c \in \mathcal{T}$ is open, and hence
  $\bigcup_{i \in I} A_i^c \in \mathcal{T}$ is open.
  So $\bigcap_{i \in I} A_i$ is closed.

  $(iii)$ Observe that
  \[
    \left( \bigcup_{i = 1}^n A_i \right)^c
    = \bigcap_{i = 1}^n A_i^c.
  \]
  Each $A_i$ is closed, so $A_i^c$ is open.
  Thus $\bigcap_{i = 1}^n A_i^c$ is open,
  and so $\bigcup_{i = 1}^n A_i$ is closed.
\end{proof}

\section{Properties of Continuity}
Recall that a function $f : (X, \mathcal{T}_X) \to (Y, \mathcal{T}_Y)$
is continuous if for every
$O \in \mathcal{T}_Y$, we have $f^{-1}(O) \in \mathcal{T}_X$.

\begin{theorem}
  A function $f : X \to Y$ is continuous if and
  only if for every $C$ closed in $Y$,
  $f^{-1}(C)$ is closed in $X$.
\end{theorem}

\begin{proof}
  $(\Rightarrow)$ Let $C \subseteq Y$ be closed.
  Note that
  \[
    f^{-1}(C) = \{x \in X \mid f(x) \in C\},
  \]
  so we have
  \[
    (f^{-1}(C))^c = \{x \in X \mid f(x) \notin C\}
    = \{x \in X \mid f(x) \in C^c\}
    = f^{-1}(C^c)
  .\]
  Since $C$ is closed, $C^c$ is open and so
  $f^{-1}(C^c) = (f^{-1}(C))^c$ is open. Thus $f^{-1}(C)$ is closed.

  $(\Leftarrow)$ Assume $S \subseteq Y$ is open.
  Note that
  \[
    (f^{-1}(S))^c
    = \{x \in X \mid f(x) \in S\}^c
    = \{x \in X \mid f(x) \notin S\}
    = \{x \in X \mid f(x) \in S^c\}
    = f^{-1}(S^c).
  \]
  Since $S$ is open, $S^c$ is closed and
  so $f^{-1}(S^c) = (f^{-1}(S))^c$ is closed by
  assumption.
  Thus $f^{-1}(S)$ is open, and so we see that
  $f$ is continuous.
\end{proof}

\begin{theorem}[Composition theorem]
  Let $(X, \mathcal{T}_X)$, $(Y, \mathcal{T}_Y)$,
  and $(Z, \mathcal{T}_Z)$ be topological spaces.
  Let
  \[f : X \to Y \quad \text{and} \quad g : Y \to Z\]
  be continuous
  functions. Then $g \circ f : X \to Z$
  is continuous.
\end{theorem}

\begin{proof}
  Let $S \subseteq Z$ be open. It suffices to
  show that $(g \circ f)^{-1}(S) \subseteq X$ is open.
  Note that
  \begin{align*}
    (g \circ f)^{-1}(S)
    = \{x \in X \mid (g \circ f)(x) \in S\}
    &= \{x \in X \mid f(x) \in g^{-1}(S)\} \\
    &= \{x \in X \mid x \in f^{-1}(g^{-1}(S))\}
    = f^{-1}(g^{-1}(S)).
  \end{align*}
  Now as $g$ is continuous, $g^{-1}(S)$ is open in $Y$.
  Finally as $f$ is continuous, $f^{-1}(g^{-1}(S))$ is
  open in $X$.
\end{proof}

\begin{theorem}
  Assume $X = \bigcup_{\alpha \in I} U_\alpha$
  for open sets $U_\alpha$ and let $f : X \to Y$.
  Assume that $f|_{U_\alpha} : U_\alpha \to Y$
  is continuous for each $\alpha \in I$.
  Then $f$ is continuous.
\end{theorem}

\begin{proof}
  Let $S \subseteq Y$ be open, and it suffices
  to show that $f^{-1}(S)$ is open. Observe that
  \[
    f^{-1}(S) = f^{-1}(S) \cap X
    = f^{-1}(S) \cap \left( \bigcup_{\alpha \in I} U_\alpha \right)
    = \bigcup_{\alpha \in I} (f^{-1}(S) \cap U_\alpha)
    = \bigcup_{\alpha \in I} f|_{U_\alpha}^{-1}(S).
  \]
  The $f|_{U_\alpha}$ are continuous,
  so each $f|_{U_\alpha}^{-1}(S)$ is open. Thus
  $f^{-1}(S)$ is open as a union of open sets.
\end{proof}

\begin{theorem}[Pasting lemma]
  Assume $X, Y$ are topological spaces and
  $A, B \subseteq X$ are open. Suppose $f_1 : A \to Y$
  and $f_2 : B \to Y$ are continuous, and that
  $f_1 \equiv f_2$ on $A \cap B$. Then
  $f : A \cup B \to Y$ defined by
  \[
    f(x) =
    \begin{cases}
      f_1(x) & \text{if } x \in A, \\
      f_2(x) & \text{if } x \in B
    \end{cases}
  \]
  is continuous.
\end{theorem}

\begin{proof}
  Let $S \subseteq Y$ be open, it suffices
  to show that $f^{-1}(S)$ is open. Observe that
  \[
    f^{-1}(S) = f_1^{-1}(S) \cup f_2^{-1}(S).
  \]
  Both $f_1^{-1}(S)$ and $f_2^{-1}(S)$ are open
  since $f_1$ and $f_2$ are continuous, so
  $f^{-1}(S)$ is open as their union.
\end{proof}

\section{Subspace Topology}

\begin{definition}
  Let $(X, \mathcal{T}_X)$ be a topological space
  and $S \subseteq X$ a set. The \emph{subspace topology}
  on $S$ is defined as follows:
  $O \subseteq S$ is open if there exists $U \subseteq X$
  open in $X$ such that $U = O \cap S$.
\end{definition}

\begin{example}
  Let $\R$ be given the metric topology and
  $S = [0, 1]$.
  \begin{itemize}
    \item The set $[0, 1]$ is not open
      in $\R$, but it is open in the subspace
      topology on $S$ since
      $[0, 1] = S \cap (-1, 2)$.
    \item The set $[0, 1 / 2)$ is neither open nor closed
      in $\R$, but
      $[0, 1 / 2) = S \cap (-1 / 2, 1 / 2)$,
      so it is open in $S$.
  \end{itemize}
\end{example}

\begin{theorem}
  The subspace topology is indeed a topology.
\end{theorem}

\begin{proof}
  Let $(X, \mathcal{T}_X)$ be a topological
  space and $S \subseteq X$ be given the subspace
  topology.

  $(i)$ We have $S = S \cap X$ and
  $\varnothing = \varnothing \cap X$, so
  $S, \varnothing$ are open in $S$.

  $(ii)$ Let $\{U_\alpha\}_{\alpha \in I}$
  be open in the subspace topology. Then
  for every $\alpha \in I$, there exists
  $O_\alpha \in \mathcal{T}$
  such that $U_\alpha = S \cap O_\alpha$. Then
  \[
    \bigcup_{\alpha \in I} U_\alpha
    = \bigcup_{\alpha \in I} (S \cap O_\alpha)
    = S \cap \left( \bigcup_{\alpha \in I} O_\alpha \right).
  \]
  The $\{O_\alpha\}_{\alpha \in I}$ are
  open in $X$, so their union is open in $X$.
  Thus $\bigcup_{\alpha \in I} U_\alpha$ is open
  in the subspace topology.

  $(iii)$ Let $\{U_i\}_{i = 1}^n $ be open in the
  subspace topology. Then there are
  $O_i$ for $1 \le i \le n$ with
  $U_i = S \cap O_i$. Then we have
  \[
    \bigcap_{i = 1}^n U_i
    = \bigcap_{i = 1}^n (S \cap O_i)
    = S \cap \left( \bigcap_{i = 1}^n O_i \right).
  \]
  As the $O_i \in \mathcal{T}$ are open,
  $\bigcap_{i = 1}^n O_i$ is open in $X$.
  Thus $\bigcap_{i = 1}^n U_i$ is open in the
  subspace topology.
\end{proof}

\begin{theorem}
  Assume $f : X \to Y$ is a continuous function and
  $S \subseteq X$ a subspace. Then
  $f|_S : S \to Y$ is continuous, where
  $S$ is equipped with the subspace topology.
\end{theorem}

\begin{proof}
  Let $O \subseteq Y$ be an open set, it suffices
  to show that $f|_S^{-1}(O)$ is open in the subspace
  topology. But observing that
  $f|_S^{-1}(O) = f^{-1}(O) \cap S$
  immediately shows that $f|_S^{-1}(O)$ is open in $S$
  since $f^{-1}(O)$ is open in $X$ due to the
  continuity of $f$.
\end{proof}

\begin{remark}
  The subspace topology is the smallest topology
  on $S$ such that the inclusion map
  $i : S \to X$ given by
  $i(s) = s$ is a continuous function.
\end{remark}

\begin{remark}
  Let $X$ be a topological space with subspaces
  $Y \subseteq X$ and $Z \subseteq Y$. Then
  the subspace topology on $Z$ induced by the
  subspace $Y$ is the same as the
  subspace topology on $Z$ induced directly by $X$.
\end{remark}

\begin{remark}
  A topological space can have a subspace
  homeomorphic to itself. For instance, consider
  $\R$ and $(-\pi / 2, \pi / 2)$ with
  a homemorphism given by the tangent function.
\end{remark}

  \chapter{Aug.~29 --- Connectedness}

\section{Connected Spaces}
\begin{definition}
  A \emph{separation} of a topological space $X$
  is two open, nonempty sets $U, V \subseteq X$ such
  that $X = U \cup V$ and $U \cap V = \varnothing$.
  A space is called \emph{connected}
  if there is no separation of the space.
\end{definition}

\begin{prop}
  If $X$ is separated, i.e.
  $X = U \cup V$ with $U, V$ open and disjoint,
  then $U$ and $V$ are both open and closed.
\end{prop}

\begin{proof}
  Observe that $U$ is open by assumption, and we have
  \[
    U^c = X \setminus U = V,
  \]
  which is also open by assumption. Hence
  $U$ is closed. The case for $V$ is identical.
\end{proof}

\begin{example}
  Consider the following:
  \begin{itemize}
    \item The singleton space $\{x\}$ is connected. There
      are no two nonempty, disjoint open sets.
    \item Consider the space $X = \{0, 1\}$. This
      case depends on the choice of topology:
      \begin{enumerate}
        \item With the trivial topology
          $\mathcal{T} = \{\varnothing, X\}$, the
          space is connected.
        \item With the discrete topology
          $\mathcal{T} = \{\varnothing, X, \{1\}, \{0\}\}$,
          $X$ is disconnected since
          $X = \{0\} \cup \{1\}$.
        \item With the topology
          $\mathcal{T} = \{\varnothing, X, \{1\}\}$,
          the space is connected. The only
          nonempty sets $\{1\}, X$ are not disjoint
          and thus there can be no separation.
      \end{enumerate}
  \end{itemize}
\end{example}

\begin{theorem}
  A space $X$ is disconnected if and only if
  there exists a surjective map
  $f : X \to \{0, 1\}$ with the discrete topology.
\end{theorem}

\begin{proof}
  $(\Rightarrow)$
  If $X$ is disconnected, then we may write
  $X = U \cup V$ with $U, V$ open, disjoint, and
  nonempty. Then define
  \[
    f(x) = \begin{cases}
      0 & x \in U, \\
      1 & x \in V,
    \end{cases}
  \]
  which is surjective as $U, V$ are nonempty. To
  see that $f$ is continuous, observe that
  \[
    f^{-1}(\varnothing) = \varnothing, \quad
    f^{-1}(\{0, 1\}) = X, \quad
    f^{-1}(\{0\}) = U, \quad
    f^{-1}(\{1\}) = V,
  \]
  each of which are open. These are all of
  the open sets in the discrete topology, so $f$ is
  continuous.

  $(\Leftarrow)$ Assume there exists a surjective
  and continuous map $f : X \to \{0, 1\}$. Define
  \[
    U = f^{-1}(\{0\}) \quad \text{and} \quad
    V = f^{-1}(\{1\}),
  \]
  which are open since $f$ is continuous. Observe
  that $U, V \ne \varnothing$ since $f$ is surjective.
  Also $U \cap V = \varnothing$ since if there is
  any $x \in U \cap V$, then
  $f(x) = 0$ as $x \in U$ and $f(x) = 1$ as $x \in V$,
  a contradiction.
  Finally, $X = U \cup V$ since $f(x) = 0$ or
  $f(x) = 1$ for every $x \in X$, i.e.
  $x \in U$ or $x \in V$. So $X$ is disconnected.
\end{proof}

\section{Connected Sets}
\begin{definition}
  Let $X$ be a topological space and $S \subseteq X$.
  Then $S$ is called
  \emph{connected} if it is connected in the
  subspace topology.
\end{definition}

\begin{theorem}
  \label{thm:union-connected}
  If $A, B$ are connected sets and
  $A \cap B \ne \varnothing$, then $A \cup B$
  is connected.
\end{theorem}

\begin{proof}
  Assume not. Then there exists a continuous,
  surjective map
  $f : A \cup B \to \{0, 1\}$ with the discrete
  topology. Consider $f|_A : A \to \{0, 1\}$,
  which is continuous in the subspace topology.
  Notice that $f(A)$ cannot be $\{0, 1\}$ since
  otherwise $A$ is disconnected. Without
  loss of generality, assume $f(A) = \{0\}$ since
  $A$ is nonempty. Now consider
  $f|_B : B \to \{0, 1\}$, which is also
  continuous. Similarly, notice that
  $f(B)$ cannot be $\{0, 1\}$. But
  there exists
  $p \in A \cap B$, and $f(p) = 0$ as $p \in A$.
  Then since $p \in B$, we must have $f(B) = \{0\}$.
  But then we get that $f(A \cup B) = \{0\} \ne \{0, 1\}$,
  a contradiction to surjectivity.
\end{proof}

\begin{corollary}
  \label{cor:union-connected}
  A union of connected sets with
  ``common points'' is connected.
\end{corollary}

\begin{proof}
  Run induction (transfinite if the union is infinite)
  using the previous theorem.
\end{proof}

\begin{theorem}
  Closed intervals in $[a, b] \subseteq \R$
  with the metric topology are connected.
\end{theorem}

\begin{proof}
  Assume otherwise that $[a, b] = U \cup V$
  with $U, V$ disjoint, open, and nonempty.
  Assume without loss of generality that
  $a \in U$. Since $V$ is nonempty, there exists
  $c > a$ such that $c \in V$. Now consider
  $[a, c] \subseteq [a, b]$ with
  $U_1 = U \cap [a, c]$ and $V_1 = V \cap [a, c]$.
  By the least upper bound property of $\R$,
  since $U_1$ is nonempty and bounded from above,
  there exists $s = \sup U_1$ with
  $s \le c$. Now either $s \in U_1$ or $s \notin U_1$.

  If $s \in U_1$ (note this implies $s \ne c$), then $s$ is an interior point of
  $U_1$ since $U_1$ is open. So one may find
  a point $t$ such that $t > s$ and $t \in U_1$. But
  then $s$ is no longer an upper bound of $U_1$,
  a contradiction.

  Otherwise $s \notin U_1$. Since $U_1, V_1$
  cover $[a, c]$, we must then have $s \in V_1$ (note this implies $s \ne a$.
  Since $V_1$ is open, $s$ is an interior point
  of $V_1$, and thus there exists $t < s$
  such that $t \in V_1$ and $t$ is an upper bound for
  $U_1$. This contradicts $s$ being the
  least upper bound of $U_1$.

  Since both cases lead to contradictions, we
  conclue that $[a, b]$ must be connected.
\end{proof}

\begin{corollary}
  Open intervals in $\R$ are connected, and
  $\R$ itself is connected.
\end{corollary}

\begin{proof}
  For some $N_0 \ge 1$ (for instance choose
  $N_0 \ge 2 / (b - a)$) we can write
  \[
    (a, b) = \bigcup_{n = N_0}^\infty \left[a + \frac{1}{n}, b - \frac{1}{n}\right],
  \]
  Each of these closed intervals is connected by
  the previous theorem, and thus the union is connected
  by Corollary \ref{cor:union-connected} since
  they overlap.
  Similarly writing $\R = \bigcup_{n = 1}^\infty [-n, n]$
  shows that $\R$ is connected.
\end{proof}

\begin{corollary}[Intermediate value theorem]
  Let $f : [a, b] \to \R$ be a continuous function.
  Then
  for any $f(a) < t < f(b)$, there exists
  $c \in [a, b]$ such that $f(c) = t$.
\end{corollary}

\begin{proof}
  Assume not. We can consider the open sets
  $(-\infty, t)$ and $(t, \infty)$ in $\R$. Then
  $f^{-1}((-\infty, t))$ and $f^{-1}((t, \infty))$
  are open sets since $f$ is continuous. They
  are clearly disjoint (since $f$ must be
  well-defined),
  and also nonempty since
  $a \in f^{-1}((-\infty, t))$ and
  $b \in f^{-1}((t, \infty))$. Also since
  $f^{-1}(\{t\}) = \varnothing$ by assumption,
  \[
    [a, b] = f^{-1}((-\infty, t)) \cup f^{-1}((t, \infty)).
  \]
  But this is a separation of $[a, b]$, a
  contradiction since $[a, b]$ is connected.
\end{proof}

\begin{prop}
  The open interval $(0, 1)$ is not
  homeomorphic to the closed interval $[0, 1]$.
\end{prop}

\begin{proof}
  Removing any point from $(0, 1)$ disconnects
  it, but $[0, 1) = [0, 1] \setminus \{1\}$
  remains connected.\footnote{To see that $[0, 1)$ is connected, we can write $[0, 1) = \bigcup_{n = 2}^\infty [0, 1 - 1 / n]$.}
\end{proof}

\begin{prop}
  The real line $\R$ is not homeomorphic
  to the plane $\R^n$ for any $n \ge 2$.
\end{prop}

\begin{proof}
  Removing a point from $\R$ disconnects it but
  the same is not true for $\R^n$ when $n \ge 2$.
\end{proof}

  \chapter{Sept.~3 --- Path-Connectedness}

\section{More on Connectedness}
\begin{remark}
  The intervals $[a, b] \subseteq \R$ are homeomorphic
  to $[0, 1]$ for any $a < b$. We can take
  $f : [a, b] \to [0, 1]$ defined by
  \[
    f(x) = \frac{1}{b - a}(x - a)
  \]
  for instance as a homemorphism.
\end{remark}

\begin{lemma}
  \label{lem:image-connected}
  If $X$ is connected and $f : X \to Y$
  is continuous, then $f(X)$ is connected.
\end{lemma}

\begin{proof}
  This is part of Homework 2.
\end{proof}

\begin{corollary}
  The plane $\R^2$ is connected.
\end{corollary}

\begin{proof}
  Express $\R^2$ as the union of
  horizontal and vertical lines. Each line is the
  image of $\R$ and is thus connected by
  Lemma $\ref{lem:image-connected}$.
  Also any pair of
  horizontal and vertical lines must intersect,
  so we can use Corollary \ref{cor:union-connected}
  to conclude that the union $\R^2$ is connected.
\end{proof}

\begin{remark}
  We can extend this to $\R^3$ by embedding
  planes (copies of $\R^2$), and similarly for $\R^n$.
\end{remark}

\begin{prop}
  The unit circle $\mathbb{S}^1 \subseteq \R^2$
  is connected.
\end{prop}

\begin{proof}
  Define $\gamma : [0, 2\pi] \to \R^2$
  by $\gamma(t) = (\cos t, \sin t)$.
  The image of $\gamma$ is precisely $\mathbb{S}^1$.
\end{proof}

\begin{prop}
  \label{prop:equivalence-connected}
  Define a relation $\sim$ on $X$ by $x \sim y$
  if there exists a connected subset
  $S \subseteq X$ such that $x, y \in S$.
  Then $\sim$ is an equivalence relation.
\end{prop}

\begin{proof}
  For reflexivity, fix $x \in X$ and let
  $S$ be the largest connected set containing
  $x$ (this exists since we know at least
  $\{x\}$ must be
  connected). Then $x \in S$, so $x \sim x$.

  For symmetry, fix $x, y \in X$. If $x \sim y$,
  then there exists a connected set $S$ such that
  $x, y \in S$. But then $y, x \in S$, so we see that
  $y \sim x$.

  For transitivity, assume that $x \sim y$
  and $y \sim z$. Then there exists $S_1$
  connected such that $x, y \in S_1$ and $S_2$
  connected such that $y, z \in S_2$. Notice
  that $S_1 \cap S_2 \ne \varnothing$ since
  $y \in S_1 \cap S_2$. Then $S_1 \cup S_2$ is
  connected by Theorem \ref{thm:union-connected}
  and $x, y, z \in S_1 \cap S_2$. In particular,
  $x, z \in S_1 \cap S_2$ and thus $x \sim z$.

  So we see that $\sim$ is an equivalence relation.
\end{proof}

\begin{definition}
  Let the equivalence relation $\sim$ be defined
  on $X$ as in Proposition \ref{prop:equivalence-connected}.
  Then we can write $X$ as the disjoint union of
  the equivalence classes of $\sim$.
  These
  equivalence classes are called
  the \emph{connected components} of $X$.
\end{definition}

\begin{remark}
  The connected components of a space are defined
  solely
  via topologies, so they must be invariant under
  homeomorphism.
\end{remark}

\begin{example}
  The letter $S$, sitting in $\R^2$, is not
  homeomorphic to the letter $T$. There is a point
  we can remove from $T$ to give three connected
  components, but removing any point from $S$ gives
  at most two such connected components.
\end{example}

\section{Path-Connectedness}

\begin{remark}
  Connectedness is usually a very difficult
  property to verify. This motivates
  \emph{path-connectedness}.
\end{remark}

\begin{definition}
  A set $S$ is \emph{path-connected} if
  for all $x, y \in S$, there exists a continuous
  map $\gamma : [0, 1] \to S$ such that
  $\gamma(0) = x$ and $\gamma(1) = y$. Here
  $[0, 1]$ is given the usual metric topology.
\end{definition}

\begin{lemma}
  If $S$ is path-connected, then $S$ is connected.
\end{lemma}

\begin{proof}
  This is part of Homework 2.
\end{proof}

\begin{remark}
  Unlike connectedness, it is immediately
  obvious that $\R^n$ is path-connected. Simply
  take the line segment between any two points.
  Then we can conclude connectedness by the previous
  lemma.
\end{remark}

\begin{example}
  There are spaces which are connected but not
  path-connected.
  \begin{itemize}
    \item Consider the
      \emph{topologist's sine curve}, given by
      the union of the vertical segment
      $\{(0, y) \mid -1 \le y \le 1\}$ and the image of
      $(0, \infty)$ under $x \mapsto (x, \sin(1/x))$, is
      an example of such a space. See Homework 2
      for more details.
    \item Consider the \emph{cone} $C$ in
      $\R^2$ defined by ($(0, 1)$ denotes an
      open interval unless otherwise specified)
      \[
        C = ([0, 1] \times \{0\})
        \cup (K \times [0, 1])
        \cup (\{0\} \times [0, 1]),
      \]
      where $K = \{1 / n : n \in \N\}$.
      Note that $C$ is clearly path-connected
      and hence also connected.
      Then define the space
      \[
        D = C \setminus (\{0\} \times (0, 1)),
      \]
      which is now not path-connected (consider
      the point $(0, 1) \in D$)
      but still connected.
  \end{itemize}
\end{example}

\begin{remark}
  Observe the following:
  \begin{itemize}
    \item One can define \emph{path-connected components} in a similar
      manner as connected components.
    \item A continuous image of a path-connected
      space is path-connected. Simply compose
      the curve with the continuous map, which is
      now a path in the image.
    \item The union of path-connected spaces
      sharing a point is path-connected.
      Take two curves to the common point
      and concatenate them using the pasting lemma.
    \item In $\R^n$, connectedness is equivalent
      to path-connectedness. In general, this
      holds if you can get a basis of only connected
      sets.
  \end{itemize}
\end{remark}

\begin{remark}
  Recall from homework that if $f : [0, 1] \to [0, 1]$
  is continuous, then $f$ has a fixed point,
  i.e. there exists $c \in [0, 1]$ with $f(c) = c$.
  This follows from a clever use of the intermediate
  value theorem. Now consider a more topological
  perspective. Consider the diagonal
  $\{(x, x) \mid x \in [0, 1]\}$ and look at the
  graph of $f$, which is contained in the closed
  unit square. This graph is path-connected as the
  image of a path-connected set and so
  there is a path between the points $(0, f(0))$
  and $(1, f(1))$. But then this path must
  intersect the diagonal at some point, which
  gives a fixed point.
\end{remark}

\begin{theorem}(Brouwer fixed point theorem)
  Let $K$ be a closed, bounded, and convex
  set in $\R$. Then any continuous map $f : K \to K$
  has a fixed point, i.e. there exists $c \in K$
  such that $f(c) = c$.
\end{theorem}

\begin{remark}
  One can see the existence of the Nash equilibrium
  as a consequence of this theorem.
\end{remark}

\begin{remark}
  In $\R^2$, this theorem follows from the
  following claim. Let $X = \maps(\sphere^1, \sphere^1)$ be the set of all
  continuous maps from $\mathbb{S}^1$ to itself.
  Then Brouwer's fixed point theorem in $\R^2$
  follows from the following:
\end{remark}

\begin{theorem}
  The space $\maps(\sphere^1, \sphere^1)$ is not
  path connected.
\end{theorem}

  \chapter{Sept.~5 --- Compactness}

\section{Note on the Subspace Topology}
\begin{remark}
  Let $X$ be a topological space with topology
  $\T_X$, and let $Y \subseteq X$ be a subset
  endowed with a topology $\tau$. Suppose that
  for any continuous $f : X \to Z$, there exists
  a continuous $\widetilde{f} : Y \to Z$
  such that the following diagram commutes,
  \[
    \begin{tikzcd}
      X \ar[r, "f"] & Z \\
      Y \ar[ur, "\widetilde{f}"', dashed] \ar[u, "i"] &
    \end{tikzcd}
  \]
  where $i : Y \to X$ is the inclusion map.\footnote{Note that at least set-theoretically, this immediately defines $\widetilde{f} = f|_Y$. But a priori we do not know that $\widetilde{f}$ is continuous.}
  Then in Homework 2 we showed that
  $\T_Y \subseteq \tau$. We can see this as a
  \emph{universal property} for the subspace topology.
\end{remark}

\section{Compactness}
\begin{definition}
  A set $C \subseteq X$ is called \emph{compact}
  if for any \emph{open cover}
  \[
    C \subseteq \bigcup_{\alpha \in I} U_\alpha,
    \quad
    \text{each $U_\alpha$ is open},
  \]
  there exists a finite subcover $C \subseteq \bigcup_{i = 1}^n U_{\alpha_i}$.
\end{definition}

\begin{example} Consider the following:
  \begin{itemize}
    \item In a finite topology, any set is compact.
      This is because any open cover is already finite.
    \item In a discrete space, i.e. $\mathcal{T} = \mathcal{P}(X)$,
      compact sets are precisely the finite sets.
      It is clear that finite sets are compact,
      for each $x$ choose a single open set in the cover
      containing $x$. Conversely, if a set is
      compact, we can pick our open
      cover to contain only singletons, and the
      existence of a
      finite subcover means that the set has
      only finitely many elements.
  \end{itemize}
\end{example}

\begin{theorem}[Heine-Borel]
  Let $C \subseteq \R^n$ be a subset, where
  $\R^n$ is given the metric topology. Then
  $C$ is compact if and only if $C$ is closed and bounded.
\end{theorem}

\begin{proof}
  We postpone this proof until later.
\end{proof}

\begin{lemma}
  Let $X$ be compact. If $Y \subseteq X$ is closed,
  then $Y$ is compact.
\end{lemma}

\begin{proof}
  Let $Y \subseteq X$ closed be given, and assume that
  $Y \subseteq \bigcup_{\alpha \in I} U_\alpha$
  an open cover. Since $Y$ is closed,
  its complement $Y^c$ is open. Then
  \[
    Y^c \cup \bigcup_{\alpha \in I} U_\alpha
  \]
  is an open cover of $X$ since $X = Y \cup Y^c$.
  Since $X$ is compact, there exists a finite subcover
  \[
    X \subseteq Y^c \cup \bigcup_{i = 1}^n U_{\alpha_i}.
  \]
  Now observe that $Y \subseteq X$ and
  $Y \cap Y^c = \varnothing$,
  so actually $Y \subseteq \bigcup_{i = 1}^n U_{\alpha_i}$,
  which is a finite subcover.
\end{proof}

\begin{theorem}
  Let $X$ be compact and $f : X \to Y$
  continuous. Then $f(X)$ is compact.
\end{theorem}

\begin{proof}
  Consider $f(X) \subseteq Y$ and let
  $f(X) \subseteq \bigcup_{\alpha \in I} V_\alpha$, an
  open cover in $Y$. Notice that
  \[
    X = f^{-1}(f(X))
    \subseteq f^{-1}\left(\bigcup_{\alpha \in I} V_\alpha\right)
    = \bigcup_{\alpha \in I} f^{-1}(V_\alpha).
  \]
  Note that each $f^{-1}(V_\alpha)$ is open in $X$ since $f$ is
  continuous and $V_\alpha$ is open in $Y$, so
  this is in fact an open cover of $X$. Thus since $X$ is
  compact, we may extract a finite subcover
  \[
    X \subseteq \bigcup_{i = 1}^n f^{-1}(V_{\alpha_i}).
  \]
  Then we see that
  \[
    f(X) \subseteq f\left(\bigcup_{i = 1}^n f^{-1}(V_{\alpha_i})\right)
    \subseteq \bigcup_{i = 1}^n V_{\alpha_i},
  \]
  which is a finite subcover of $f(X)$. Therefore
  $f(X)$ is compact.
\end{proof}

\begin{theorem}
  Assume $\{C_j\}_{j = 1}^m$ are compact subsets
  of $X$. Then $\bigcup_{j = 1}^m C_j$ is compact.
\end{theorem}

\begin{proof}
  Assume $\bigcup_{j = 1}^m C_j \subseteq \bigcup_{\alpha \in I} U_\alpha$,
  an open cover. Observe this is also an open
  cover of $C_j$ for each $1 \le j \le m$, so
  we can extract a finite subcover, i.e. we can
  find $\alpha_{j, 1}, \dots, \alpha_{j, n_j}$
  with
  \[
    C_j \subseteq \bigcup_{i = 1}^{n_j} U_{\alpha_{j, i}}.
  \]
  Then we see that
  \[
    \bigcup_{j = 1}^m C_j
    \subseteq \bigcup_{j = 1}^m \bigcup_{i = 1}^{n_j} U_{\alpha_{j, i}},
  \]
  which is still a finite union. This is then a finite
  subcover of $\bigcup_{j = 1}^m C_j$, so
  $\bigcup_{j = 1}^m C_j$ is compact.
\end{proof}

\begin{theorem}[Weierstrass]
  Let $f : [a, b] \to \R$ be continuous. Then
  $f([a, b])$ is bounded, and moreover
  there exist $x_{\mathrm{max}}, x_{\mathrm{min}} \in [a, b]$
  such that $f(x_{\mathrm{max}}) \ge f(x) \ge f(x_{\mathrm{min}})$
  for all $x \in [a, b]$.
\end{theorem}

\begin{proof}
  Since $f$ is continuous and $[a, b]$ is compact
  (by Heine-Borel), $f([a, b]) \subseteq \R$ is
  compact. Thus by Heine-Borel, $f([a, b])$ is
  bounded. In particular, we can find $M, m$
  such that
  \[
    m \le f(x) \le M \quad \text{for all } x \in [a, b].
  \]
  For the second part, observe that $f([a, b])$
  is bounded and nonempty, so $s = \sup f([a, b])$.
  Since this is the supremum, there must exist
  $y_i \in f([a, b])$ such that $y_i \to s$
  as $i \to \infty$. Now observe that
  $f([a, b])$ is closed by Heine-Borel, and in
  particular it contains its limit points. Thus
  we obtain $s \in f([a, b])$. Then pick
  $x_{\mathrm{max}} \in f^{-1}(\{s\}) \subseteq [a, b]$,
  which will satisfy
  $f(x_{\mathrm{max}}) = s \ge f(x)$ for all $x \in [a, b]$.
  by construction.

  The argument for finding
  $x_{\mathrm{min}} \in [a, b]$ is similar.
\end{proof}

\begin{theorem}
  Let $X$ be a topological space, $K \subseteq X$
  compact, and $f : K \to \R$ a continuous function.
  Then $f$ is bounded over $K$
  and attains its minimum and maximum on $K$.
\end{theorem}

\begin{proof}
  The same argument goes through, replacing
  $[a, b]$ by the compact set $K$.
\end{proof}

  \chapter{Sept.~10 --- More Compactness}

\section{The Cantor Set}
Define $I_0 = [0, 1]$ and remove the open middle-thirds
interval to get
\[
  I_1 = [0, 1] \setminus (1 / 3, 2 / 3) = [0, 1 / 3] \cup [2 / 3, 1].
\]
Continue by removing the middle thirds of each interval
to get $I_2, I_3, \ldots$. Then the
\emph{Cantor set} is defined to be
$K = \bigcap_{I \ge 0} I_n$. The Cantor set is
compact and uncountable. See more on Homework 3.

\section{The Heine-Borel Theorem}
\begin{theorem}[Heine-Borel]
  If $C \subseteq \R$, then $C$ is compact if
  and only if $C$ is closed and bounded.
\end{theorem}

\begin{proof}
  $(\Rightarrow)$ This direction is easy, see
  Homework 3 for details.

  $(\Leftarrow)$ First we show that
  $[a, b] \subseteq \R$ is compact. Let
  $\{U_\alpha\}_{\alpha \in I}$ be an open
  cover for $[a, b]$, i.e.
  $[a, b] \subseteq \bigcup_{\alpha \in I} U_\alpha$.
  Now define
  \[
    R = \{x \in [a, b] \mid \text{$[a, x]$ has a finite subcover}\}
  \]
  Clearly $a \in R$ since
  $a \in [a, b] \subseteq \bigcup_{\alpha \in I} U_\alpha$,
  so picking any single $U_\alpha$ with $a \in U_\alpha$
  gives a finite subcover for $[a, a] = \{a\}$. The
  goal is now to show that $b \in R$. Observe that
  $a \in R$ implies $R = \varnothing$, and $R \subseteq [a, b]$,
  so
  \[
    s = \sup R
  \]
  exists by the completeness of $\R$. We proceed
  to show that $s \in R$ and then $s = b$, which
  will show that $b \in R$. As $s \in [a, b] \subseteq \bigcup_{\alpha \in I} U_\alpha$,
  we can find $\alpha_s$ such that
  $s \in U_{\alpha_s}$. Since $U_{\alpha_s}$ is
  open, we can find $\delta > 0$ such that
  $(s - \delta, s + \delta) \subseteq U_{\alpha_s}$.
  Then since $s$ is a least upper bound of $R$,
  we can find $r \in R$ such that
  $s - \delta < r \le s$. Now since $r \in R$,
  $[a, r]$ admits a finite subcover
  $\{U_{\alpha_i}\}_{i = 1}^n$. Then
  \[
    [a, s] = [a, r] \cup (s - \delta, s]
    \subseteq \left(\bigcup_{i = 1}^n U_{\alpha_i}\right) \cup U_{\alpha_s}
  \]
  is a finite subcover for $[a, s]$, so $s \in R$.
  Now observe that we actually covered
  \[
    \left[a, s + \frac{\delta}{2}\right]
    = [a, r] \cup (s - \delta, s + \delta)
    \subseteq \left(\bigcup_{i = 1}^n U_{\alpha_i}\right) \cup U_{\alpha_s}
  \]
  in the previous construction. Then
  $s + \delta / 2 \in R$, which contradicts the
  minimality of $s$ unless $s = b$. Thus
  $b \in R$, so $[a, b]$ admits a finite subcover
  and thus $[a, b]$ is compact.

  Now let $C \subseteq \R$ be an arbitrary
  closed and bounded set. Since $C$ is bounded,
  there exists $I = (a, b)$ such that $C \subseteq I$.
  But then $C \subseteq I \subseteq \overline{I} = [a, b]$, so
  $C$ is a closed subset of a compact set, hence
  compact.
\end{proof}

\begin{remark}
  The Heine-Borel theorem also holds more
  generally in $\R^n$.
  A later theorem will
  say that the product of compact sets is compact in
  the product topology,
  and thus we can run the same argument as above but
  with boxes in $\R^n$ instead of intervals.
\end{remark}

\section{The Bolzano-Weierstrass Theorem}
\begin{definition}
  A point $x$ is an \emph{accumulation point}
  for a set $S$ if for all open sets $U$ containing
  $x$, we have
  $(U \setminus \{x\}) \cap S \ne \varnothing$.
\end{definition}

\begin{remark}
  We disallow constant sequences when talking about
  accumulation points.
\end{remark}

\begin{prop}
  Let $\mathrm{Acc}(A)$ be the set of accumulation
  points of a set $A$. Then $\overline{A} = A \cup \mathrm{Acc}(A)$.
\end{prop}

\begin{proof}
  We show that $A \cup \mathrm{Acc}(A)$ is closed,
  which will imply $\overline{A} \subseteq A \cup \mathrm{Acc}(A)$
  by the minimality of the closure. Write
  \[
    (A \cup \mathrm{Acc}(A))^c = A^c \cap \mathrm{Acc}(A)^c.
  \]
  Now assume $x \in A^c \cap \mathrm{Acc}(A)^c$.
  Since $x \notin \mathrm{Acc}(A)$, there exists
  $U_x$ open such that $x \in U_x$ and
  \[(A \setminus \{x\}) \cap U_x = \varnothing.\]
  But also $x \notin A$, so $A \setminus \{x\} = A$
  and $A \cap U_x = \varnothing$. Then we can write
  \[
    (A \cup \mathrm{Acc}(A))^c
    = A^c \cap \mathrm{Acc}(A)^c
    = \bigcup_{x \in A^c \cap \mathrm{Acc}(A)^c} U_x.
  \]
  This is a union of open sets, hence open, and
  so $A \cup \mathrm{Acc}(A)$ is closed.

  For the other direction, assume $x \in A \cup \mathrm{Acc}(A)$.
  If $x \in A$, we are done, so assume
  $x \in \mathrm{Acc}(A) \setminus A$. Now assume
  otherwise that $x \notin \overline{A}$.
  Then $x \in (\overline{A})^c$, which is open.
  Set $U = (\overline{A})^c$, so that
  \[
    U \cap (A \setminus \{x\}) = U \cap A = \varnothing.
  \]
  But then this says that $x$ is not an accumulation
  point, in contradiction.
\end{proof}

\begin{definition}
  We say that a topological space $X$ is \emph{sequentially compact} if
  every bounded sequence has a convergent
  subsequence.
\end{definition}

\begin{theorem}[Bolzano-Weierstrass]
  Any bounded infinite set $S \subseteq \R^n$ has an
  accumulation point.
\end{theorem}

\begin{proof}
  Since $S$ is bounded, find a compact set containing
  $S$. Then apply the later Theorem \ref{thm:compact-bw}.
\end{proof}

\begin{remark}
  In general,
  compactness is \emph{not} equivalent to sequential
  compactness, but both imply the Bolzano-Weierstrass
  theorem. However, in many spaces (including
  metric spaces, in particular), the two notions coincide (and are
  also equivalent to the Bolzano-Weierstrass theorem).
\end{remark}

\begin{theorem}
  A sequentially compact space has the Bolzano-Weierstrass
  property, namely that any bounded infinite set has an
  accumulation point.
\end{theorem}

\begin{proof}
  This is easy, pick a countable subset
  (i.e. a sequence)
  and apply sequential compactness.
\end{proof}

\begin{theorem}
  \label{thm:compact-bw}
  A compact space has the Bolzano-Weierstrass property, namely
  that any infinite set has an accumulation point.
\end{theorem}

\begin{proof}
  Let $A$ be an infinite set in $X$, where $X$ is compact.
  Assume otherwise that $A$ has no accumulation points in $X$.
  Then there is no accumulation point for $A$
  outside of $A$, so $\mathrm{Acc}(A) \subseteq A$. This
  gives
  \[
    \overline{A} = A \cup \mathrm{Acc}(A) = A,
  \]
  so $A$ is closed. Thus $A$ is a closed subset
  of a compact space, hence compact. Now for
  any $a \in A$, pick an open set $U_a$
  such that $a \in U_a$ and
  $U_a \cap (A \setminus \{a\}) = \varnothing$.
  Write $A \subseteq \bigcup_{a \in A} U_a$, and
  by compactness we can find a finite
  subcover $A \subseteq \bigcup_{i = 1}^n U_{a_i}$.
  Then observe that
  \[
    A = A \cap \bigcup_{i = 1}^n U_{a_i}
    = \bigcup_{i = 1}^n (A \cap U_{a_i})
    = \bigcup_{i = 1}^n \{a_i\}
    = \{a_1, \ldots, a_n\},
  \]
  This is in contradiction with $A$ being infinite.
\end{proof}

\begin{remark}
  Usually, this proof goes by showing that compactness
  implies sequential compactness, which then
  implies the Bolzano-Weierstrass property.
  But this proof avoids going through convergent
  sequences.
\end{remark}

\end{document}
