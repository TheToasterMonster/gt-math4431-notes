\documentclass[12pt, letterpaper, oneside]{book}
\usepackage[margin={0.6in, 0.75in}]{geometry}
\usepackage{microtype}
% \usepackage{kpfonts}
\usepackage{amsmath, amssymb, amsthm}
\usepackage{parskip}
\usepackage[many]{tcolorbox}
\usepackage{footnote}
\usepackage{cancel}
\usepackage{titlesec}
\usepackage{pgffor}
\usepackage[shortlabels]{enumitem}
\usepackage{hyperref}

\usepackage[overload]{textcase}

\renewcommand{\chaptername}{Lecture}
\newtheorem{axiom}{Axiom}[chapter]
\newtheorem{theorem}{Theorem}[chapter]
\newtheorem{prop}{Proposition}[chapter]
\newtheorem{corollary}{Corollary}[theorem]
\newtheorem{lemma}{Lemma}[chapter]
\theoremstyle{definition}
\newtheorem{definition}{Definition}[chapter]
\newtheorem{exercise}{Exercise}[chapter]
\newtheorem{example}{Example}[definition]
\newtheorem*{remark}{Remark}

\tcbset{sharp corners, breakable, enhanced, parbox=false}
\newtcolorbox{mybox}[3][]
{
  colframe = #2!150,
  colback  = #2!5,
  coltitle = #2!0!white,  
  title    = {#3},
  #1,
}

\titleformat{\chapter}[display]
    {\normalfont\huge\bfseries}{\chaptertitlename\ \thechapter}{20pt}{\Huge}
\titlespacing*{\chapter}{0pt}{0pt}{40pt}

\newcommand{\R}{\mathbb{R}}
\newcommand{\N}{\mathbb{N}}
\newcommand{\Z}{\mathbb{Z}}
\newcommand{\C}{\mathbb{C}}
\newcommand{\Q}{\mathbb{Q}}
\newcommand{\F}{\mathbb{F}}

\newcommand{\T}{\mathcal{T}}
\newcommand{\B}{\mathcal{B}}

\DeclareMathOperator{\Vol}{Vol}
\DeclareMathOperator{\Int}{int}
\DeclareMathOperator{\area}{area}
\DeclareMathOperator{\curl}{curl}

\title{MATH 4431: Introduction to Topology}
\author{Frank Qiang\\Instructor: Asaf Katz}
\date{Georgia Institute of Technology\\Fall 2024}

\begin{document}
  \maketitle

  \begingroup
  \let\cleardoublepage\clearpage
  \tableofcontents
  \endgroup

  % \foreach \i in {00, 01, 02, 03, 04, ..., 50} {%
  %   \edef\FileName{lectures/lecture\i.tex}%     The % here are necessary to eliminate any
  %   \IfFileExists{\FileName}{%  spurious spaces that may get inserted
  %      \input{\FileName}%       at these points
  %   }
  % }
  \chapter{Aug.~20 --- Review of Metric Spaces}

\section{Metric Spaces}
Recall the definition of a \emph{metric space}:

\begin{definition}
  Given a set $X$, a function $d : X \times X \to \R$
  is called a \emph{metric} if
  \begin{enumerate}[(i)]
    \item (strong positivity)
      $d(x, y) \ge 0$ for all
      $x, y \in X$, and $d(x, y) = 0$ if and only if
      $x = y$,
    \item (symmetry) $d(x, y) = d(y, x)$,
    \item and (triangle inequality)
      $d(x, z) \le d(x, y) + d(y, z)$
      for all $x, y, z \in X$.
  \end{enumerate}
\end{definition}

\begin{example}
  For any set $X$, we can define the
  \emph{discrete metric}
  by
  \[
    d(x, y) =
    \begin{cases}
      1 & \text{if $x \ne y$}, \\
      0 & \text{otherwise}.
    \end{cases}
  \]
  Verify as an exercise that this satisfies
  the triangle inequality.
\end{example}

\begin{example}
  The Euclidean metric in $\R^n$ is
  \[
    d(\overline{x}, \overline{y})
    = \sqrt{\sum_{i = 1}^n (x_i - y_i)^2}
  \]
  where $\overline{x} = (x_1, \ldots, x_n)$
  and $\overline{y} = (y_1, \ldots, y_n)$.
\end{example}

\section{Open Sets}
\begin{definition}
  The \emph{open ball} of radius $R > 0$ around
  $x_0 \in X$ is
  \[
    B_R(x_0) = \{y \in X \mid d(x_0, y) < R\}
  .\]
  Given a set $S \subseteq X$, a point $x_0$
  is called an interior point of $S$ if there exists
  $r > 0$ such that $B_r(x_0) \subseteq S$.
  The set $S$ is called \emph{open} if all of its
  points are interior points.
\end{definition}

\begin{prop}
  The open ball $B_R(x)$ is open.
\end{prop}

\begin{proof}
  Fix an arbitrary $y \in B_R(x)$, and observe that
  it suffices to show that $y$ is an interior point.
  Take $r = R - d(x, y)$, and
  first note that $r > 0$ since
  $d(x, y) < R$. Now note that for all $z \in B_r(y)$,
  we have
  \[
    d(x, z) \le d(x, y) + d(y, z)
    < d(x, y) + (R - d(x, y))
    = R,
  \]
  so that $z \in B_R(x)$.
  Thus $B_r(y) \subseteq B_R(x)$, and so $y$ is an interior point.
\end{proof}

\begin{corollary}
  $B_R(x) = \bigcup_{y \in B_R(x)} B_{r_y}(y)$,
  where $r_y = R - d(x, y)$.
\end{corollary}

\begin{proof}
  We have $B_{r_y}(y) \subseteq B_R(x)$
  for each $y \in B_R(x)$,\footnote{Using the argument from the previous proposition.}
  and so
  $\bigcup_{y \in B_R(x)} B_{r_y}(y) \subseteq B_R(x)$.
  For the reverse inclusion simply observe that
  $y \in B_{r_y}(y) \subseteq \bigcup_{y \in B_R(x)} B_{r_y}(y)$ for
  each $y \in B_R(x)$.
\end{proof}

\begin{prop}
  In a metric space $(X, d)$, the following
  are true:
  \begin{enumerate}[(i)]
    \item $\varnothing, X$ are open,
    \item if $\{S_i\}_{i \in I}$ are open, then
      $\bigcup_{i \in I} S_i$ is open,
    \item and if $\{S_i\}_{i = 1}^n$ are open, then
      $\bigcap_{i = 1}^n S_i$ is open.
  \end{enumerate}
\end{prop}

\begin{proof}
  $(i)$ The empty set is open vacuously.
  To see that $X$ is open, simply take $R = 1$
  for any $x \in X$.

  $(ii)$ Fix $x \in \bigcup_{i \in I} S_i$
  arbitrary, so there exists $i_0 \in I$ with
  $x \in S_{i_0}$. Since $S_{i_0}$ is open,
  $x$ is an interior point and thus there exists
  $r > 0$ such that $B_r(x) \subseteq S_{i_0}$.
  But then $B_r(x) \subseteq S_{i_0} \subseteq \bigcup_{i \in I} S_i$,
  so $x$ is an interior point of
  $\bigcup_{i \in I} S_i$ also and thus
  $\bigcup_{i \in I} S_i$ is open.

  $(iii)$ Now assume $x \in \bigcap_{i = 1}^n S_i$.
  Then for each $1 \le i \le n$, there exists
  $r_i > 0$ such that $B_{r_i}(x) \subseteq S_i$.
  Then we can choose
  \[
    r = \min\{r_1, \ldots, r_n\} > 0,
  \]
  so that $B_r(x) \subseteq B_{r_i}(x) \subseteq S_i$
  for each $1 \le i \le n$. Thus
  $B_r(x) \subseteq \bigcap_{i = 1}^n S_i$
  and $\bigcap_{i = 1}^n S_i$ is open.
\end{proof}

\begin{remark}
  The above argument for the finite intersection
  property requires that there are only finitely
  many $r_i$. Otherwise it may very well be that
  $r = \inf\{r_i\} = 0$ and the argument fails.
\end{remark}

  \chapter{Aug.~22 --- Topology, Basis, Continuity}

\section{Topological Spaces}
\begin{definition}
  A \emph{topology} $\mathcal{T} \subseteq \mathcal{P}(X)$ is a
  collection of sets such that
  \begin{enumerate}[(i)]
    \item $\varnothing, X \in \mathcal{T}$,
    \item for any index set $I$, if $\{s_i\}_{i \in I} \subseteq \mathcal{T}$,
      then $\bigcup_{i \in I} s_i \in \mathcal{T}$ (closure under arbitrary union),
    \item and if $\{s_i\}_{i = 1}^n \subseteq \mathcal{T}$, then
      $\bigcap_{i = 1}^n s_i \in \mathcal{T}$ (closure under finite intersection).
  \end{enumerate}
  A set with a topology, i.e. a pair
  $(X, \mathcal{T})$,
  is called a \emph{topological space}.
  Elements of $\mathcal{T}$ are called
  \emph{open sets}.
\end{definition}

\begin{example}
  The following are examples of topologies on a set $X$:
  \begin{itemize}
    \item The trivial topology: $\mathcal{T} = \{\varnothing, X\}$.
    \item The discrete topology: $\mathcal{T} = \mathcal{P}(X)$.\footnote{Note that the discrete topology is induced by the discrete metric.}
    \item If $(X, d)$ is a metric space, then
      $\mathcal{T} = \{\text{collection of metrically open sets}\}$
      is a topology on $X$.
  \end{itemize}
\end{example}

\begin{remark}
  Not every topology is induced by a metric.
  For instance consider the trivial topology
  on $\R$.
\end{remark}

\section{Basis for a Topology}
\begin{definition}
  A collection $\mathcal{B} \subseteq \mathcal{P}(X)$
  is called a \emph{basis} if
  \begin{enumerate}[(i)]
    \item $\bigcup_{b \in \mathcal{B}} b = X$, i.e.
      $\mathcal{B}$ is a covering of $X$,
    \item and if $x \in b_1 \cap b_2$ for any
      $b_1, b_2 \in B$, then there exists $b_3 \in \mathcal{B}$
      such that $x \in b_3$
      and $b_3 \subseteq b_1 \cap b_2$.
  \end{enumerate}
\end{definition}

\begin{theorem}
  Given a set $X$ and a basis $\mathcal{B}$, define
  \[
    \mathcal{T}_\mathcal{B} = \left\{\bigcup_{i \in I} s_i \mid \text{$I$ is any index set and $\{s_i\}_{i \in I} \subseteq \mathcal{B}$}\right\}.
  \]
  Then $\mathcal{T}_\mathcal{B}$ is a topology on $X$.
\end{theorem}

\begin{proof}
  First observe that $\varnothing, X \in \mathcal{T}_\mathcal{B}$:
  Picking $I = \varnothing$ gives
  $\bigcup_{i \in I} s_i = \varnothing \in \mathcal{T}_\mathcal{B}$
  and picking $I = \mathcal{B}$ gives
  $\bigcup_{b \in \mathcal{B}} b = X \in \mathcal{T}_\mathcal{B}$
  by the covering property of a basis.

  Now assume $\{s_i\}_{i \in I} \subseteq \mathcal{T}_\mathcal{B}$.
  For each $i \in I$, we have $s_i \in \mathcal{T}_\mathcal{B}$
  and so there exists an index set $J_i$
  such that $s_i = \bigcup_{j \in J_i} b_j$, where
  the $b_j \in \mathcal{B}$. Then
  \[
    \bigcup_{i \in I} s_i = \bigcup_{i \in I} \bigcup_{j \in J_i} b_j,
  \]
  which is a union of elements of
  $\mathcal{B}$ and hence is in
  $\mathcal{T}_{\mathcal{B}}$.

  Finally assume $\{s_i\}_{i = 1}^n \subseteq \mathcal{T}_\mathcal{B}$.
  Now as each $s_i \in \mathcal{T}_\mathcal{B}$,
  there exists $J_i$ such that
  $s_i = \bigcup_{j \in J_i} b_j$.
  Then
  \[
    \bigcap_{i = 1}^n s_i = \bigcap_{i = 1}^n \bigcup_{j \in J_i} b_j.
  \]
  Now assume $x \in \bigcap_{i = 1}^n s_i = \bigcap_{i = 1}^n \bigcup_{j \in J_i} b_j$.
  For each $1 \le i \le n$, there exists
  $j_i \in J_i$ such that $x \in b_{j_i}$.
  Hence $x \in \bigcap_{i = 1}^n b_{j_i}$.
  Now by induction on the intersection property of a basis,
  we can find $b_x \in \B$ with
  \[
    x \in b_x \subseteq \bigcap_{i = 1}^n b_{j_i}
  \]
  Also observe that
  \[\bigcap_{i = 1}^n b_{j_i} \subseteq \bigcap_{i = 1}^n \bigcup_{j \in J_i} b_j
    = \bigcap_{i = 1}^n s_i
  \]
  by construction,
  so we may write
  \[
    \bigcap_{i = 1}^n s_i = \bigcup_{x \in \bigcap_{i = 1}^n s_i} b_x \in \mathcal{T}_\mathcal{B}
  \]
  as a union of elements of $\mathcal{B}$.
\end{proof}

\begin{definition}
  A \emph{subbasis} $\mathcal{B} \subseteq \mathcal{P}(X)$ is a collection
  of sets such that
  $\bigcup_{b \in \mathcal{B}} b = X$.
\end{definition}

\begin{remark}
  One may define a basis $\widetilde{\mathcal{B}}$ from a
  subbasis $\mathcal{B}$ by adding all finite intersections
  of elements of $\mathcal{B}$.
  We get the covering property for free and adding
  the finite intersections gives us the
  intersection property of a basis.
\end{remark}

\begin{example}
  For $\R$ with the Euclidean metric, the following
  are bases for the standard topology:
  \begin{itemize}
    \item $\{B_R(x) \mid x \in \R, R > 0\}$.
    \item $\{B_R(x) \mid x \in \R, R > 0, R \in \Q\}$.
      For this use the fact that $\Q$ is dense in $\R$.
  \end{itemize}
  In particular this shows that a basis for a topology
  is not unique in general.
\end{example}

\section{Continuous Functions}

\begin{definition}
  Let $(X, \mathcal{T}_X)$ and $(Y, \mathcal{T}_Y)$
  be two topological spaces. A function $f : X \to Y$
  is called \emph{continuous} if for any
  $O \in \mathcal{T}_Y$, we have
  $f^{-1}(O) \in \mathcal{T}_X$, i.e. the
  preimage of an open set is open.\footnote{Recall that $f^{-1}(O) = \{x \in X \mid f(x) \in O\}$.}
\end{definition}

\begin{example}
  Let $X$ be equipped with the trivial
  topology $\{\varnothing, X\}$ and let
  $\R$ be equipped with the standard topology.
  Then the only continuous functions $f : X \to \R$
  are the constant functions $f : x \mapsto c$
  for fixed $c \in \R$. To see this, observe that
  \begin{itemize}
    \item $x \mapsto c$ is continuous since
      any open set in $\R$ either contains $c$ or does not,
      and so the preimage is either $X$ or $\varnothing$.
    \item Suppose $f(x_1) = y_1$ and $f(x_2) = y_2$.
      Let $\epsilon = |y_1 - y_2|$ and observe that
      $x_1 \in f^{-1}(B_\epsilon(y_1))$ while
      $x_2 \notin f^{-1}(B_\epsilon(y_1))$, so
      $f^{-1}(B_\epsilon(y_1))$ is not open
      in $X$ despite $B_\epsilon(y_1)$ being open in $\R$.
  \end{itemize}
\end{example}

\begin{example}
  Let $X$ have the discrete topology
  $\mathcal{T} = \mathcal{P}(X)$ and let
  $\R$ have the standard topology.
  Then all functions $X \to \R$ are continuous
  since any preimage is a subset of $X$ and thus
  in $\mathcal{P}(X)$.
\end{example}

\begin{remark}
  In a way, the trivial topology has too few
  open sets while the discrete topology has too many.
\end{remark}

\begin{definition}
  Two topological spaces $(X, \mathcal{T}_X)$ and
  $(Y, \mathcal{T}_Y)$ are
  \emph{topologically equivalent} or
  \emph{homeomorphic} if there exists a
  bijection $f : X \to Y$ such that
  $f$ and $f^{-1}$ are continuous.
\end{definition}

\begin{remark}
  A bijective function $f$ being continuous does
  not necessarily imply that its inverse $f^{-1}$ is.
\end{remark}

\begin{example}
  Consider $(-\pi / 2, \pi / 2)$ equipped with the
  Euclidean metric. This is homeomorphic to $\R$
  equipped with the Euclidean metric.\footnote{Note that $(-\pi / 2, \pi / 2)$ is bounded while $\R$ is not.} One
  homeomorphism is given by $\tan : (-\pi / 2, \pi / 2) \to \R$.
\end{example}

\end{document}
